% Let's remember, if we're looking for a particular symbol, this link can help:
% http://detexify.kirelabs.org/classify.html

\documentclass[letterpaper,12pt]{report}
    \usepackage[margin=1in,letterpaper]{geometry}%%his shaves off default margins (which are too big)
    \usepackage{amsmath} %This lets us do more advanced math stuff -- in particular create unnumbered equations
    \usepackage{graphicx} %This lets us add figures
    \usepackage{float} %This lets us put figures smack in the middle of text, if we need to.
    \usepackage{grffile} %To prevent filenames from showing
    \usepackage{siunitx} %To put units properly, allow \SI[parse-numbers=false (if necessary)]{value}{unit}
    \usepackage{booktabs} %for fancier tables
    \usepackage{amssymb} % gives some fun symbols to play with, e.g. $\blacksquare$
    \usepackage[bottom]{footmisc} % have footnotes at bottom of page
    \usepackage{enumerate}
    \usepackage[normalem]{ulem} % allows for strikeout with \sout{}
    \usepackage{xcolor}% http://ctan.org/pkg/xcolor
    \usepackage{hyperref}% allow for links via \url{} and \href{}: http://ctan.org/pkg/hyperref
    \hypersetup{% because the red outline boxes are very bleh
      colorlinks=true,
      % allcolors=purple
      urlcolor = blue,
      linkcolor = purple
    }
    \usepackage{titlesec}
    \usepackage{enumitem} % be able to reference specific items in a list
    % \usepackage{bookmark} % maybe fixes silly refuse-to-compile-for-no-reason issues?
    %         % i.e. "File '*.out' has changed (rerunfilecheck) Rerun to get outlines right (rerunfilecheck) or use package 'bookmark'."

    \titleformat{\chapter}{\bfseries\Huge}{\thechapter.}{1.5ex}{}
    \titlespacing{\chapter}{0pt}{10pt}{30pt}


    \usepackage{textcomp} % lets \texttildelow -> ~
    \usepackage{stmaryrd} % have, e.g., \mapsfrom



    \usepackage{physics} % contains sane definitions for the different flavors of derivatives (/dv{}{}, /pdv{}{}, /fdv{}{})



    \usepackage{bbm} % allow for fancy-case of numbers

    \usepackage{xparse}
  % see https://tex.stackexchange.com/questions/29973/more-than-one-optional-argument-for-newcommand

    \NewDocumentCommand{\innerprod}{m m}{\ensuremath{\left<#1,#2\right>}}
    \NewDocumentCommand{\domain}{m}{\ensuremath{\mathbbm{#1}}}
    % \NewDocumentCommand{\norm}{O{\cdot}}{\ensuremath{\left\lVert#1\right\rVert}}
    % \NewDocumentCommand{\abs}{m}{\ensuremath{\left|#1\right|}}
    \NewDocumentCommand{\fancy}{m}{\ensuremath{\mathcal{#1}}}
    \NewDocumentCommand{\normexpanded}{}{}
    \NewDocumentCommand{\gcf}{m}{\ensuremath{\text{gcf}\left(#1\right)}}
    % \NewDocumentCommand{\grad}{m}{\ensuremath{\nabla #1}}
    \NewDocumentCommand{\definedas}{}{:=}
    \NewDocumentCommand{\setequalto}{}{:=}
    \NewDocumentCommand{\suchthat}{}{\mid}

    \DeclareMathOperator*{\argmax}{argmax}
    \DeclareMathOperator*{\argmin}{argmin}


    \usepackage{makeidx} % allows for indexing
    \makeindex



% from pandoc conversion (from markdown to LaTeX)
\providecommand{\tightlist}{%
  \setlength{\itemsep}{0pt}\setlength{\parskip}{0pt}}

% from jupyter notebook conversion (I can agree with the font choice)
  \usepackage[T1]{fontenc}
  % Nicer default font (+ math font) than Computer Modern for most use cases
  \usepackage{mathpazo}


\begin{document}

% commands




% Document parameters
\title{A quick(ish) reference for some concepts whose
 intuitions/simple methods of understanding sometimes escape me.}
\author{David Khachatrian}
\date{Ongoing.}

\maketitle

\newpage


\tableofcontents

\newpage


\chapter{Information Theory}\label{information-theory}

\section{Entropy and Cross-Entropy}\label{entropy-and-cross-entropy}

There are a number of ways to think about/get to the Shannon entropy
(\(H\)). One simple way is to say what we want
it to mean qualitatively and impose some desired characteristics to
determine its mathematical formulation. In this case, it may make more
sense to start with cross-entropy first.

Say you're sending me messages from an alphabet of symbols \(A\). Some
are more likely than others, and we'll call the probability distribution
associated with the actual generator of symbols \(p\). I'm over here on
the other side thinking like I'm a pair of smartypants that
has figured out how likely you are to send each symbol to me. We'll call
my expected distribution (which may or may not be the correct
distribution) \(q\).

Now we want to measure how surprised I am by any given message you send
me -\/- let's call it \(\tau\).\footnote{
    Rhymes with ``wow''. 
    Alas, not standard notation for a measure of surprise.
    }
We would imagine the following properties would be useful for \(\tau\) to have:
\begin{enumerate}
  % \def\labelenumi{\arabic{enumi}.}
  \tightlist
  \item
    The more surprised I am, the larger
    \(\tau\) gets (otherwise, it wouldn't exactly be doing a good job
    measuring my surprise) 
  \item
    If symbols are independently generated, we can
    construct the total surprise of my message by adding the total surprise
    of each symbol in the message, i.e.,
    \(\tau_{M} = \sum_{i=1}^{m} \tau_{M[i]}\) where \(M \in A^m\) is a
    string of symbols, \(m\) is the cardinality of \(M\), and \(M[i]\)
    denotes the \(i\)'th element of \(M\).
\end{enumerate}


Using these two criteria, a reasonable mapping is

\[\tau_{a \sim q} = \log\left(\frac{1}{q(a)}\right), a \in A\]

where \(q(a)\) is the probability I think you'll send the symbol \(a\).

I mean, that's great and all, but that works for individual instances of
strings or symbols. How much should I \emph{expect} to be surprised by
any given symbol? Well, that'd depend on how often you \emph{actually}
send me a symbol, alongside how often I expect to receive that
particular symbol.

Hmm, this smells of expectation! And indeed, that's all we do -\/- take
the expectation over \(A\) (using the \emph{actual} probability of each
symbol occurring, i.e., using \(p\)) of my surprise per symbol:

\[E_{a \sim p}[\tau_{a \sim q}] = \sum_{a \in A} p_a \log\left(\frac{1}{q(a)}\right) = H(p,q)\]

We denote this metric \(H(p,q)\) (where \(p\) and \(q\) are the actual
and presumed distributions, respectively) and call it the
\textbf{cross-entropy} \index{cross-entropy}
(because that sounds pretty cyberpunk to me. I
like to think they throw in ``cross'' because it measures the surprise
caused by ``crossing'' the distributions \(p\) and \(q\) together.)

Now, we'd imagine that I'd be the least surprised if my presumed
distribution of symbols were in fact the actual distribution, i.e. when
\(p = q\). In fact, this is true!
\[H(p,p) = H(p) = \sum_{a \in A} p_a \log\left(\frac{1}{p_a}\right)\]

is called the \textbf{entropy} (or \textbf{Shannon entropy})
\index{entropy (information theory)!for distributions}
of the
probability distribution and written \(H\).\footnote
{
  You know, it'd arguably make more sense for the 
  \emph{\textbf{S}hannon} entropy to be denoted \(S\),
  which would also be a happy notational coincidence 
  with the symbol used for entropy in most other fields. 
  Instead, we have a notational collision with \emph{enthalpy}.
  Ah well, such is the arbitrariness of a symbol's meaning. 
  (I suppose it's fitting.)
}
Usually, the \(log\)s written above are base-2. This permits a way of
thinking of the value of the Shannon entropy: if I'm only allowed to ask
the same series of questions to you to figure out which symbol you want
to send me and I know the actual probability distribution \(p\) of
symbols, how many questions should I expect to ask (i.e. mean/average)
before I figure out the answer? The cross-entropy is the same thing,
expect I don't necessarily know the actual probability distribution
\(p\) of symbols, I just think I do (and I think it's \(q\)) and base my
series of questions based on \(q\).

\section{KL divergence and mutual information.}\label{kl-divergence-and-mutual-information}


Now, hopefully that makes it clear that \( p \neq q \implies H(p,q)
> H(p) \) --- if I don't know the actual distribution, I'm
not going to be able to answer the most optimal series of questions.
This suggestions to us a notion of ``distance'' (i.e. a metric) between
the probability distributions \(p\) and \(q\):
\[ KL(p \mid\mid q) = H(p,q) - H(p) \]

This metric is called the \textbf{Kullback--Leibler divergence}
\index{KL divergence} \index{relative entropy (information theory)}
(because
names) or the \textbf{KL divergence} (because initialisms), or seemingly
most rarely but probably most clearly the \textbf{relative entropy}. Out
loud, you'd say \(KL(p \mid\mid q)\) is ``the {[}blah{]} of p with
respect to q". The closer the KL divergence is to 0, the closer \(q\) is
to being \(p\), and \( KL(p \mid\mid q) = 0 \implies p = q \) at every
point in the domain of \(q\) (which is the same as the domain of \(p\)). 
(From the above formula, 
it should be clear that the KL divergence is not symmetric. The
first argument is the ``correct'' distribution and we're measuring how suboptimal
the second argument/distribution is at replicating the first one.)

This makes describing the
\textbf{mutual information between two random variables \(X\) and \(Y\)}
\index{mutual information (information theory)}
in terms of a KL divergence fairly intuitive. Just
running off the name, if X and Y were independent, you'd expect no
mutual information between them --- 
observing one variable wouldn't tell you anything about the other variable, you don't gain
a lot of information about one from the other. 
In such a case, we know something about their probability distributions,
namely that they're independent, i.e., \(P(X,Y) = P(X)P(Y)\), which implies 
\( KL\left(P(X,Y)\mid\mid P(X)P(Y)\right) = 0\). 
And in fact, one way of writing the mutual information 
between random variables X and Y is exactly

\[ I(X;Y) = KL\left(P(X,Y)\mid\mid P(X)P(Y)\right) \]
\\
which I think is pretty neat.
\\
\\
- DK (4/24/18)

\section{Entropy of variables vs. entropy of distributions}\label{entropy-of-variables-vs-entropy-of-distributions}

Now, there is another way of approaching mutual information by 
defining a conditional entropy between two random variables 
(itself requiring a defintion of joint entropy in order to put
the equation in a ``plug n' chug'' form). This raises some questions, the
first and most pressing of which being ``\emph{Variable?} We've been talking
about distributions this whole time!''\footnote
{
  More an outcry than anything, but still appropriate.
} 
, and the answer to which is ``Yes, variable.'' Since we've been 
focusing on distributions this whole time, the switch to a random variable \(X\) is
actually fairly straightforward -- we use the distribution from which \(X\) is drawn.

More explicitly, say \(X\) is drawn from a probability distribution \(p\). Then the
\textbf{entropy of \(X\)} is
\index{entropy (information theory)!for random variables}

\[ H(X) = \sum_{x \in X} p(x) \log\left(\frac{1}{p(x)}\right) \]

This may be reminiscent of our \(\tau_{a \sim q}\) from earlier, except that now we're using
the actual distribution of \(X\) (which is \(p\)), so it'd be \(\tau_{x \sim p}\). 

We can sort of think of it like this: the information entropy only really makes sense
for probability distributions. So, if we want to figure out how hard it 
is to encode a random variable \(X\),
we kind of ``pull out'' the probability distribution that \(X\) is drawn from and
use that in our formula.
\\
Now, unfortunately, the notation gets muddy when we allow ourselves to shove in
these random variables as arguments. For example, what does \(H(X,Y)\) (X and Y being
random variables) mean? H is provided two arguments -- is it the cross-entropy between
the underlying distributions of X and Y? Nope, it's the \textbf{joint entropy} of X and Y,
i.e., \index{joint entropy}

\[ H(X,Y) = H(p)\]

where \(p\) is the joint distribution of the random variables \(X\) and \(Y\).
Our main defense against confusing the two formulae is that random variables are
(normally -- hopefully!) denoted by capital letters while distributions are usually
denoted by lowercase letters.
\\
\\
You may wonder ``what's the point?'' Well, this begins to allow information theory analogues
for intuitions on random variables gleaned from statistics. The main missing piece at this point
is the \textbf{conditional entropy} \index{entropy (information theory)!conditional}
of a random variable Y with respect to X.
We'd expect that we could relate the joint entropy and conditional entropy of random variables
with one another, like how we can do so with the joint and conditional probability
distributions of the variables, especially since we've defined the entropy of a random
variable in terms of its probability distribution. And since:
\begin{enumerate}
  \tightlist
  \item 
    we applied a logarithm to the probability distribution (and took the expected value) when defining the entropy of a random variable; and
  \item
    the relationship between conditional and joint probability depends on multiplication:
    \[P(Y|X) \times P(X) = P(X,Y) \]
\end{enumerate}
we'd want the relationship between conditional and joint entropy to hold via addition:
\[ H(Y|X) + H(X) = H(X,Y) \]

In fact, that's exactly the case! You can derive the formula 
for conditional entropy based on the above relationship.
And just to circle back to information gain, we'd expect that we might gain some information
about a random variable X when we observe the random variable Y, depending on how the joint
distribution P(X,Y) compares with P(X). Earlier, we defined the mutual information in terms
of a KL divergence:
\[ I(X;Y) = KL\left(P(X,Y)\mid\mid P(X)P(Y)\right) \]

but we can also capture the quantity of ``how much does knowing Y (on average) help us figure
out X (and vice-versa)?'' using the language of entropies of random variables:
\[ I(X;Y) = H(X) - H(X|Y) \]

This is the expected amount of information gained 
by knowing the state of the random variable Y. 
Specific values of Y may give more information
about X -- may lead to a much tighter conditional distribution -- than others.
Then, you can look at it for specific cases, i.e., the
\textbf{information gain} \index{information gain}
about X from observing a specific state y for Y,
IG(X,Y = y), which is related to the mutual information via
\(E_{y \sim Y}\left[IG(X,Y)\right] = I(X;Y)\)).
So if we observe that Y took on the
value y, the information gain for X would be
\[ IG(X, Y = y) = KL\left(P(X,Y)\mid\mid P(X | Y = y)\right) \]

Writing the above in terms of information entropy would be
\[ IG(X, Y = y) = H(X) - H(X|Y = y) \]


where \(H(X|Y = y)\) is an expectation over the conditional distribution 
\(P(X | Y = y)\):
\[
  H(X| Y = y) = \sum_{x \in X} p(x|y) \times \log\left(\frac{1}{p(x|y)}\right)
\]

The main takeaway is that \emph{defining the entropy of a random variable as it is above
allows for the migrations of intuitions about random variables
from statistics and probability.} A worthy cause!\footnote{The notational confusion is still unfortunate though.}
\\
\\
- DK, 5/14/18

\newpage
\chapter{Statistics}\label{statistics}
\section{Bayes' Theorem}\label{bayes-theorem}

I can never seem to remember Bayes' Theorem directly, as they write it
out in textbooks. It makes so much more sense to me to think about it
from the relationships between conditional and joint
probabilites/distributions, and one of the common tricks to make Bayes'
Theorem useful in practice also comes to me far more easily when
explicitly thinking about events as being sampled from a \emph{sample
space} of possibilities/outcomes.

Consider two possible events A and B. Let's keep in mind that A is just
one possible outcome out of a set of possibilities, as is B; we'll say
\(\alpha\) is the set of possibilities from which \(A\) was drawn and
\(\beta\) is the set of possibilities from which \(B\) was drawn, i.e.
\(A \sim \alpha\) and \(B \sim \beta\) (this will be good to remember
later). Now:

\[ \text{Pr[A and B both occur]} := P(A,B) \]

Assuming individual events happen separately, there are two ways for
both A and B to occur: 
\begin{enumerate}
  \tightlist
    \item
      A happens first, then B happens. 
    \item 
      B happens first, then A happens.
\end{enumerate}
(``Duh'', I know.)
\\
Keeping in mind that the first event might affect the probability of the
second event occurring (i.e. remembering that conditional probabilities
exist), we can write:

\[ P(A,B) = P(A) \times P(B \mid A) = P(B) \times P(A \mid B) \]

And then it's simple to write out Bayes' Theorem as it's often written
(we'll write it perhaps a bit more evocatively):

\[ P(B) \times P(A \mid B) = P(A) \times P(B \mid A) \]

\[\begin{split} P(A \mid B) &= \frac{P(A) \times P(B \mid A)} {P(B)} \\
                        &= P(A) \times \frac {P(B \mid A)} {P(B)} \end{split}\]

Using the Bayesian interpretation: At first we thought the probability
that \(A\) occurs is \(P(A)\). After we saw that \(B\) happened, we
re-evaluate the probability that \(A\) occurs with a scaling factor \(
\frac {P(B \mid A)} {P(B)} \), which answers the following question: considering I've
seen \(B\) occur, how much \emph{more} likely did I observe \(B\) due to
\(A\) also being the case (the numerator \(P(B \mid A)\)), versus my having observed
\(B\) just because of how common/rare it is (the denominator, \(P(B)\))? To see
why this scaling factor makes sense, let's consider some edge cases:

\begin{itemize}
  \tightlist
  \item 
    \emph{A and B are uncorrelated}: Then observing B is irrelevant when it comes
    to predicting A. So our scaling factor should be 1. And in fact, the lack of 
    correlation implies \(P(B \mid A) = P(B) \implies \frac {P(B \mid A)} {P(B)} = 1\).
  \item
    \emph{A precludes B}: Then via contrapositivity, if we saw B, A must not be the case.
    So our scaling factor should be 0. And since A precludes B,
    \(P(B \mid A) = 0 \implies \frac {P(B \mid A)} {P(B)} = 0\).
  \item
    \emph{B implies A (and no other outcome from \(\alpha\))}:
    Then the total probability
    must become 1, and the scaling factor must come out to be \(\frac {1} {P(A)}\).
    We'll come back to this in a bit.
\end{itemize}

Also worth knowing the fancy terminology: 
\begin{enumerate}
  \tightlist
  \item
    the \textbf{\textit{a priori} probability} or just the \textbf{prior} 
    \index{prior (Bayesian statistics)}
    is what 
    we thought would be the
    probability that a random variable takes on a certain value before we
    observed anything. So the \emph{a priori} probability (or just prior)
    for the event \(A\) would be \(P(A)\). If we consider \(A\) to be a
    random variable instead of an event, we're guessing the distribution of
    \(A\) and so \(P(A)\) would be an \textbf{\textit{a priori} distribution}
    (or again, just the prior). 
  \item
    the \textbf{\textit{a posteriori}
    probability} or just the \textbf{posterior}
    \index{posterior (Bayesian statistics)}
    is what we think the
    probability that a random variable takes on a certain value is after
    observing something. In this case, the \emph{a posteriori} probability
    (or just posterior) of the event \(A\) after observing \(B\) is
    \(P(A \mid B)\). If we consider \(A\) to be a random variable instead of
    an event, we're guessing the distribution of \(A\) after observing a
    random variable/event B and so \(P(A \mid B)\) would be an
    \textbf{\textit{a posteriori} distribution}, (or again, just the
    posterior).
\end{enumerate}

\subsection{Substitution to the rescue.}\label{subsitution-to-the-rescue}
Now, in cases of inference via supervised learning, 
we have some data on the probability of one of the observable variables
\textemdash{} let's say we observe variables \(A \sim \alpha\) \textemdash{}
and the labels/targets for the observed variables (let's say \(B ~ \beta\)).
Then we can approximate \(P(B \mid A\)) and \(P(A)\) via empirical counts \textemdash{}
and would want to fit a continuous function, e.g. a Gaussian, to \(P(A)\) to
handle out-of-sample feature combinations.
So we've already guessed some prior \(P(A)\), and we're trying to improve it by
calculating the posterior \( P(A \mid B) \). But what if we don't know
\( P(B) \)? Do we need to also guess a function for \( P(B) \)? Well, that would
involve another outside assumption which may or may not be true, and usually we want
to make as few assumptions as possible and
``let the data speak for itself''.
So then is our guessing and data collection all for naught!?
Thankfully, not so! The answer lies right under our noses -\/- or in
this case, in our previous calculations.

Consider \(P(A,B)\) again. What would we get if we added \(P(A,B)\) over
all possible values of A? (Remember we said that \(A \sim \alpha\), so A
could have been some other event within the set \(\alpha\).) That's
basically just saying that we don't care what value \(A\) is, so we end
up with \(P(B)\)!\footnote{
  `!' used to denote excitement, not factorialization.
  }
And conveniently, we'd already have a way to estimate these values:

\[\begin{split} P(B) &= \sum_{A \in \alpha} P(A,B) \\
                     &= \sum_{A \in \alpha} P(A) \times P(B \mid A)          \end{split}\]

Our summand is the same as the values we've estimated either by guessing
(\(P(A)\)) or from our data (\(P(B \mid A)\))! With that, we can rewrite
our earlier equation as

\[ P(A \mid B) = 
  \frac{P(A) \times P(B \mid A)} {\sum_{A' \in \alpha} P(A') \times P(B \mid A')} \]

With that, we can crunch the numbers and perform Bayesian inference like
a champ or have a machine do it for us like a prudent delegator.
\\
\\
Returning to our discussion of our scaling factor, we needed to show that if
we're trying to predict \(P(A \mid B)\) via
\[P(A \mid B) = P(A) \times \frac {P(B \mid A)} {P(B)} \]
and B implies A (and nothing else from \(\alpha\), then the right-hand side should equal 1.

Expanding P(B), we get
\[ P(A \mid B) = 
  \frac{P(A) \times P(B \mid A)} {\sum_{A' \in \alpha} P(A') \times P(B \mid A')} \]

Via our implication, we have that \(P(B \mid A') = 0 \forall A' \neq A\).
So the denominator simplifies to just \(P(A) \times P(B \mid A)\), which is exactly the
numerator! So then \(\left(B \implies A\right) \implies P(A \mid B) = 1\) (which implies
\( \frac {P(B \mid A)} {P(B)} = \frac {1} {P(A)} \)), as expected.
% Though
% for our everyday activities, we often don't have the luxury of having
% someone/something checking our heuristics. So it's always worth trying
% to keep in mind that oftentimes, many different factors that you may not
% know or take into consideration can culminate in observations that
% surprise you -\/- there's a good reason you aren't told to constantly
% get yourself tested for a medical condition if you don't believe to be
% at risk!
\\
\\
- DK (4/24/18)

\newpage


\chapter{Linear Algebra}\label{linear-algebra}


\section{Integral transforms.}

While this certainly won't answer all questions about integral transforms, it can hopefully
shed some light on ``where'' integral transforms come from.\par
tl;dr: they can be thought of as changes of basis in a particular function space,
with the basis vectors chosen based on what best fits the problem at hand.
However, our transformation matrix (the kernel) is necessarily ``infinite-dimensional''
in order to span the function space, so finding the inverse transformation matrix
(and therefore the inverse (integral) transform) is nontrivial.
% Like how a transformation on a vector \(T(x)\) can be modeled as matrix-vector multiplication
% \(T_Mx\), the integral transform is like a continuous, infinite-dimensional matrix-vector
% multiplication (with the function we're transforming as the vector and the
% kernel of the transform being the matrix). Just as not all ``regular'' transformations are
% invertible (the matrix representation needs to be non-singular), 
% not all transforms have an inverse operation
% (the kernel, our ``infinite-dimensional matrix'', may be singular),
% and discovering the inverse transform is generally nontrivial.

\subsection{Short refresher on vector spaces.}

(In the following discussion,
may be lazy and drop the \(\vec{\cdot}\) arrow where we believe the fact
that it's a vector is clear.)

We're already pretty comfortable with the idea of describing vectors as a linear combination
of basis vectors. If we have some vector space \(V \subseteq \mathbb{R}^n\),
% some \(\vec{v} := \left[v_1, v_2, \cdots, v_n\right]^T \in V\),
some \(\vec{v} \in V\),
and
a basis \(\{\vec{b}_1, \vec{b}_2, \cdots, \vec{b}_n\}\) for \(V\), we can write
\(v\) as a linear combination of basis vectors:

\[\vec{v} = \sum_{i=1}^{n} a_i\vec{b}_i \]

% or, if we build a matrix out of the basis vectors via 
% \(D = 
% \begin{bmatrix}
%   e_1 & e_2 & \dots & e_n 
% \end{bmatrix}
% \)
% and stack our coefficients into a vectors \(a = \left[a_1, a_2, \cdots, a_n\right]\),
% then we can write

% \[\vec{v} = aB\]

Let's emphasize the fact that we chose \emph{a} basis for \(V\)
and in most cases there is no such thing as \emph{the} basis for \(V\), since a
\textbf{basis}\index{basis!in linear algebra} (of a vector space \(V\))
is just a set of linearly independent vectors that span \(V\).
(We'll refer back to linear independence in a bit.)\par

Going back to \(\vec{v}\), if we agree of the basis of \(V\) we're working with,
we can fully ``encode'' \(v\) via an n-tuple where the \(i\)'th location holds \(a_i\),
the scalar coefficient by which we multiply the basis vector \(\vec{b}_i\).
In that sense, we can say that \(\vec{v} = \left[a_1, a_2, \cdots, a_n\right]\).
But we can always choose some \emph{other} basis
\(\{\vec{\beta}_1, \vec{\beta}_2, \cdots, \vec{\beta}_n\}\)
with which to express \(v\), in which case we'd (almost certainly) need different coefficients
\(\alpha_1, \alpha_2, \cdots, \alpha_n\) in order construct the same vector.
We haven't changed the \emph{vector spaces}, we've only changed how we represent points
\emph{in} our vector space.\par

\subsubsection*{Tangent: The ``meaning'' of vectors.}

So, what are we doing when we change bases? What does a vector of
\(\left[1, 0, 0, \cdots, 0\right]\) mean anyway?
Arguably, the answer can be given in a ``hippie''-sort of way:
\textit{``It means, like, whatever you want it to, man\texttildelow''}\par

As an example, think of how one can solve a linear system of equations,
say of three variables x, y, and z,
using a matrix.
Often, you'd represent an equation by a row vector comprised of
the \emph{coefficients} of the three variables,
augmented by the scalar value they equal. That implies that
a left-hand-side row vector \([1, 0, 0]\) ``maps to'' the expression \(x\),
\([1,0,0] \mapsto x\). That's exactly why you know you're done when you have a 
diagonal of ones on the left-hand-side of the augmented matrix, for example the row
\(
\left[
\begin{array}{ccc|c}
1 & 0 & 0 & a
\end{array}
\right]
\)
would mean that \(x=a\).
But if there's some better way to encode the equations, we could always choose
one of those instead, e.g.
\([1,0,0] \mapsto \frac{x+y}{2}\), 
\([0,1,0] \mapsto \frac{y+z}{2}\),
\([0,0,1] \mapsto \frac{x+z}{2}\).
It's all in \emph{your} hands \texttildelow\par

\subsubsection*{Back to bases, and inner products.}

All that said, when we change our basis from
\(\{\vec{b}_1, \vec{b}_2, \cdots, \vec{b}_n\}\)
to 
\(\{\vec{\beta}_1, \vec{\beta}_2, \cdots, \vec{\beta}_n\}\),
we can think of it as changing our mapping of basis vectors from 
\(e_i \mapsto b_i\)
to
\(e_i \mapsto \beta_i\)
where \(e_i\) is our standard basis vector for the \(i\)'th coordinate, 
\(e_i \in V, \, e_i[j] = \delta(i,j)\).
\par

But that's only part of the story. How do we describe \(\vec{v}\)
with our new basis? That is to say, how do we find the mapping for our coefficients
\(\{a_1, a_2, \cdots, a_n\} \mapsto \{\alpha_1, \alpha_2, \cdots, \alpha_n\}\)
so that they ``encode'' the same point \(\vec{v}\) in the vector space, i.e. that
\[\sum_{i=1}^{n}\alpha_i\vec{\beta}_i \]
and
\[\sum_{i=1}^{n}a_i\vec{b}_i\]
refer to the same point in the vector space?\par

Well, so far we have no way of measuring ``how much'' of one vector is in another.
To make this clearer, consider two vectors \(\vec{v}\) and \(\vec{w}\), both in \(V\).
We want to create an operation that allows us to describe \(v\) as an addition of two vectors,
one ``parallel'' to \(w\) and one ``orthogonal'' to \(w\)
(both in the intuitive senses of the word):

\[\vec{v} = \vec{v}_{v\parallel w} + \vec{v}_{v\perp w}\]
 

If we have an operation that gets us one of the two component vectors, we can
always define the other as a subtraction from the resultant vector.
Turns out, we focus on similarity more (what a pleasant thought)!
More specifically, we need to 
define an \textbf{inner product} \innerprod{} over the vector space \(V\),
which is just an operation that that takes in two elements from \(V\) and returns a scalar,
and which also follows four key properties:
\begin{itemize}
  \tightlist
  \item
    Symmetry: \innerprod{u}{v} = \innerprod{v}{u}
  \item
    Linearity: \(\alpha \in \domain{R} 
          \rightarrow \innerprod{\alpha u}{v} = \alpha \innerprod{u}{v}\)
  \item
    Distributivity:
      \(\innerprod{u+v}{w} = \innerprod{u}{w} + \innerprod{v}{w}\)
  \item
    Positive Definiteness:
      \(\innerprod{v}{v} \geq 0. \innerprod{v}{v} = 0 \rightarrow v = \vec{0}\)
\end{itemize}

A vector space with an associated inner product is called an \textbf{inner product space}.
We can choose \emph{any} operation that fulfills these four requirements to be our
definition of an inner product! There are some standard ones though.
In the case of vectors in \(\domain{R^n}\), the standard is
\(\innerprod{u}{v} = \sum_{i=1}^{n}u_i v_i \).
With our definition of an inner product in hand, we can have our \textbf{projection}\index{projection}

\begin{equation}
\vec{v}_{v\parallel w} = \frac{\innerprod{v}{w}}{\norm{w}} \hat{w}
\end{equation}\label{equation:projection}

where \(\hat{w}\) is the unit vector in the direction of \(w\) and \(\norm{w}\)
is the
\textbf{norm} (which we'll discuss in a moment)
of the vector \(w\).
The remaining component \(\vec{v}_{v\perp w}\) is \textbf{orthogonal} to \(w\),
meaning \(\innerprod{\vec{v}_{v\perp w}}{w} = 0\).
\par 

Conveniently, if \(\vec{w}\) is already of unit magnitude, 
then \(\norm{w} = 1, \hat{w} = \vec{w}\), and
\(\vec{v}_{v\parallel w} = \innerprod{v}{w}\vec{w}\).
That is, the ``amount/number'' of a 
unit vector \(w\) inside \(v\) is simply \innerprod{u}{v}.
And if the unit vector \(w\) is a basis vector \(b_i\), then 
\(\vec{v}_{v\parallel w}\) is 
that basis vector, scaled by \innerprod{v}{b_i}!\footnote{
  ``!'' used to denote excitement, not factorialization.
}
\par

Now we can finally answer our question! 
If we want to describe the same vector \(\vec{v}\)
in terms of a new basis \(\{\vec{\beta}_i\}\) 
we can calculate the coefficients \(\{\alpha_i\}\)
simply by calculating \emph{the projection of \(\vec{v}\)
onto each of the basis vectors \(\vec{\beta_i}\)}
\(\frac{\innerprod{v}{\beta_i}}{\norm{\beta_i}}\)
(with both \(v\) and \(\{\beta_i\}\) represented
in terms of a common basis
\textemdash{} presumably the natural basis \(\{e_i\}\) \textemdash{}
as needed in order to calculate the inner product).

\subsubsection*{Wait, what's a norm?}

A \textbf{norm}\index{norm} is a measure of the ``size''
of a vector in a vector space.
One of the most common norms seen
in linear algebra is
the \(L^2\) norm 
(the \(L\) may be named after \textbf{L}ebesgue), 
given in the discrete form by
\[\norm{w}_2 := \left(\sum_{i} \abs{w_i}^2\right)^{1/2}\]

This corresponds to the notion of
length/distance that is physically intuitive to us,
i.e. the Euclidean distance (where absolute values 
\(\abs{\cdot}\) around the inputs
are usually not explicitly written because each term
is raised to an even power and so are not strictly necessary).
Most likely, if no explict definition of a norm is given,
the \(L^2\) norm is probably implied.\par
We can generalize quite readily to the \(L^p\) norm:
\[\norm{w}_p := \left(\sum_{i} \abs{w_i}^p\right)^{1/p}\]

The above works for discrete vectors. 
For continuous vectors (which we'll be thinking about in a bit),
we can adapt our definition of a norm
by switching from summation to
integration (let's say \(w(n)\) gives the coefficient at
``index'' \(n\)):
\[\norm{w}_p = \left(\int_{n \in Dom(n)} \abs{w(n)}^p dn\right)^{1/p} \]

We'll see this continuous form return when considering
the norms of continuous functions. (It's almost as if
functions are vectors or something...)

\subsection{Functions, function spaces.}

Now what if I told you that a function is just a vector?
More accurately, a function \emph{can be viewed}
as an element of a \textbf{function space} (which needs qualification \textemdash{}
e.g. ``the space of all continuous functions'' \(\domain{C}^0\)), 
with specific coefficients
based on the choice of \emph{basis functions} we use to span the function space in question.

\subsubsection{Power series and discrete function spaces.}


As a stepping stone, let's consider an n'th-degree power series:

\[p(x;n) = \sum_{i=0}^{n}a_n x^n \]

We're looking at a weighted sum of monomials. 
What does this remind us of?\par

Well, it kinda looks like a linear combination of vectors, doesn't it?\par

To show the mapping, we can map the natural basis vector:
\[\vec{e}_i \mapsto x^i \]

If we agree to this basis,
our function \(p(x;n)\) is fully described by the coefficients \(a_i\):

\[p(x;n) = \left[a_1, a_2, \cdots, a_n\right]\]

So we have \(p(x;n)\) as a vector
in the vector space of all polynomials
of degree no greater than \(n\).
Let's call this vector space of functions
(or \emph{function space}) \(P^n\).
Then \(p(x;n) \in P^n\).
\par


But what's stopping us from
letting \(n \rightarrow \infty\)?
Let's let loose.
If we do, we get the normal, ``full'' power series\footnote{
A quick aside on \textbf{power series}\index{power series}:
when dealing with the ``full'' (infinite-order) power series,
we should consider radii of convergence if we want our 
power series to represent a desired fucntion.
Many vectors \(\{a_i\}\) map to functions that diverge
pretty much anywhere besides \(x=0\). 
This may sound exotic, but it pops up in ``mundane'' functions.
The seemingly innocuous vector 
\(\left[1, 1, 1, \cdots\right]\) corresponds
to \(f(x) = \frac{1}{1-x}\) only within a radius of
convergence of \(\abs{x} < 1\) 
\textemdash{} outside that radius, the function explodes. If we instead had a vector
\(\left[1, 2, 3, \cdots \right]\), and our function will
blow up quite quickly for \(x=0\).\par

But we needn't always care whether or not the function it
represents converges if all that interests us 
are the monomials' coefficients. 
When we're viewing a power series just as a sequence of
coefficients 
(with operations that reflect
polynomial arithmetic), we refer to it as a
\textbf{formal power series}
\index{power series!formal power series}.
The way polynomial multiplication works allows us to
solve potentially really tricky problems by creating a
(finite or infinite) power series with the appropriate
coefficients, exponentiating the power series,
and reading off the coefficient of a particular monomial.
More on that in \ref{sec:generating-functions}.
}:

\[p(x) = \sum_{i=0}^{\infty}a_n x^n\]


When we start playing with infinity, we have to start
being a bit more clever.
Our previously used ``vector as an \(n\)-tuple''
representation of a function \(f\) becomes a \emph{bit}
unreasonable. Even if we've agreed on our basis mapping
\(\{e_i \mapsto b_i(x)\}\)
and we know exactly what the coefficient \(a_i\)
for the \(i\)'th basis vector is for all \(i \in \domain{N}\),
we can't literally sit here and list them all out manually:
\(\vec{f} = \left[a_0, a_1, a_2, \cdots \right]\). \par

But how about this? 
Instead of encoding \(f\) into our vector space as a \emph{tuple}
\(f \mapsto \vec{f} = \left[a_0, a_1, \cdots \right]\), 
how about we encode 
\(f\) as \emph{another function \(A_f(k)\) which
outputs the appropriate coefficient for the \(k\)'th index of
our (infinite-dimensional) vector}?
Then if we ever need the value at the \(i\)'th index of our
encoding, instead of indexing a tuple (\(\vec{f}[i]\)),
we just call our coefficient-encoding function \(A_f(i)\).
As long as there \emph{is} an underlying function that
can give us \(A_f(k)\), we're good. Let's come back to that 
in a moment.\par

\subsection{An inner product for our function space.}

Currently we have a vector space.
But we \emph{don't} yet have an inner product space.
What would be a sensible inner product,
a sensible measure of (norm-sensitive) ``similarity''
between functions?\par

Well, these are functions over a variable \(x\).
Let's hearken back to ye olde days of first-year calculus.
Back then, we would construct Taylor-series approximations \(T\) 
of some original function \(f\) around
a point \(x=c\) and proclaim that \(T\) is ``similar'' to \(f\)
at and in some neighborhood around that point \(c\). 
Why? (Let's say \(T\) was a \(k\)'th-order approximation.)
Well, because we specifically
constructed \(T\) so that
\[\forall \ i \in \{0, 1, \cdots, k\}, T^{(i)}(c) = f^{(i)}(c)\]
\textemdash{} i.e., so that \(T\) matched \(f\)'s slope, concavity, etc., around \(x=c\)
\textemdash{}
we expect the \emph{output} of T to be approximately the same
as the \emph{output} of f around \(x=c\), even for wacky 
original functions \(f\).\footnote{
  This breaks down with ``pathological''
  functions such as
  \(f(x) = 
  \begin{cases}
    0 & x = 0 \\
    e^{1/x^2} & x \neq 0
  \end{cases}
  \),
  which has \(f^{(i)}(0) = 0 \ \forall i \in \domain{N}\)
  and therefore has a Taylor series of exactly \(T(x) = 0\)
  if centered around \(c = 0\) \textemdash{} hardly how \(f\)
  acts outside the origin!
  But in this situation we can only grumble about these sorts of functions,
  called \emph{non-analytic functions},
  and explicitly exclude them from our analysis.
}
All this to say, in choosing the inner product for functions
in a particular space,
we probably want to measure the ``responses'' of these
functions to the input variable(s) they permit.
Keeping this in mind and looking back at the requirements to
be an inner product, we can see that a reasonable choice
is an integral over the domain of \(x\) if \(x\) is continuous:

\[\innerprod{f}{g} = \int_{Dom(x)} f(x)g(x) dx \]

or, if \(x\) is a discrete variable, then a summation instead:

\[\innerprod{f}{g} = \sum_{x_i \in Dom(x)} f(x)g(x) \]

Note that an inner product space only has one inner product.
Part of the description of a function space is the domain 
of the input variable, which would determine which form
of the inner product would be appropriate.\par

Before we continue, we should consider what we're going to
consider the ``domain'' of the input variable for a function.
More specifically, over what interval should we integrate
a periodic function: all of \(\domain{R}\) or just one period's
worth \(T\)?
The ``better'' option might become clearer if we consider
the \textbf{metric}\index{metric}, or \textbf{distance function},
induced by our norm (which was in term induced by our
inner product).\footnote{
  Note that the inner product ``induces'' a norm
  because if the inner product and norm of a vector space
  are defined independently of each other,
  we suddenly have inconsistent notions of ``similarity''
  and ``length''. Likewise for norms and metrics
  with ``lengths'' and ``distances''.
  \href{http://people.math.gatech.edu/~heil/books/metricbrief.pdf}{This document}
  seems promising in explaining why these ``inducements''
  make sense \textemdash{} we will report back and
  update this when it's been read.
}
When an inner product is defined, the metric 
induced on our inner product space is

\[d(x,y) = \norm{x-y} = \sqrt{\innerprod{x-y}{x-y}} \]

Now, say we have \(f = \cos(t)\) and \(g\) as a sawtooth
wave of period, amplitude, and phase equal to \(f\)
(i.e., \(g(t) = Saw(t; T = 2\pi, A = 1, \phi = 0)\).
What would be the distance between our functions if
we decide to integrate over all of \(t\) for which
they're defined?

\[d(f,g) = \sqrt{\int_{t = -\infty}^{\infty} \left(\cos(t) - Saw(t; 2\pi, 1, 0)\right)^2 dt} \rightarrow \infty \]

Ouch. I mean sure, the sawtooth doesn't perfectly match
the sinusoid, but do they really feel \emph{infinitely apart}
from each other? I don't know about you, but these two
functions feel a lot closer to each other than, say,
\(\cos(t)\) and \(\sin(t)\). What can we do to fix it?\par

Well, when you think about it, \emph{all} of the information
about both of our functions 
is contained in the interval \([0, 2\pi)\).
Really, it would be more appropriate to say that our
functions are defined over an interval of length \(2\pi\),
e.g.,
\(t \mod 2\pi \) (where \(t \in \domain{R}\)), and that
we're just making copies of our functions outside of
the defined interval as a ``courtesy'' more than
anything. If you don't perform the modulo operation,
you end up with weirdness \textemdash{} if there's only
\(2\pi\) radians in a circle, how do you get the
\((x,y)\) coordinate (read: sine or cosine) of the
point \(3\pi\) radians into the circle? We immediately
say ``Oh that's just \(-1,0\)'', but that's just because
we've been sort of ``programmed'' to perform the modulo
operation without realizing what we're doing.
And the
fact that we usually tile \(\domain{R}\) 
with our periodic function
reinforces the idea that \(\cos(3\pi) = -1\) makes sense, when
really what makes sense is that we've defined the cosine function
over an interval (say \([0, 2\pi)\)), \(3\pi \mod 2\pi = \pi\),
and \(\cos(\pi) = -1\).\par

All this to say, it feels like if we want to compare periodic
functions with identical periods, we should compare
only over one period (which captures all the ``content''
of the functions) instead of amplifying the difference an
infinite number of times over a repeatedly tiled domain:

\[
\begin{split} d(f,g) &= \sqrt{\int_{t = 0}^{2\pi} \left(\cos(t) - Saw(t; 2\pi, 1, 0)\right)^2 dt} \\
  &= \sqrt{4 \times \left(\frac{5\pi}{12} + \frac{4}{\pi} - 2\right)} \\
  &< \infty
\end{split}
\]

which just feels a whole lot more reasonable.
In general then, if two functions are periodic with the same period \(T\),
we would have the inner product be

\[\innerprod{f}{g} = \int_{t=-T/2}^{T/2} f(t)g(t) dt \]

with the norm and metric updated accordingly.
This would seem to break the inner product for aperiodic functions.
But what is an aperiodic function but a periodic function with
infinite period \sout{and one less space}?
Simply let \(T \rightarrow \infty\) in these cases.
% What about if you have different periods?
% Well, we want 
% Well, we want to stay
% in our vector space, which means some sort of ``average
% value over all possible phases'' scheme is out of the picture
% since there is no way to create a phase-shifted periodic function
% out of a linear combination of basis functions without explicitly
% adding basis functions unless we just happen to ``luck out''\footnote{
%   As an example, remember that \(\sin(\omega t + \phi) = \sin(\omega t)\cos(\phi) + \cos(\omega t)\sin(\phi)\),
%   which is not a linear combination of sines and cosines.
% }. So we're stuck with the same starting phase for both.
% Then we'll just take \(T\) to be the maximum of the periods of
% the functions in question.

\subsubsection{Aside on cross-correlations.}

What if \(f\) and \(g\) have different periods? 
\emph{Then} how would you measure the similarity between them?\par
% OK, but what if we \emph{did} allow phase shifts?
% \emph{Then} how would you measure the similarity between two functions
% if the periods of the two functions \(f\) and \(g\) are different?
Well, in that case we'd probably want to know how ``similar''
the two are for a phase shift for each function \(\phi_f\) and \(\phi_g\),
where we'd have \(\phi_X = 1\) be a phase shift corresponding to a full
period of the function \(X\).\footnote
{
  Note that at the very least for sinusoidal waves, we can
  describe a phase-shifted periodic function as a linear combination
  of two non-shifted functions. That is to say, the statement
  \(\exists b_1, b_2 \left(a\sin(\omega t + \phi) = b_1\cos(\omega t) + b_2\sin(\omega t)\right)\) is true.
}
Interestingly, only the \emph{relative} phase shift 
\(\left(\phi_f - \phi_g\right) \mod 1 := \phi \)
matters \textemdash{} when you're integrating over a full period,
it doesn't matter where you start.
So then we can calculate the ``similarity'' between two periodic
functions as a function of phase difference between them:

\[ \begin{split}
  CC(s; F(s), G(s)) &:= 
    \int_{\phi = 0}^{1} F(s + \phi) G(s) d\phi \\
    &= \int_{\phi = 0}^{1} F(s) G(s + \phi) d\phi
\end{split}
\]

% where we define a variable \(s_{\cdot}\) so that for a function \(x(t)\)
% with period \(T\), \(s_x = \frac{t}{T}\).
% This normalizes the units so that \(F(s)\) and \(G(s)\) 
where we've defined \(F\) and \(G\) such that 
\(F(s) = f(sT_f)\)
and
\(G(s) = g(sT_g)\). That is, we \emph{normalize} our input variable
across all functions so that \(s\) reports how many periods into
the function we are. 
This assumes a measure where we weight a differential phase shift
\(d\phi\) equally between the two functions, despite the same
magnitude of phase shift requiring ``more'' of the original
input variable in one function (the function with the larger period)
than the other.\par

While I feel the above form makes more intuitive sense (though
it may take a bit of work to intuitively understand \(s\) and \(\phi\) since
we need to decouple these variables from the original input variables
in our minds), 
we often
instead see things in terms of periods and time.
Given two periodic functions with periods \(T_1\) and \(T_2\),
any arithmetic combination (sum or product) of them yields
another periodic function of period \(T = \frac{T_1 T_2}{\gcf{T_1, T_2}}\). With this, we can continue:

\[ \begin{split}
  CC(t; f(t), g(t)) &:= 
    \int_{\tau = 0}^{T} f(t + \tau) g(t) d\tau \\
    &= \int_{\tau = 0}^{T} f(t) g(t + \tau) d\tau
\end{split}
\]

Note that the \(CC\) function is slightly different depending
on whether we're using our normalized input variable \(s\) or
our unnormalized input variable \(t\) (they \emph{are}
from different domains, after all). The latter
requires some ``tiling'' of our periodic functions in
\(t\)'s domain
in order to compare all possible relative phase shifts.\par

It would be fine if we kept things like this (the form suggests
an intuition), but often people just immediately jump to
aperiodic functions and so suddenly take limits to infinity\footnote
{
  The most common bounds given are
  \([0, \infty)\) or \(\domain{R}\) depending on where
  the function is defined \textemdash{} we'll show the most
  commonly seen one (over \(\domain{R}\)), though it pains us slightly.
  The pain is because sending both \(T_f\) and \(T_g\) to \(\infty\)
  makes it extremely difficult to consider what changes when we
  shift \(g\) instead of \(f\) \textemdash{} the darn thing
  looks exactly the same, after all!
}:

\[ \begin{split}
  CC(t; f(t), g(t)) &:= 
    \int_{\tau = -\infty}^{\infty} f(t + \tau) g(t) d\tau \\
    &= \int_{\tau = -\infty}^{\infty} f(t) g(t + \tau) d\tau
\end{split}
\]

This is called the \textbf{cross-correlation}\index{cross-correlation}
operator
and can be used to determine ``how much'' of an aperiodic 
function defined over \domain{R} is in another aperiodic function 
defined over \domain{R}. Hopefully, now that we discussed it
from its more natural situation with periodic functions,
it makes more sense! \par

And I suppose to answer our initial question,
we would have to integrate over \(s\) (or \(t\)) so that
our final value is a scalar (i.e. that our inner product maps
to a field). In \(s\) form:

\[ \innerprod{f(t)}{g(t)} =
\int_{s=0}^{1}CC\left(s; F\left(s\right), G\left(s\right)\right) ds = 
\int_{s=0}^{1} \int_{\phi = 0}^{1} F(s + \phi) G(s) d\phi ds
\]

and in \(t\) form:

\[ \begin{split}
  \innerprod{f(t)}{g(t)} &= \int_{t=0}^{T} CC(t; f(t), g(t)) dt \\
  &= \int_{T = 0}^{T} \int_{\tau = 0}^{T} f(t + \tau) g(t) d\tau dt \\
  &= \int_{T = 0}^{T} \int_{\tau = 0}^{T} f(t) g(t + \tau) d\tau dt
\end{split}
\]

though as far as I'm aware, this value isn't used all that
often (people seem to usually stick to the autocorrelation). \par

We can define another very useful operator
(the \emph{convolution} operator)
by taking the above equation and tweaking it seemingly innocuously
\textemdash{} which ends up being the \emph{Hermitian adjoint}
of the cross-correlation operator\footnote{
  Something I hope to actually understand the meaning of soon.
} \textemdash{}
but that equation comes far more naturally from consider
\href{https://stats.stackexchange.com/a/332127}{the addition of two random variables}
and would send us on an (even more) obtuse tangent from our talk of
vector spaces and inner products. 

\subsubsection*{Inner product wrap-up.}

Now this definition of an inner product would be great 
\textemdash{}
we'd have a way of describing any function in terms of 
our desired basis by doing an integral!
\textemdash{}
\emph{if}
we can get it to converge.
Hardly a guarantee: take \(f(x) = x^0, g(x) = x^2\)
(both natural unit vectors in 
our current basis of polynomial space!)
and \(x \in \domain{R}\) as just one example.
But let's not give up on it just yet \textemdash{}
we may just be able to make things work with another basis!

\subsection{Beyond the countable, and picking a basis.}

It would help if
we expanded our horizons a bit.
I mean, getting to use \(\{x^i \mid i \in \domain{N}\}\)
is great and all, but we're definitely not spanning nearly
as much of our function space (say the space of smooth
(i.e., infinitely differentiable) functions, 
\(\domain{C}^\infty\)) as we could.
I mean, I can't even fully encode \(x^\pi\) or \(x^e\) 
with such a puny basis! And I like both \(\pi\) and \(e\)!
\sout{And especially pie! \textit{Mmmm... Pie.}}\par

Enough belly-aching. We're now going to expand our
set of basis functions to be \emph{continuous}:
\[B_{P+} = \{x^i \mid i \in [1, \infty)\} \] 

Now we can handle \(x^\pi\) and \(x^e\) quite easily.
But we still can't describe \emph{every} function (namely,
any functions which contain \(x^i, i < 1\)).
And we're probably still in trouble with using our
inner product \textemdash{} outside of \(\abs{x} < 1\),
we're probably exploding.\par

Alright, let's not be so negative! Or wait... 
Actually, let's be negative!

\[B_{P} = \{x^{-i} \mid i \in [1, \infty)\} \] 

Alright, \emph{this} looks promising! Now if we restrict
the domain of our input variable to 
something like \(x \in [1, \infty)\),
our inner product
won't explode just by taking the norm of a basis vector.
This is promising, but there may be an even more useful
set of basis functions to use \textemdash{} it's kind of
lame to have to start at \(x=1\). What if we're modeling
something over time? We don't want to just chop
off the first horizontal unit's worth of data if we 
could help it! \par

Before we get carried away, we should realize what these
changes of basis mean. You'll notice that in our switch
from \(B_{P+}\) to \(B_{P}\), we lost the ability to
fully/easily encode \(x^\pi\) and \(x^e\), and we
had to change our allowed input domain from \((-1,1)\)
to \([1, \infty)\). So when we change our basis, it
isn't without consequence. We end up
\emph{changing the part of function space we can describe}. 
Put in a way that may sound almost tautological,
\emph{different sets of basis functions
describe different (potentially disjoint) sets of functions}.
This is worth keeping in mind as we finally talk about
integral transforms.


\subsection{The integral transform: a concatenation of inner products.}

So what is an integral transform?
You could make your own if you wanted to!
(Whether or not it would be useful is another question.)
Essentially all it is is \emph{a bunch of inner products
over a set of previously selected basis functions}.
As we talked about before, taking inner products of
a vector \(\vec{v}\)
with basis vectors \(\{\vec{b_i}\}\)
and dividing by the norm of the basis vector
provides us the coefficients needed to describe \(\vec{v}\)
in the vector space spanned by \(\{\vec{b_i}\}\).
Going back to functions again, 
we can say that \emph{the integral transform
produces from the input function \(f(x)\)
the vector representation of \(f\),
in the form of the coefficient-generating function \(A_f(n)\),
in the function space spanned by a set of basis functions
\(\{K(x;n)\}\)}. In mathematical form,

\[A_f(n) = \innerprod{K(x;n)}{f(x)} 
  = \int_{Dom(x)} K(x;n) f(x) dx \]

You're free to choose whatever set of basis functions
\(\{K(x;n)\}\) you'd like, and have the size of this set
(measured by \(n\)) be whatever you'd like.
Since the choice of \(K(x;n)\) is at the core of
the integral transform (it determines what functions you
can represent and in what way they're represented),
\(K(x;n)\) is called the 
\textbf{kernel}\index{kernel!in integral transforms}
of the transform.\footnote{
  Which is unrelated to the kernel, i.e. nullspace, of a matrix in linear algebra\par
  ...Yeah, I know, it would have been nice if they were a
  bit more creative with the names. 
  \sout{I mean, ``nullspace''
  was perfectly good for what it was describing \textemdash{}
  why'd they have to go and call it the ``kernel'' too?
  All it succeeded in doing was make me try to understand
  the connection between these two kernels in vain until
  I realized there was no connection and they're just
  named the same thing.}
  I found it confusing too.
}
With a properly restricted input domain and properly
chosen kernel, the mapping between \(f\) and \(A_f\)
is one-to-one, implying both the uniqueness of \(A_f\)
\emph{and} a possible to ``invert'' the transform
back into the form of the original function!\footnote{
  Technically, not entirely true.
  Two functions \(f\) and \(g\) would have identical encodings
  (\(A_f = A_g\)) if \(f\) and \(g\)
  differ at only a countable set of points \(S\),
  i.e., that the Lebesgue measure of \(S\) is 0.
  This is because we're using an integral as our inner product
  \textemdash{}
  we're performing a ``simple'' integral over the input domain,
  implying the use of the Lebesgue measure,
  and since \(S\) would has a Lebesgue measure of zero,
  it doesn't affect the output of the inner product at all.
  So it isn't foolproof, but it works if you restrict
  the portion of function space so as to not have
  these sorts of ``antagonistic'' examples in them.
}\par

You'll notice that in the above definition of the transform,
I didn't divide by the norm of the basis vectors.
Technically, if we want \(A_f\) to capture the \emph{projections}
of \(f\) onto the basis functions, we would indeed need
to divide through by the norm
(expansion assumes an \(L^2\) norm):

\[\begin{split}
  \vec{f} &= \frac{\innerprod{K(x;n)}{f(x)}}{\norm{K(x;n)}} \\
  &= \frac {\int_{Dom(x)} K(x;n) f(x) dx } {\sqrt{\int_{Dom(x)} K(x;n)^2 dx }}
\end{split}
\]

Personally, I kind of like this form a little bit more, mainly
because its seemingly out-of-nowhere form might prompt a student
to ask ``What's up with that denominator?''. Which
may lead to a discussion about 
this pretty interesting way of thinking of functions,
which may help demystify the Transforms\textsuperscript{TM} 
that otherwise appear to be delivered from on high.
However, in this case, simplicity prevails.
See, because the denominator is invariant to the input function
\(f(x)\), all coefficient-generating functions \(A_f\) for
\emph{every} function is off by the same factor
\(1/\sqrt{\int_{Dom(x)} K(x;n)^2 dx}\) at any particular \(n\).
But if they're \emph{all} off by the exact same factor at the
exact same locations,
the coefficient-generating functions will be unique
whether you correct by that factor or not.
This is good, because many times the norm will contain
factors which would complicate the transform's output.
And so often they don't until/unless they perform an
inverse transform, 
at which point they make up for the difference.
We'll argue through an example of this in a moment.

\subsection{The Laplace Transform: a differential equations-friendly integral transform.}

One of the most common integral transforms one comes across,
often in the context of differential equations,
is the Laplace Transform, often written with 
\(t\)'s and \(s\)'s instead of \(x\)'s and \(n\)'s as
I had done above:

\[\mathcal{L}(f(t)) = F(s) = \int_{t=0}^{\infty}e^{-st}f(t)dt\]

Considering this from a function space perspective,
we're choosing to describe a function \(f(t)\)
using the set of basis functions \(\{e^{-st}\}\).
So the kernel of the Laplace transform is \(K(s,t) = e^{-st}\),
chosen because exponential functions turn on Easy Mode
for integration/differentiation, making them
incredibly useful for certain classes of
differential equations (which anyone who has 
taken a differential equations course is probably 
already familiar with). Not only
that, they have an associated inverse transform!
(More on that in a moment.)
The transform gives us the (unscaled)
coefficient-generating function \(F(s)\),
which, given an \(s\), computes the (function space's)
inner product of \(f(t)\) 
with the appropriate basis vector \(K(s,t)\), and therefore
the (not properly rescaled) coefficient of \(f\) at the index \(s\)
in the chosen basis \(\{e^{-st}\}\).\par

Technically, we haven't fully specified the transform yet.
As with any transform, the domain of the input variable 
(in this case, \(t\)) and the basis we've chosen
(specified by both the kernel \(K(s,t)\) and the domain of
the ``indexing'' variable, \(s\)) determine which functions
we can describe, because we need the inner product 
\innerprod{K(s,t)}{f(t)} to converge.\par

As it turns out, there are two variations on a theme here:
imaginary vs complex domains.
Let's start with the one I personally find more interesting.

\subsubsection{The Fourier transform: Orthogonal Bases.}

If we want to transform a function \(f(t)\) that's defined
for \(t \in (-\infty, \infty)\), we need \(f(t)\) to be a member
of \(L^1\) space; i.e. that \(f(t)\) be \(L^1\)-integrable; 
i.e. (less cryptically) we need the \(L^1\) norm of \(f(t)\)
to converge:
\[\norm{f(t)}_1 = 
  \int_{t=-\infty}^{\infty} \abs{f(t)} dt < \infty \]

If this condition is met, then we would want to constrict
\(s\) to be an imaginary number, which we'll parametrize
i.e. \(s \in S = (-i\infty, i\infty)\) 
where \(i := \sqrt(-1)\).
To make the path explicit, we can have \(s = i\omega\),
in which case we'd have \(\omega \in (-\infty, \infty)\).\par

What does that make our kernel look like?
In parametrized form, we have \(K(\omega) = e^{i\omega t}\).
But wait, we know from Euler's formula
(provable using Taylor series) that
\[e^{i\omega t} = \cos{\omega t} + i\sin{\omega t} \]

that is, \(e^{i\omega t}\) involves a rotation around 
the unit circle in the real-imaginary plane 
(counterclockwise if \(\omega t\) is positive,
clockwise if it's negative).\footnote
{
  What's especially neat is
  that it can be shown that the two functions that ``comprise''
  \(e^{i\omega t}\), 
  \(\{\cos{\omega t} \mid \omega \in \Omega_{C} = (0, \infty) \}\) and 
  \(\{\sin{\omega t} \mid \omega \in \Omega_{S} = [0, \infty) \}\) are 
  \emph{mutually orthogonal} functions, i.e. that
  for a period \(T\) over the implied resultant waves
  for the integrands below,
  \begin{itemize}
    % \tightlist
    \item
      \(\int_{T}\cos{\omega_1t}\cos{\omega_2t} > 0 \implies \omega_1 = \omega_2\)
    \item
      \(\int_{T}\sin{\omega_1t}\sin{\omega_2t} > 0 \implies \omega_1 = \omega_2\)
    \item
    \(\forall 
      (\omega_1, \omega_2) \in \Omega_{C} \times \Omega_{S}, 
    \int_{T}\cos{\omega_1t}\sin{\omega_2t} = 0 \)
  \end{itemize}
  So we can view \(e^{i\omega t}\) as ``packaging together''
  these mutually orthogonal basis functions into one nice,
  easily integrable/differentiable function.\par

  The cosine function is even (\(\cos(-x) = \cos(x)\)) and
  the sine function is odd (\(\sin(-x) = -\sin(x)\)), so
  changing \(\Omega_{C}\) and \(\Omega_{S}\) to 
  \((-\infty, \infty)\) would gain no new information.
  (And in fact, the \emph{cosine} integral transform and
  \emph{sine} integral transform need not be over \domain{R}.)
  However, \(e^{i\omega t}\), being the sum of an even function
  and an odd function, is neither even nor odd and so
  has no simple symmetry and
  \(\{e^{i\omega t} \mid \omega \in (-\infty, \infty)\}\)
  are all linearly independent.
}
So when we perform the integral transform:

\[\fancy{F}(f(t)) = F(\omega) 
  \propto \innerprod{e^{-i\omega t}}{f(t)}
  = \int_{t=-\infty}^{\infty} e^{-i\omega t} f(t) dt\]

(we'll explain the \(\propto\) in a bit)
we can interpret the (probably complex-valued) coefficient
``indexed'' by \(\omega\) as answering the following
(very long-winded) question:\par

Let's say
I coiled output of \(f(t)\), 
starting from \(t=-t'\) and going until \(t=t'\),
around the complex unit circle clockwise\footnote{
  ``Clockwise'' because the exponent of \(e^{-i\omega t}\)
  has a negative sign at the front.
}
with a winding angular frequency of \(\omega\). I'd end
up somewhere in the complex plane, say at a point \(p(t')\).
Where would I end up as I use more and more of
\(f(t)\), i.e., what is \(\lim_{t' \rightarrow \infty} p(t')\)?
\par

This doesn't necessarily help us in performing the transform
itself (called the \textbf{Fourier transform}),
and it may be harder to wrap one's head around it
compared to the (potentially) more straightforward
``\(F(\omega)\) is the inner product between the input function
and the basis function indexed by \(\omega\)''.
But this complex-plane rotation interpretation is
incredibly important in just about any topic involving
the \emph{phasor} interpretation of (sinusoidal) waves. 
For example, the above question
pegs \(\omega\) as a frequency of rotation. If we observed
a circuit element's 
voltage response to a signal over time as \(f(t)\),
then the Fourier transform of that signal 
\(\fancy{F}(f(t)) = F(\omega)\) shows how the circuit element
respond to the \emph{frequencies} of the signal.
This \emph{transfer function} becomes useful in determining
what wave content is amplified/preserved/attenuated when
passing through the element, among other things.\par

Perhaps more importantly for the moment, this gives us
an intuitive way of determining the form of the 
\emph{inverse Fourier transform}.
Remember, we said that
by coiling \(f(t)\) around the complex unit circle clockwise
with a winding frequency of \(\omega\), we got \(F(\omega)\).
How would we undo such an operation?\par
Well, how about taking \(F(\omega)\) and rotating it
\emph{counterclockwise}?
\[f(t) \propto 
  \int_{\omega=-\infty}^{\infty} e^{i\omega t} F(\omega) d\omega \]

We're \emph{almost} there, but notice the use of \(\propto\)
and not \(=\).
As it turns out, the proportionality constant is \(1/2\pi\).
The constant comes out a bit more naturally if one arrives
at the Fourier transform by taking the limit of
the Fourier series,
but we can still make sense of it.
We'll try to make sense of it in a few ways:

\paragraph{An appeal to physical intuition: ``natural'' units.}

In our phasor interpretation of the Fourier trasnform,
we pegged \(\omega\) as a frequency of rotation.
More specifically, it is an \emph{angular} frequency of rotation,
where one full revolution around the unit circle
(in the real-imaginary plane)
has a period of \(2\pi\).\par

Let's continue to ``interpret'' the inputs of our functions
by considering \(f(t)\). In particular, let's say \(f(t)\)
is a periodic function in time \(t\) with period \(T\). 
We already said we're converting to a function based
on frequency (currently, angular frequency).
In terms of units,
the simplest way to convert from time to frequency?
Well, just inversion, no?

\[t \mapsfrom \frac{1}{t} := f \]

where \(f\) is the \emph{temporal frequency} of the function.
With such a transformation, \(f=1\) corresponds to
a winding frequency such that a time-interval
of \(\Delta t = 1\) corresponds to one revolution around
the unit circle (in the real-imaginary plane).
It's like there's the same ``density'' between the two units,
i.e., \(df = dt\).
That just \emph{feels} right, doesn't it?
\par

But as it stands, we \emph{don't} have such a setup.
Instead we have \(\omega = 2\pi\) corresponding to
a wrap of \(\Delta t = 1\) around the unit circle.
So we kind of have to go through ``more'' \(\omega\)
to wrap all of \(t\). 
So it's like if we were comparing
differential amounts of \(t\) with \(\omega\), we'd have

\[d\omega = 2\pi dt\]

rather than our cleaner \(df = dt\).\par
But if we're going back to a domain in \(t\) 
via an inverse transform, we don't
want our unit's ``thickness'' to have dilated by \(2\pi\)
\textemdash{} that'd be a different function (with a period
larger than our original by a factor of \(2\pi\))!
So we need to change our thickness of \(d\omega\) to match
that of \(dt\):

\[  dt = \frac{d\omega}{2\pi} \]

and so

\[\begin{split}
  f(t) &=
  \int_{\omega=-\infty}^{\infty} e^{i\omega t} F(\omega) \frac{d\omega}{2\pi} \\
  &= \frac{1}{2\pi}\int_{\omega=-\infty}^{\infty} e^{i\omega t} F(\omega) d\omega
\end{split}\]

A much cleaner way of having this all work out
is to have our transform map directly to the unit that just
``feels'' right, i.e., the temporal frequency.
That would mean making a \(\Delta f = 1\) correspond to
our phasor spinning around the unit circle exactly once.
This is easily done by pulling out the period of \(2\pi\)
out of our frequency unit and sticking it as a constant
in the exponennt, i.e., changing our basis function
for our original transform:

\[e^{-i\omega t} \mapsto e^{-2\pi ift} \]

making our Fourier transform be

\[\fancy{F}(x(t)) = X(f) 
  = \int_{t=-\infty}^{\infty} e^{-2\pi ift} x(t) df\]

where we've denoted the function as \(x\)
to avoid confusion with the temporal frequency \(f\).
Now because our unit's ``thicknesses'' are equal,
the ``size'' of the domain of \(t\) and the size of the
domain of \(f\) are equal, so when we perform the
inverse transform and integrate over \(f\), we don't
have to divide our differential by any factor:

\[x(t) \propto 
  \int_{f=-\infty}^{\infty} e^{2\pi ift} X(f) df \]

This line of reasoning did get us to the factor we needed
(and suggested a ``nicer'' option to boot), but it required
to interpret the math to reach it. Physical intuitions
of mathematical operations are wonderful, but it would
be shocking if there weren't a way to get to the answer
while ``sticking to math''.
% Well, such methods exist, but it involves considering
% how to construct unitary operators over a Hilbert space
% (and the revelation that we usually ``intuitively''
% think of functions as elements of a Hilbert space).
% I hope to get the chance to give such explanations
% once I learn them properly myself.
% When that's the case, we'll expand this section
% more mathematical rigor.\par

% All this to say, inverse transforms are tricky things
% to find! After all, such a task amounts to inverting
% an infinite-dimensional transformation matrix \textemdash{}
% would be a shock if it \emph{were} simple!

\paragraph{An appeal to norms.}

Well, what about norms? I made such a big hullabaloo about
it earlier. Why don't we use them here? \par

Say our basis function \(e^{-i\omega t}\) is a periodic function
with period \(T = 2\pi\).\footnote
{
  Remember Euler's formula: \(e^{ix} = \cos(x) + i\sin(x)\).
  The periodicity of the complex exponential \(e^{ix}\) is then
  the period of the sum of the two sinusoidal functions.
  Since they both have the same period \(T = 2\pi\),
  so too does \(e^{ix}\).
} What is its norm? 
(Remember that for complex numbers, we use an 
\(L^2\) norm and consider \(z\)'s domain isomorphic to 
\(\domain{R}^2\), so that \(\abs{z} = \abs{x + iy}
= \sqrt{\abs{x}^2 + \abs{y}^2}\).)


\[\begin{split}
  \norm{e^{-i\omega t}} &= \sqrt{\int_{\omega t=0}^{2\pi} \abs{e^{-i\omega t}}^2 d(\omega t)} \\
  &= \sqrt{\int_{\omega t=0}^{2\pi} \abs{\cos(\omega t) + i\sin(\omega t)}^2 d(\omega t)} \\
  &= \sqrt{\int_{\omega t=0}^{2\pi}
  \left(
    \left(\cos(\omega t)^2 + \sin(\omega t)^2\right)^{1/2}
    \right)^2 
    d(\omega t)} \\
  &= \sqrt{\int_{\omega t=0}^{2\pi} 1 d(\omega t)} \\
  &= \sqrt{2\pi}
\end{split}\]

This is regardless of one's choice in \(\omega\) or \(t\) and
so is the same even if the the exponent didn't have the
negative sign.
(The result of the integral is
also not too suprising when you consider the phasor interpretation of the complex exponential function).
So then our basis functions for \emph{both} the Fourier
transform \emph{and} the inverse Fourier transform have
norms of \(\sqrt{2\pi}\). Then it would probably be better
to divide through by the relevant norm in both directions, no?

\[\fancy{F}(f(t)) = F(\omega)
  = \frac{1}{2\pi} \int_{t=-\infty}^{\infty} e^{-i\omega t} f(t) dt\]

\[\fancy{F}^{-1}(F(\omega)) = f(t)
  = \frac{1}{2\pi} \int_{\omega=-\infty}^{\infty} e^{i\omega t} F(\omega) d\omega\]

Dividing through by the norm has had the added consequence
of preserving inner products of two functions \(f\) and \(g\)
before and after our transform, i.e., 
\(\innerprod{f}{g} = \innerprod{\fancy{F}(f)}{\fancy{F}(g)}\).
This makes the transform as we defined it just now a
\textbf{unitary operator}\index{unitary operator} over
our function space (which, through our choice of inner product,
is a Hilbert space and so has a notion of ``unitary operator'').
Intuitively, a unitary operator doesn't squash or stretch
the norms of the objects it operates on. For a more easily
graspable example, the linear transformation \(T\)
on \(\domain{C}\)
corresponding to a rotation in the counterclockwise direction
by an angle \(\phi\) is an example of a unitary operator.
If we find another unitary operator \(T^{-1}\) 
that ``undoes'' \(T\), then everything ends up exactly where
it used to be. And that's exactly what we just did,
but with functions 
over an uncountably infinite-dimensional vector space!
Pretty neat, huh?

\subsubsection{The Laplace transform: just a slightly augmented Fourier transform.}

Now, not all functions \(f(t)\)
are \(L^1\)-integrable over their domain
at first, so the Fourier transform wouldn't work on those. 
But what if we actively tried to squish our 
function down with our basis functions? Specifically,
let's squash it by an exponentially decaying function.
That would mean allowing \(s\) of our basis functions 
to have a real component to them and so be complex.
Say \(s = r + i\omega\)
All we'd need to do is choose an \(r\) large enough to
make the following integral converge:

\[ \fancy{L}(f(t)) = F(s) = \int_{t \in Dom(t)} e^{-(r + i\omega)t} f(t) dt \]

This would be our \textbf{Laplace transform}. It amounts
to the same sort of winding action described in the Fourier
transform, except that we're also squashing all the vectors
we eventually sum (and so the resultant vector)
by a factor of \(e^{-rt}\). Since it's so similar to
the Fourier transform, we can use the same idea as
before to get the inverse transform: wind the output
of the forward transform in the opposite direction
(i.e., clockwise). We just will also need to de-squashify
our values by now stretching each of our points by
\(e^{rt}\), but besides that, we'd be set to migrate
over our inverse transform from our analysis of the Fourier
transform:

\[ \fancy{L}^{-1}(F(s)) = f(t) = \frac{1}{2\pi} \int_{\omega = -\infty}^{\infty} e^{i\omega t} e^{rt} f(t) d\omega \]

where we have that factor of \(1/2\pi\) because of the reasons
we've discussed before.\par

Some people like to be real fancy and describe the above
as a path integral in the complex plane that covers the
same path we've implicitly described above:

\[ \fancy{L}^{-1}(F(s)) = f(t) = \frac{1}{2\pi i} \int_{s = r - i\infty}^{r + i\infty} e^{st} f(t) ds \]

You'll notice a factor of \(1/i\) pop out of seemingly
nowhere.
Well, when we integrate over a domain \(D\),
our resulting value is expressed with the domain we integrated
over as our ``reference point''. This works great if
we're integrating over one domain and going to
that same domain (say, integrating over \(\omega \in \domain{R}\)
and ending up with a function of \(t \in \domain{R}\)).
But if we integrate along the imaginary axis, our result
will be in reference to an axis \emph{perpendicular} to
our target domain of \(t \in \domain{R}\), specifically
to an axis that is oriented \(\pi/2\) radians counterclockwise
compared to the axis we desire. So we stick
a \(1/i = i^{-1}\), a root of unity corresponding
to a clockwise rotation of a point in the
complex plane by\(\pi/2\) radians 
when thought of as an operator (and applied via multiplication),
to orient our output to the axis we want, that is, the
real numbers.
\\
\subsection{Exiting the rollercoaster.}
Well, that was quite a ride.
And this was to derive a relatively \emph{straightforward}
inverse transform \textemdash{}
inverse transforms are tricky things
to find! After all, such a task amounts to inverting
an infinite-dimensional transformation matrix
(set by our choice of basis functions and therefore our choice
of kernel).
It would be a shock if such a 
monumental task \emph{were} simple!\par

In any case, I think that's enough for now.
\\
\\
- DK, 6/19/18

\newpage

\chapter{Analysis.}\label{chapter:analysis}

(Currently in its infancy \textemdash{} we're starting
with discussions about ``simple'' calculus at the moment.)

\section{Directional derivatives and the gradient.}

Why does the gradient of a function \(f\), \(\grad{f}\),
point to the ``direction of steepest ascent'' for functions?
To answer that, we need to remember 
about directional derivatives.

\subsection{Partial derivatives.}

As a reminder, we define a 
\textbf{partial derivative}\index{partial derivative}
as a limit along one variable, akin to the derivative
of a one-variable function.

\[\begin{split}
  \pdv{}{x_i}f(x_1, x_2, \cdots, x_i, \cdots, x_n) \definedas \lim_{h \rightarrow 0} \frac{f(x_1, x_2, \cdots, x_i + h, \cdots, x_n) - f(x_1, x_2, \cdots, x_i, \cdots, x_n)}{h}, 
  \\
  h \in \domain{R}, Dom(f) \in \domain{R}^n
\end{split}\]

Let's remember that for a limit to exist at a point, it
must approach the same value regardless of the ``side''
from which the point is approached. For limits that approach
a point \(p\) in \domain{R}, 
there are only two ``sides'' in \domain{R} from
which the point can be approached: from values smaller
than \(p\) and from values larger than \(p\).

\[p \in \domain{R}, \lim_{x \rightarrow p}g(x) = c 
\quad \longleftrightarrow \quad
\lim_{x \rightarrow p^-}g(x) = \lim_{x \rightarrow p^+}g(x) = c\]

\subsection{Directional derivatives.}

Anyway, this is great for when we happen to want to 
approach our point \(p\) 
along one of the ``predefined'' axes \(x_1, x_2, \cdots, x_n\).
But what if we want to approach our point along a
\emph{different} line? For example, let's stick to two
dimensions (so \(f: \domain{R}^2 \to \domain{R}\)) and say we want to approach \(p\) along the line
\(x_2 = bx_1, \ b \in \domain{R}\). We could calculate this
\textbf{directional derivative} via the following expression:

\[\lim_{h \to 0} \frac{f(x_1 + h, x_2 + bh) - f(x_1,x_2)}{h} \]

More cleanly, we can associate the line of approach with a vector
\(\vec{v} = (1,b)\) and use vector notation:

\[\lim_{h \to 0} \frac{f(\vec{x} + h\vec{v}) - f(\vec{x})}{h} \]

where in this case \(\vec{x} \in \domain{R}^2\).\par
But wait! Aren't there an infinite number of points on our line
and so an infinite number of potential \(\vec{v}\) to choose
from? Indeed there are, and for the limit above to exist
they should all yield the same result (i.e., 
we should get the same answer whether we start
from the ``left'' or the ``right'' of \(h = 0\)). To disambiguate
a bit, we'll constrain our line-of-approach vector to have
a norm of 1, i.e., 
we'll have \(\vec{u}\) be a \emph{unit vector}.
This allows us to finally properly define the 
\textbf{directional derivative} operator \(D_{\vec{u}}\)
as

\[D_{\vec{u}} = \lim_{h \to 0} \frac{f(\vec{x} + h\vec{u}) - f(\vec{x})}{h} \]

From this definition, we can see that our definition of a
partial derivative \(\pdv{f}{x_i}\),
our approach along a ``predefined'' axis, 
is simply a directional
derivative along the standard basis vector \(\vec{e}_i\).\par

So our directional derivative definition is a valid 
\emph{extension}
of the definition of a partial derivative. Neat. 
But can we actually calculate it? Finding limits along an
arbitrary axis seems like a huge pain. (And in fact,
it is.) But wait, we observed that
a partial derivative along a natural axis \(x_i\)
is the directional derivative 
along the natural basis vector \(e_i\):

\[ D_{e_i}f = \pdv{f}{x_i} \]

And normally in linear algebra, we can define vectors
in terms of a linear combination of 
a set of basis vectors (say the natural basis
vectors \(\{\vec{e}_i\}\)):

\[\exists \vec{\lambda} = \left(\lambda_1, \lambda_2, \cdots, \lambda_n\right) \suchthat \vec{v} = \sum_i \lambda_i \vec{e}_i \]

Can we define our directional derivative along a unit vector
\(\vec{u}\) as a linear combination of 
the directional derivative along
the natural basis vectors \(e_i\)? In fact, we can!
And with both \(\vec{u}\) and \(e_i\) having unit norm, 
it's just as simple
as weighting the partial derivative of each \(\pdv{f}{x_i}\)
by \(\lambda_i \setequalto \innerprod{\vec{u}}{\vec{e}_i} = u_i\):

\[\begin{split}
  D_{\vec{u}}f &= \sum_i \lambda_i D_{\vec{e_i}}f \\
      &= \sum_i \innerprod{\vec{u}}{\vec{e}_i} \pdv{f}{x_i} \\
      &= \sum_i u_i \pdv{f}{x_i} \\
      &= \innerprod{\vec{u}}{\grad{f}}
\end{split}
\]

where 
\(\grad{f} = \left(\pdv{f}{x_1}, \pdv{f}{x_2}, \cdots, \pdv{f}{x_n}\right)\)
is the \textbf{gradient operator} and 
\(\innerprod{\cdot}{\cdot}\) denotes the inner product
on our domain \(\domain{R}^n\), i.e.,
the \emph{dot product}.\footnote
{
A rigorous proof would involve defining a single-input function
\(g(z)\) which moves along the unit vector \(u\) and
noting that \(g'(0)\) equals both the definition of the
directional derivative and the result we achieved by our
linear-algebra reasoning.
}\par

Note that this expression for the directional derivative
explains why the gradient of a function at that point is
often intuitively explained as 
``the direction of steepest ascent''. Specifically,
if we wished to maximize \(D_{\vec{u}}\), we'd have

\[
  \argmax_{\vec{u}}(D_{\vec{u}}(f)) 
  = \argmax_{\vec{u}}\innerprod{\vec{u}}{\grad{f}}
\]

and by the properties of the dot product,
\(\innerprod{\vec{u}}{\grad{f}} = 
\norm{\vec{u}}\norm{\grad{f}}\cos(\theta)\) where
\(\theta\) is the angle between the two vectors.
We have constrained \(\norm{u} = 1\) and so we
maximize the expression by having \(cos(\theta) = 1\), 
i.e. \(\theta = \cos^{-1}(1) = 0\).
So there is \emph{no} angle between \(\vec{u}\)
and \(\grad{f}\), so \(\vec{u}\) points in the direction
of \(\grad{f}\). In math-speak,

\[
  \argmax_{\vec{u}}(D_{\vec{u}}(f)) 
  = \frac{\grad{f}}{\norm{\grad{f}}}
\]

and by substituting the value for \(\vec{u}\)
and noting that \(\cos(\theta) = 1\), we have

\[
  \max_{\vec{u}}(D_{\vec{u}}(f)) 
  = \norm{\grad{f}}
\]

To find the direction of steepest \emph{descent} at 
some point \(p\),
we instead find the vector such that \(\cos(\theta) = -1\),
i.e. \(\theta = \pi\), meaning the vectors are antiparallel.
Then we have

\[
  \argmin_{\vec{u}}(D_{\vec{u}}(f)) 
  = \frac{-\grad{f}}{\norm{\grad{f}}}
\]

and now

\[
  \min_{\vec{u}}(D_{\vec{u}}(f)) 
  = -\norm{\grad{f}}
\]

which, when considering partial derivatives as ``slopes''
along each axis (like how we often think of 1-D derivatives),
should make sense.
A differential step ``up'' the \(x_i\) axis 
and a differential step ``down'' the \(x_i\) axis
would have the same magnitude \(\pdv{f}{x_i}\) and so differ
only by a negative sign.
This holds for all \(x_i\), so it ought to hold
for the gradient as well.

% To find the direction of steepest \emph{descent} at 
% some point \(p\),
% we note that if the directional derivative exists
% along the \(\grad{f}\) axis, then 
% a step ``up'' from \(p\) along the \(\grad{f}\) axis 
% and a step ``down'' from \(p\) along the \(\grad{f}\) axis 
% will have the same magnitude \(\norm{\grad{f}}\)
% for a differential step \(dp\).
% And we previously found that \(\norm{\grad{f}}\)
% is the largest magnitude we can get for a directional derivative
% and is attained with the unit vector along the \(\grad{f}\)
% axis.
% So then the direction of steepest descent is ``down'' the
% axis, i.e., in the direction of \(-\grad{f}\).

% \subsection{Beyond lines: conservative vector fields.}

% This is great if we're only interested getting to our 
% destinations along specific lines, but things may break
% if we \sout{read} move between the lines and 
% go along different \emph{curves}. 
% For example, what if we did some silly ``rotational derivative''
% toward a point \(p\), i.e.,

% \[\vec{u}(r) = \left(r\cos(\frac{\theta}{r}), r\sin(\frac{\theta}{r})\right) \implies \lim_{r \to 0}\frac{f(\vec{p} + \vec{u}(r)) - f(\vec{p})}{r} = R_{\theta}f(p) = \ ? \]

% I mean, it feels like \(R_{\theta}f(p)\) \emph{should} have
% a value, since it's approaching \(p\)
% and we can define a directional derivative for any
% arbitary angle of approach \(u\) via \(D_{u}f\). 
% But which \(u\) should correspond to \(D_{u}f = R_{\theta}f\)?
% How would one choose the \(u\)?\par

% Well, the answer is ``''


% \subsection{Directional derivatives in the complex plane.}

% Now let's consider a function \(f: \domain{C} \to \domain{C}\) which takes in a \emph{complex} input \(z \in \domain{C}\)
% and returns a complex output \(f(z)\).
% \(\domain{C}\) is quite similar to \(\domain{R}^2\) in some
% ways, with real and imaginary axes. So we can break down
% our input \(z = x + iy\) in terms of its real and imaginary
% components, and also describe the output components
% in terms of these components
% \(f(z) = f(x + iy) = u(x,y) + v(x,y)i\).\par

% Now, if we want a derivative to exist in the \emph{complex}
% plane, we'd the limit of a directional derivative with
% unit vector \(\vec{c} \in \domain{C}\)

% \[ D_{\vec{c}}f(z) = \lim_{h \to 0} \frac{f(z + h\vec{c}) - f(z)}{h} \]

% to exist and be equivalent for \emph{all} unit vectors 
% \(\vec{c} \in \domain{C}\).
% Otherwise the limit as \(h \to 0\) doesn't make sense.
% This is analogous to how, for a regular 1-D real-valued
% derivative, we'd consider it to exist only if the limit
% as we approached it from \emph{every possible ``angle'' of
% approach} agreed with one another. It's just that in
% the 1-D case, there's only two ``angles'' from which
% we can approach a point: 
% from its ``left'' and from its ``right''.\footnote{
%   Recall the mathematical formulation of the above:
%   \[p \in \domain{R}, \lim_{x \rightarrow p}g(x) = c 
% \quad \longleftrightarrow \quad
% \lim_{x \rightarrow p^-}g(x) = \lim_{x \rightarrow p^+}g(x) = c\]

% \(p^-\) and \(p^+\) are the two angles, ``left'' and ``right'',
% from which we can approach the point.
% }\par

% OK, then how could we make \emph{every} possible angle of
% approach agree with one another? There's an (uncountably)
% infinite number of such approaches! Well, we can
% rely on our previous realization: if we have directional
% derivatives along a set of basis vectors \(b_1\) and \(b_2\)
% for the domain in
% question, we can describe any other directional derivative
% as a linear combination of the ones we've computed.
% Once again, we'd choose the natural basis vectors
% \(e_1\) and \(e_2\), where this time
% \(e_1 = \hat{x}\) (for the real axis)
% and \(e_2 = \hat{iy} = i\hat{y}\) (for the imaginary axis).
% Then once again, the weighting scheme for a directional
% derivative defined by the unit vector \(\vec{c}\)
% is just
% \(\lambda_i = \innerprod{\vec{c}}{\vec{e_i}}\).\par
% % Once again, limiting ourselves to unit vectors \(\hat{c}\),
% % and assuming we choose natural basis vectors (of unit norm)
% % \(e_1\) and \(e_2\), the scalar value we'd have for each
% The complex plane is isomorphic to \(\domain{R}^2\) and
% only has two basis vectors along which we need to evaluate
% \textemdash{} let's choose the real and imaginary axes
% \(x\) and \(iy\). Say they were equal at some point \(p\), i.e.,

% \[\pdv{f}{x}(p) = \pdv{f}{iy}(p) = g(p)\]

% Then, since we would describe every other derivative
% as a linear combination of \(\pdv{f}{x}(p)\) and
% \(\pdv{f}{iy}(p)\)



\subsection*{References.}

\href{http://tutorial.math.lamar.edu/Classes/CalcIII/DirectionalDeriv.aspx}{Paul's notes} 
seem to always be there for all things calculus.

\chapter{Constrained Optimization
Problems}\label{constrained-optimization-problems}

Motivated to do this when I was reading
\href{https://arxiv.org/pdf/1606.05579.pdf}{this paper} and realized I
forgot how we get to/use the KKT conditions (which is implied in Eq. 2
in the paper).

\section{Equality constraints only: The Method of Lagrange
Multipliers/The
Langrangian}\label{equality-constraints-only-the-method-of-lagrange-multipliersthe-langrangian}

NB:
\href{https://www.khanacademy.org/math/multivariable-calculus/applications-of-multivariable-derivatives/lagrange-multipliers-and-constrained-optimization/v/constrained-optimization-introduction}
{Khan
Academy's videos on the subject} really make clear one potential
intuition for one constraint.

\subsection{Intuition with just one
constraint}\label{intuition-with-just-one-constraint}

Say you want to optimize (maximize or minimize) a smooth function
\(f(\vec{x})\) subject to the constraint \(g(\vec{x}) = c\) (let's say
\(\vec{x} \in R^n\) -\/- and we may just write \(x\) instead of
\(\vec{x}\)). (It's worth remembering that the constraints themselves
can also be expressed as functions -\/- they just happen to be set to
specific constants.) So our goal is to find the best point(s),
\({\vec{x}^*}\). For the purposes of explanation, we'll say our goal is
maximization, but it all applies for minimization too.

A way to build up the intuition is to consider the contour lines of
\(f(\vec{x}) = m\) for particular values of m. Our goal then is to
maximize \(m\) while having \(\vec{x}\) satisfy \(g(\vec{x}) = c\). If
it didn't fulfill this second requirement, then we'd just be ignoring
the constraint and solving an unconstrained optimization problem -\/- in
which case, we'd just set the gradient equal to zero and solve (say,
what a useful thought -\/- let's put that in our back pocket for
later...).

Let's call the constraint contour line (which we aren't allowed to
change and is set to some constant \(c\)) \(G\) and the function contour
line (which we can change by varying \(m\)) \(F(m)\). If we think about
it for a bit, we'll see that \emph{solving our problem is analogous to
choosing the largest value of \(m\) so that \(F(m)\) ``touches'' \(G\) in
the fewest number of places while still actually ``touching'' \(G\).}
Consider the alternatives:

\begin{enumerate}
% \def\labelenumi{\arabic{enumi}.}
\tightlist
\item
  \emph{\(F(m)\) does not touch \(G\) at all:} Well, that means that
  when we look at the set of points that comprise the contour line
  \(f(x,y) = m\) (i.e., the \textbf{level set} \index{level set}
   of \(f\) corresponding to
  a value of \(m\)), \emph{none} of those points lie on
  \(g(\vec{x}) = c\). So we ignored the constraint again -\/- oops.
\item
  \emph{\(F(m)\) touches \(G\) too many times:} This implies that
  \(F(m)\) cuts across \(G\) (if it didn't, then where did the ``extra''
  touches come from?) And in that case, why not increase \(m\) a bit
  more? We're assuming \(f\) is relatively well-behaved, so if you nudge
  \(m\) up a bit to \(m^+\), the contour line \(F(m^+)\) will be fairly
  close to \(F(m)\) -\/- and so, still cross \(G\) somewhere.
\end{enumerate}

(By the way, I've been using the tortured phrase "minimal number of
times but not zero like c'mon don't be cheeky" because there can be more
than one location on the constraint curve that have the same maximal
value for \(f\). Consider maximizing \(f(x,y) = x^2 y^2\) subject to
\(x^2 + y^2 = 1\). The symmetry leads to four points that satisfy the
criteria -\/- bump up \(f\)'s output higher and you're off the circle,
bump it down and you're cutting across the circle (and also not
maximizing \(f\)).)

Now, if \(F(m)\) and \(G\) just barely touch but do not cross, then
their instantaneous ``slopes'' at their touching point must be in the same
orientation and the contour lines are moving in the exact same
``direction'' -\/- if they weren't, then the touching point is a crossing
point. So how do we capture this notion?

\subsection{Doin' me some gradients}\label{doin-me-some-gradients}

The (nonzero) gradient of a function is always perpendicular to its
contour lines. (Seems like a deep statement, but with a bit of thought,
you can see that it just comes from the definition/intuition of contour
lines (on which the value is constant) and gradients (which points
toward the direction that increases the function output the largest -\/-
with no portion of the ``step'' being wasted on movement that would keep
the value constant).)

So, we can convert out ``contour lines \(F(m)\) and \(G\) are in the same
direction'' directly to ``the gradients of \(f\) and \(g\) are in the same
direction'', i.e. \[\nabla f = \lambda \nabla g \]

where \(\nabla\) is the usual gradient operator and \(\lambda \neq 0\),
the scalar proportionality constant (it's the \emph{direction} that
matters, not the magnitude), is called the \textbf{Lagrange multiplier}
\index{Lagrange multiplier}
(dude got a lot of things named after him).

It's actually worth looking at this a bit more. We originally framed our
goal by trying to get \(F(m)\) and \(G\) to touch as little as possible.
But we can also frame it in terms of \(\nabla f\) and its relation to
the constraint functions:

\[\nabla f = \lambda \nabla g \quad \impliedby \quad \nabla f \ \text{is perpendicular to the contour } G \]

It makes sense that the right-hand side would be the case -\/- if
\(\nabla f\) \emph{did} have some part of it along \(G\), then we're
wasting that part! Why wouldn't we go along \(G\) a bit more? We'd still
be meeting our constraint, and since we stepped partially along the
direction of steepest ascent, we'd have increased \(f\) while we were at
it! It'll be worth remembering this observation a little down the line,
so keep it in your pocket for later (or some other handy container, if
you are doomed to the fate of clothing without functional storage
capabilities).

Anyway, that's great and all, but we only have \(n\) equations (each of
the \(n\) elements of the vector formulation above) and now we have
\((n+1)\) unknowns (all the coordinates for \(x\), plus \(\lambda\)).
Well, there's a reason the above relation used a(n) \(\impliedby\). On
the left-hand side, we've only captured that the gradients have to be in
the same direction -\/- we haven't added our constraint! (The right-hand
side encapsulates both, since we have \(G\) as the contour corresponding
to the specific constraint \(g(x) = c\).) So our full set of \((n+1)\)
equations with \((n+1)\) unknowns is

\[\nabla f = \lambda \nabla g \] \[g(x) = c\]

Now go to town! Worth remembering that all of this provides
\emph{necessary} but \textbf{not} \emph{sufficient} conditions for
optimality. Sufficient conditions would involve, for example, proving
the that Hessian matrix of \(f\) is negative semidefinite when trying to
maximize \(f\) (analogous to the second-derivative test in the
single-dimensional case), and even if you manage that you're only
guaranteed local maximality. Sounds like a lot of qualifications, but
we've actually narrowed the search space a great deal with these
conditions, so it's not as terrible as it sounds.

\label{sec:the-lagrangian-a-packaged-function}
\subsection{The Lagrangian: a packaged function} 

The above system of equations works great for people, but people have
also spent so much time and energy to make computers solve math problems
for us! Most of these programs are particular good and finding the zeros
of a function (without any fancy constraints). So how could we repackage
the (n+1) equations above into one function we can find into a
zero-finder?

Well, let's rewrite the above equations first:

\[\nabla f - \lambda \nabla g = 0\] \[g(x) - c = 0\]

Alright, now what? Well, if we were to write something like

\[ L(x) = f(x) - g(x) \]

we'd be \emph{almost} there, because if we took the gradient of \(L\)
and set it equal to zero, we get the ``direction'' constraint back. But at
the moment, we're making it so that the magnitudes of the two gradients
have to be the same too (which doesn't have to be true) \emph{and} we
forgot to incorporate our constraint again!

Well, why don't we reintroduce \(\lambda\) as a variable in such a way
that it handles the proportionality problem \emph{and} have it so that
\(L_{\lambda} = g(x) - c\) (so that we reincorporate our constraint into
the function)? Might sound tricky, but in fact we can modify \(L\) to
satisfy these requirements fairly simply:

\[ \mathcal{L}(x, \lambda) = f(x) - \lambda \left(g(x) - c\right) \]

Now that it's achieved its final form (thankfully didn't take ten
episodes of powering up), we change \(L\) to \(\mathcal{L}\) and call it
the \textbf{Lagrangian} \index{Lagrangian}
of \(f(x)\) (because dude needs more things
named after him -\/- and you know, he \emph{did} revolutionize the study
of classical mechanics with this formulation).

Worth noting is that if we define the Lagrangian as

\[ \mathcal{L^+}(x, \lambda ^+) = f(x) + \lambda^+\left(g(x) - c\right) \]

we still get the same answer to our optimization problem -\/- the only
difference is that compared with the \(\lambda\) we get from the
\(\mathcal{L}\) formulation, \(\lambda ^+ = - \lambda\).

A neat consequence of the formulation of \(\mathcal{L}\) is that we can
consider \(\lambda\) as a measure of how much we could improve \(f(x)\)
by incrementing the value of \(c\) (which we've been considering a
constant) by a differential amount. While it may seem to be ``clear'' just
by taking \(\mathcal{L}_c = \lambda\), it's a bit more subtle than that,
since \(\mathcal{L}(x, \lambda; c)\) was formulated with \(c\) as a
constant. The proof for this observation involves:

\begin{itemize}
\tightlist
\item
  forming a new function,
  \(\mathcal{L}^*(x^*(c), \lambda ^*(c), c) = \mathcal{L}^*(c)\), a
  single-variable function that parameterizes the input coordinates of
  the answer(s) to the optimization problem (and also the Lagrange
  multiplier) with respect to \(c\);
\item
  doing the multivariable chain rule;
\item
  thinking a bit to notice that a lot of things equal zero to get that
  \(\frac{d\mathcal{L}^*}{dc} = \lambda \);
\item
  and, having remembered that we're interested specifically about the
  points on \(\mathcal{L}\) that optimize \(f\) (which is exactly what
  \(\mathcal{L}^*(c)\) captures), realizing that the above result
  implies that first statement of the paragraph before this bulleted
  list.
\end{itemize}

What a mouthful.

\subsection{Extension to more than one
constraint}\label{extension-to-more-than-one-constraint}

Would be kind of a shame if we did all of this just to solve problems
with just one constraint. But thankfully, the extension is fairly simple
to describe!

Say we want to optimize \(f\) subject to \(k\) constraints
\(g_1 = c_1, g_2 = c_2, \cdots, g_k = c_k\). Now, in all but the most
trivial of cases, it would be impossible to have the gradients of all of
these different functions in the same direction. Intuitively (-ish, and
assuming you feel comfortable-ish with concepts in linear algebra), if
they can't all be in the same direction, you'd think that the "next best
thing" would be that the gradient of \(f\) is in the same direction as
some linear combination of the gradients of \(g_1, g_2, \cdots, g_k\),
i.e. that

\[ \nabla f = \sum_{i=1}^k \lambda _i \nabla g_i \]

That statement above is in fact a condition that is met in the answer to
our optimization problem! (How convenient.)

Now we have \(n\) equations but \(n+k\) unknowns. We once again fix that
by actually incorporating our constraints:

\[ \nabla f = \sum_{i=1}^k \lambda _i \nabla g_i \] \[ g_1 = c_1 \]
\[ ... \] \[ g_k = c_k \]

We can once again package everything together as a Lagrangian by having
that \(\mathcal{L}_{\lambda _i} = g_i - c_i\):

\[\begin{split} \mathcal{L}(x, \lambda _1, \cdots, \lambda _k) &= 
                                            f - (\lambda _1 (g_1 - c_1) + \cdots + \lambda _k (g_k - c_k) ) \\
                     &= f - \sum_{i=1}^k \lambda _i (g_i - c_i) \end{split}\]

And Bob's your uncle.

\subsection{Explaining the convenience, and a more generalizable intuition.}\label{explaining-the-convenience}

Earlier we just kind of accepted the convenience of our guess, but it's
worth figuring out why it works. Remember that observation you kept in
your pocket (or other handy container)?

\[\nabla f = \lambda \nabla g \quad \impliedby \quad \nabla f \ \text{is perpendicular to the contour } G \]

If \(\nabla f\) had any part of it along \(G\), then we could step along
\(G\) and further increase \(f\). This sounds extensible to more than
one constraint! And in fact, the ``convenient'' result captures this for
the contour line created by the intersection of all the constraints:

\[
% \begin{multline*}  
  \nabla f = \sum_{i=1}^k \lambda _i \nabla g_i  \impliedby
  \nabla f \text{is perpendicular to the intersection
  of all constraint contours } \bigcap _i G_i 
% \end{multline*}
\]

Here, \(\bigcap _i G_i\) serves as the one contour line on which we can
move along that satisfies all the constraints. (Presumably such
continuous arcs exist -\/- otherwise, there are only discontinuities
(i.e. discrete points), and so each point in the set would have to be
checked individually.)

Now, since we're dealing with the same construct (a contour along which
we can move that satisfies the constraints), the same reasoning applies
pretty much verbatim -- \emph{\textbf{if \(\nabla f\) weren't perfectly
perpendicular to \(\bigcap _i G_i\), we'd be wasting the component of
the gradient going along \(\bigcap _i G_i\), \(\nabla f ^{\parallel}\).
So we'd just step along the contour and improve our result.}}\footnote{
  Excessive bolding, italicizing, and a gratuitous footnote 
  used to emphasize the fact that this is the
  mathematically rigorous ``intuition'' to have/remember.
}
So that explains why the right-hand side makes sense. But how does that
imply the left-hand side? Specifically, how do we get the linear
combination part, \( \sum_{i=1}^k \lambda _i \nabla g_i \)?

Well, since we're all on contour lines at the same time,
\(\bigcap _i G_i\) is necessarily perpendicular to all the gradients
\(\nabla g_i\). Then any linear combination \(\sum_{i=1}^k
\lambda _i \nabla g_i \) is perpendicular to \(\bigcap _i G_i\).
In fact, the gradients form a \emph{basis} for the space perpendicular
to \(\bigcap _i G_i\), because of the symmetric property of the
perpendicularity relation. More curtly, call the span of the gradients
\(S\). By construction, \(\bigcap _i G_i \perp S\), and by symmetry, \(S
\perp \bigcap _i G_i\). We want \(\nabla f\) to be perpendicular to
\(\bigcap _i G_i\) (as we've said before). Well then, that means
\(\nabla f \in S\), which implies that it \(\nabla f\) can be written as
a linear combination of \(S\)'s basis vectors, i.e., \( \nabla f =
\sum _{i=1}^k \lambda _i \nabla g_i \). Bam! Stick that beautiful
\(Q.E.D.\) square in the corner, we are done! \sout{Would do it myself were I
writing this in \(LaTeX\) and not Markdown.} 
Well, since we've migrated to LaTeX, I think we owe ourselves a box!
\begin{flushright}$\blacksquare$\end{flushright}
Totally worth it.
\\
\\
- DK, 4/28/18

\newpage

\section{Constraints with Inequalities: the Karush-Kuhn-Tucker (KKT)
Conditions}\label{constraints-with-inequalities-the-karush-kuhn-tucker-kkt-conditions}

\subsection{Not nearly as scary as they make it out to
be.}\label{not-nearly-as-scary-as-they-make-it-out-to-be.}

For all the pomp and circumstance around this, with the caravan of names
in the name itself and the esoteric terms used in its description like
``complementary slackness'' and ``dual feasibility'', the
Karush-Kuhn-Trucker (KKT) conditions aren't nearly as hard to follow as
one would expect if the method of Lagrange multipliers for multiple
constraints makes sense/is comfortable.

First, we pose the optimization problem in ``standard form'' (which mainly
just saves us from lugging around extra constants like we did with \(c\)
in the Lagrange multipliers explanation):

Optimize \(f(x)\) subject to \(g_i(x)\leq 0\), \(h_j(x)=0\), with
\(i \in \{1, \cdots, k\}\) and \(j \in \{1, \cdots, l\}\) (so \(k\)
inequality constraints and \(l\) equality constraints). Below, we'll
assume ``optimize'' = ``maximize''. We'll point out where changes will occur
if you're minimizing instead.

\subsubsection{Primal feasibility}\label{primal-feasibility}

This time, before we do anything else, we're going to stick down the
original constraints so we don't forget they exist:

\[ g_i(x)\leq 0 \ \forall \ i, \ h_j(x) = 0 \ \forall \ j \]

This is called the \textbf{primal feasibility} condition because it's a
condition for the feasibility of the original, main, ``primal'' problem.

\subsubsection{The dual formulation}\label{the-dual-formulation}

We refer to the original problem as the ``primal'' problem to contrast it
with the ``dual'' problem. That sounds all fancy, but we've made a dual
problem before when we formed the Lagrangian for our
equality-constraints-only version. That is, the 
\textbf{dual problem} \index{dual problem} is simply a
reframed but equivalent form of the primal problem. We did it before by
solving a function that had all our constraints wrapped in one clean
package (the Lagrangian form of the problem). And hey, that was both a
neat \emph{and} a useful idea, and those don't come around all that
often, so let's use it until it goes out of style.

Worth noting is that we want to make our dual problem mirror the primal
problem exactly (in \emph{optimal value} as well as optimal location),
i.e. form a strong duality, i.e. have no
\href{https://en.wikipedia.org/wiki/Duality_gap}{duality gap}. The
following provide \emph{necessary} conditions, but not \emph{sufficient}
conditions for a strong duality.

So what would be the dual problem? We can do the same thing we did for
the Lagrangian -\/- make a new function with some extra variables whose
partial derivatives yield the constraints of our problem. Let's try it:
\\
We define a function

\[L(x,\mu_1, \cdots, \mu_k, \lambda_1, \cdots, \lambda_l) = f(x) + \sum_i \mu_i g_i(x) + \sum_j \lambda_j h_j(x)\]

Maximize \(L\) (with no external constraints; i.e., find the locations
where \(\nabla L = \vec{0}\)). (We'll come back to minimization later.)
\\
\\
Alright, well that's certainly something. Now we can recover the
constraints by noting that \(L_{\mu_i} = g_i\) and
\(L_{\lambda_j} = h_j\). But there are some things that are still funky
with the inequality constraints here, so let's work on those.

(By the way, there are some signs we'd have to flip if we were 
doing a minimization instead. We'll discuss that later on.)

\subsubsection{Dual feasibility}\label{dual-feasibility}

For one thing, our dual problem won't mimic our primal problem of
optimizing \(f\) at all if we let any \(\mu_i\) be less than zero. If we
did, then we'd easily ``win'' the optimization game by choosing some \(x\)
such that \(g_i(x) < 0\) (which is still satisfies our primal
feasibility conditions), then setting the corresponding \(\mu_i\) to
arbitrariliy large negative numbers -\/- who cares about \(f(x)\)
anyway!? Oh wait, we do. So we should probably make sure that we can't
break our problem:

\[ \mu_i \geq 0 \]

This is called the \textbf{dual feasibility} condition because, well,
otherwise our dual problem isn't all that useful.

\subsubsection{Complementary slackness and
stationarity}\label{complementary-slackness-and-stationarity}

For one thing, we can notice that there are two possibilities for the
value \(g_i\) takes at an optimal point:

\begin{enumerate}
\def\labelenumi{\arabic{enumi}.}
% \tightlist
\item
  \(g_i(x^*) < 0\): That means the inequality constraint is not actually
  stopping the function from getting to a ``better'' location where
  \(g_i > 0\) (our optimal-point finder didn't hit a wall -- there's an
  open neighborhood around \(x^*\) that still satisfies the constraint),
  so the constraint isn't being restrictive at all. So we don't have to
  worry about it!
\item
  \(g_i(x^*) = 0\): The inequality constraint \emph{is} potentially
  stopping the function from reaching a more optimal point, so the
  constraint is actually affecting the outcome. So we should be sure to
  be othorgonal to the contour in our final answer.
\end{enumerate}

These observations inform the following two constraints,
\textbf{complementary slackness} and \textbf{stationarity}:

\[\begin{split} \mu_i g_i = 0 \ \forall \ i \qquad & \text{complementary slackness} \\
\nabla f = \sum _i \mu _i \nabla g_i + \sum _j \lambda _j \nabla h_j \qquad & \text{stationarity} \end{split}\]

Let's once again consider our two cases:

\begin{enumerate}
% \def\labelenumi{\arabic{enumi}.}
% \tightlist
\item
  \(g_i(x^*) < 0\): This constraint is inactive and so is not part of
  the contour lines we use to find the set of points on which \(f\) must
  lie. Recall that we form this set by evaluating the intersection of
  all the active constraints \(C\); that the span of the gradients of
  the active constraint functions form a space \(S\) orthogonal to
  \(C\); and that since \(\nabla f\) must be orthogonal to \(C\) in
  order to be a potential extremum, \(\nabla f \in S\) and can be
  written as a linear combination of the gradients of the active
  constraints. (Boy, another mouthful.) Well, \(g_i\) is not an active
  constraint, and so \(\nabla g_i\) is not part of the basis for \(S\).
  So we want its contribution in the stationary condition to be
  \(\vec{0}\). We can do this by setting the corresponding
  \(\mu_i := 0\). Not-so-coincidentally, this necessary consequence
  leads to compliance with the complementary slackness condition as
  well.
\item
  \(g_i(x^*) = 0\): The complementary slackness condition is met no
  matter what value \(\mu_i\) takes. And that's perfect: the constraint
  is active, so \(\nabla g_i\) is part of the basis for \(S\). So we
  \emph{need} \(\mu_i\) to be nonzero so that we can describe any vector
  in \(S\) (of which \(\nabla f\) is an element, as we talked about when
  we ``explained the convenience'' for the multiple-equality-constraint
  formulation earlier).
\end{enumerate}

So the two conditions are tied to one another in this interesting way
that makes it a more-or-less direct extension of the
multiple-equality-constraint formulation!

\subsubsection{Finishing our duel with
duals.}\label{finishing-our-duel-with-duals.}

...And with that, we've formed our dual problem with only equalities!
We'll copy them here to show we have enough equations for our unknowns:

\[\begin{split} h_j(x) = 0 \qquad            & \text{primal feasibility} \\
\mu_i g_i = 0 \ \forall \ i \qquad           &\text{complementary slackness} \\
\nabla f = \sum _i \mu _i \nabla g_i + \sum _j \lambda _j \nabla h_j \qquad & \text{stationarity}\end{split}\]

That's \(k + l + n\) equations for \(k + l + n\) unknowns! Now stick the
system of equations into a (probably numerical) zero-finder and you're
done!

\subsubsection{\texorpdfstring{On minimizing
\(f\)}{On minimizing f}}\label{on-minimizing-f}

Let's not forget that we had done all this assuming we were
\emph{maximizing} \(L\) (and thus \(f\)). If we were \emph{minimizing}
\(f\), we would be trying to minimize \(L\) and we could change the
above equations in one of the following ways:

\begin{enumerate}
\def\labelenumi{\arabic{enumi}.}
\tightlist
\item
  Replace \(L\) with
  \(L(x,\mu_1, \cdots, \mu_k, \lambda_1, \cdots, \lambda_l) = f(x) - \sum_i \mu_i g_i(x) - \sum_j \lambda_j h_j(x)\)
  (note the minus signs). Replace the stationarity condition with
  \(-\nabla f = \sum _i \mu _i \nabla g_i + \sum _j \lambda _j \nabla h_j\)
  (again, note the minus sign).
\item
  Keep \(L\) the same. Replace the dual feasibility condition with \(
  \mu_i \leq 0 \)
\end{enumerate}

The sign changes just ensure that our dual problem is still well-formed
for the minimization problem (if you don't flip the sign somewhere, then
we can just make arbitrarily ``good'' values by playing with \(\mu_i\)
again; and if you flip the sign in the formulation of \(L\), you have to
make sure you update the gradient expression accordingly (Option 1)).

\subsection{If it's not that bad, why did you talk so
much?}\label{if-its-not-that-bad-why-did-you-talk-so-much}

...OK, so maybe this was a lot more to explain than I had given credit
at first. But it's really not that bad! The main intuition is that either:
\begin{enumerate}
  \tightlist
  \item
    the inequality constraint is lax and doesn't actually do any
``constraining''; or 
  \item 
    the inequality constraint is actually stopping us,
    in which case it just becomes another equality constraint!
\end{enumerate}
All the blah-blah-blah is to make sure we dotted our \(i\)'s and crossed
our \(t\)'s. (And boy, did we...)
\\
\\
- DK, 4/30/18

\newpage


\chapter{Machine learning methods.}\label{machine-learning-methods}



\section{Boosting.}\label{boosting}

There's a lot of buzz about the winning models of many Kaggle competitions 
use gradient-boosted trees. Considering its apparent effectiveness, it'd be worth understanding
what ``gradient-boosted trees'' are and how they work. But before we jump into \emph{that}, 
we should probably figure out what ``boosting'' means in this context.

\subsection{An analogy to Taylor expansions.}


\textbf{Boosting}\index{boosting (machine learning)} is a meta-algorithm involving
the ensembling of many ``weak'' learners to form arbitrarily ``strong'' learners.
A ``weak'' learner is a classifier/regressor (which we'll just call a \emph{predictor}) 
that can only be weakly correlated with the true underlying 
classication/regression (\emph{prediction}) model we are trying to 
learn. A ``strong'' learner should be able to approximate the true prediction model
arbitrarily closely.

To make sense of the somewhat vague denotation above, 
I like to think of writing a Taylor expansion of some (presumably non-polynomial)
function, say \(f(x) = e^x\). Let's say we're expanding about \(x_0 = c\).\footnote{
  Since the Taylor expansion is focused on being accurate
  \emph{in the neighborhood of the point the expansion is centered around}, 
  we can interpret this ``Taylor model'' as weighting inputs/examples near \(c\) 
  as far more important than examples from other parts of the input space.
} 
Then the n'th-order Taylor expansion is
\[ T_{n,c}(x) = \sum_{i=0}^{n} f^{(i)}(c) \times \frac{(x-c)^i}{i!}\]

In a sense, each term of the sum is weakly correlated with the target function $f$ in that
if we consider the \(i\)'th term \(T_n[i]\), its \(i\)'th derivative is equal to 
that of \(f\) at \(x = c\) and so can be called a ``weak predictor'' of \(f\):
\[T_{n,c}[i]^{{i}}(c) = f^{(i)}(c) \]

Another important note is that none of our weak predictors are redundant; each of them
contains at least \emph{some} new information about \(f\).\footnote{
  If you fancy appropriating a linear algebra term, you can also consider each of the weak
  predictors ``linearly independent'' of each other. 

  This is actually more on-the-nose than
  you might think. Just as we can think of vector spaces of \(R^n\) as being spanned by
  \(n\) linearly independent basis vectors, we can think of a 
  \emph{function space}\index{function space} being
  spanned by an infinite number of linearly independent basis vectors, some of which
  are \((x-c)^0, (x-c)^1, (x-c)^2, \cdots\). 
  So a Taylor expansion approximates a function \(f\) using a linear combination of
  only the basis vectors corresponding to polynomials, with each basis vectors scaled 
  by a certain amount (\(f^{(i)}(c) \times {i!}\)).
}
Alone, \(T_n[i]\) is a pretty underwhelming approximator \textendash{} it's only guaranteed to
approximate \(f\) in a specific way at one specific point. 
That isn't necessarily useful for our goals \textemdash{}
if we tried approximating \(e^x\) with the Taylor expansion's second term
\(T_{n,c}[i = 1](x) = (x - c)\), we'd be off by hundreds, thousands, millions almost everywhere!

The power of our ``weak predictors'' comes when we \emph{combine} them in some way 
\textemdash{} in the case of the Taylor expansion, a uniformly weighted sum. 
As we use more terms (i.e., more ``weak predictors''),
the ensemble becomes arbitrarily accurate to the target function \(f\) near \(c\) and so is
a ``strong predictor'' of \(f\).



\subsection{Meta-algorithm vs. algorithm}\label{meta-algorithm-vs-algorithm}

Importantly, we said that boosting is a \emph{meta}-algorithm, a meta-strategy that we
apply onto a strategy that \emph{actually} performs the approximating. 
In our ``Taylor boosting'' method, the strategy we use to approximate \(f\) is to create an
approximator \(T_{n,c}[i]\) that has the same \(i\)'th derivative of \(f\) at \(c\), and the
meta-strategy was to combine all those approximators together via addition. We can perform
boosting on other strategies. For example, consider a ``Dirac boosting'' method, where our
weak approximators are functions \(D_i\) where
\[D_i(x) = \delta(x-i) \times f(x) = \mathbbm{1}\left[x = i\right] \times f(i) \]

where \(\delta\) is the Dirac delta function and \(\mathbbm{1}\) is the indicator function.
So basically, our approximator memorizes the function at a single point perfectly and guesses 
that it equals zero everywhere else. Alone, this weak predictor is pretty terrible, but
we can perform ``Dirac boosting'' by combining many such Dirac approximators \(D_i\) in the
following way: given an input \(x\), we find the Dirac approximator whose center is closest
to \(x\), and output that as the result.\footnote{
  (Basically a \emph{k-nearest neighbors} regressor where \(k=1\).)
} 
The greater the number/density of Dirac approximators along the input space,
the more accurate our boosted approximator becomes!\footnote{
  Fun fact: Although there are cases where a Taylor expansion can perfectly recreate the
  target function (e.g., \(f(x) = e^x\)), our ``Dirac boosting'' method cannot \emph{ever}
  perfectly recreate \emph{almost any} function whose input space is over \(\mathbb{R}\), even if
  we use an infinite number of Dirac approximators in our ensemble! More specifically,
  it can never perfectly recreate any function \(f\) if
  \[\lambda\left(\{x \mid f(x) \neq 0\}\right) > 0\]
  where \(\lambda\) is the \emph{Lebesgue measure}\index{Lebesgue measure}.
  This is because 
  our Dirac ensemble can only be made up of a \emph{countable} number of approximators which
  each only memorize a single output, and it can be shown (via, e.g., Cantor's diagonlization
  argument) that the set of real numbers is a larger sort of infinity
  (whose size is 
  \emph{uncountably infinite}\index{sizes of infinity!uncountably infinite}) 
  than the set of natural numbers (whose size is
  \emph{countably infinite}\index{sizes of infinity!countably infinite}).
}

Even though we changed the algorithm by which we approximated our function (using our Dirac
approximators instead of Taylor approximators), we still employed the \emph{meta}-algorithm
of boosting to combine our weak predictors into a strong predictor. This should hopefully make
the distinction between the two clear.

\subsection{Where the analogy is left wanting.}\label{where-the-analogy-is-left-wanting}

Now of course, the analogy isn't perfect. A Taylor expansion is performed on a known target
function \(f\), but in a data science context, we don't have access to the actual
function that is responsible for our data. We \emph{do} have a dataset, a set of samples from
the actual function, which can serve as an \emph{approximation} of the true function.
So the best we can do is to use the data to create such an approximation (our model). 

Since the dataset is almost certain not to contain all of the intricacies of the underlying
function, we normally don't want to make our model fit the data \emph{too} well
(good ol' \emph{overfitting}\index{overfitting}). Instead, we tune our model so that it
minimizes a
\emph{loss function (or objective function)}\index{loss function}, 
which we hope we've set up cleverly enough so that
the model is a really good approximation of the underlying function when it reaches a minimum
of the loss function (for example, by introducing terms in the loss function that punish the
model for overfitting the dataset). We didn't specify an explicit loss function when
improving our ``Taylor'' and ``Dirac'' ensemble examples since we could explicitly observe the
actual function we were trying to fit.

% Keeping in mind all these
% \sout{caviar} caveats, we cam use the Taylor expansion as an example of 
% ``making a strong predictor by ensembling weak predictors''.

\subsection{Gradient boosting.}\label{gradient-boosting}



Alright then, so what is gradient boosting? 
% It's when we construct our \(i+1\)'th weak predictor
% based on the residuals of the target value (serving as a sort of predictor for the gradient
% of the loss function) and the value we predicted with our best-fit
% ensemble of the first \(i\) weak predictors, then weight the \(i+1\)'th weak predictor so that
% the ensemble of the first \(i+1\) weak predictors minimizes the loss as much as possible.

% That was a bit of a mouthful, so let's break it down a bit. 
To follow the usual algorithm
on which gradient boosting is employed, we'll use decision trees.
We have our model \(M(x)\), training examples \((x_a, y_a)\) and a loss function
\(L(y_a, y_p)\), where \(y_p = M(x_a)\) is our current best prediction for \(y_a\).\footnote{
  It's worth
  remembering that the loss function can be as fancy as we'd like as long as its gradient
  can be computed analytically from the actual and predicted values.
  % For example, if we had target vectors and we only cared about being in the same direction
  % (and magnitude didn't matter), the loss function could be the cosine of the angle between
  % the two vectors:
  % \(L(y_a, y_p) = \frac{y_a^Ty_p}{\lvert y_a \rvert \lvert y_p \rvert} + 1\)
}
We'll be creating various weak learners \(h_i(x)\), and we'll denote our ensembles of the first
\(i\) weak learners (including \(i=0\)) as \(E_i(x)\). We decide that the way we're going
to ensemble our weak learners is by a summation \(E_i(x) = \sum_{i} h_i(x)\),
and we enter the rodeo.
% for some \(\gamma_i\) for each learner that's learned during training 
% (as shall be explained below).

We start with a baseline prediction, the mean of \(y_a\), i.e. 
\(M(x) \leftarrow E_0(x) = h_0(x) = \bar{y_a}\).
Now, we're going to create and fit a new weak learner \(h_1(x)\). But since we're going to
make our stronger learner via \(E_1(x) = E_0(x) + h_1(x)\), we can just fit \(h_1(x)\)
to a sort of \emph{pseudo-residual determined by the loss function}, 
\(y_{pr,1} = - \frac{dL}{dE_0} \Bigr|_{\left(x_a, y_a\right)}\) 
for all \(\left(x_a, y_a\right)\) in our dataset. 
We then fit \(h_1(x)\) to the dataset of
pseudo-residuals \((x_a, y_{pr,1})\) as is appropriate for our weak learner \textemdash{}
in the case of decision trees, we perform recursive partitioning until some 
user-specified condition is met (e.g. any further partitioning would yield leaves with
fewer than \(c\) entries, or the \href{https://en.wikipedia.org/wiki/Information_gain_ratio}
{information gain ratio} of any such partition would be smaller
than some threshold) and then taking the mean of each leaf as the predictor for any
inputs that fall into that leaf at prediction time).
% \footnote{
%   If our weak learner were instead, say, \(h_i(x) = k_i \times x^i\), our method of fitting might
%   instead be to pick \(k_i\) such that the mean-square-error over the examples is minimized.
% }
% Now, we want our new ensemble \(E_1(x) = E_0(x) + \gamma_1 h_1(x)\) to minimize \(L\) as much
% as it can (or at least to improve the performance by at least a certain amount), so you find 
% \(\gamma_1\) that \(\frac{dL}{dy_p(\gamma_1)} = 0\). Since we've made our loss function
% easily differentiable by \(y_p\) and \(y_p\) has a simple relationship to 
% (For tree-based models, you can perform this for each individual leaf.)
% Now that you found \(\gamma_1\),
Now that we've fit \(h_1(x)\) to the pseudo-residuals as best as we can, adding \(h_1\) to our
previous best predictor will further decrease the loss function, so we 
update our model \(M(x) \leftarrow E_1(x) = E_0(x) + h_1(x)\). 

Now we're sort of at the same place we were at earlier, just with a slightly better model.
So we can create a new weak learner \(h_2(x)\) by fitting it to the ``second-order''
pseudo-residuals \(y_{pr,2} = - \frac{dL}{dM} \Bigr|_{\left(x_a, y_a\right)}\) 
(where now \(M = E_1(x)\)),
then find the constant multiplier that minimizes the
main loss function, then we update our model \textemdash and on and on we go.

Here, we can view the \((i+1)\)'th weak predictor as attempting 
to approximate the \emph{gradient}
of the loss function when the previous best predictor was used.
We fit to the \emph{negative} of the derivative since we're trying to \emph{minimize}
our loss function, so we want to correct toward the direction opposite the derivative
(and since we're ensembling via addition, the ``opposite'' part comes into play
at the pseudo-residual level).
Since the gradient of the loss function determines how we create our weak learners/ensemble, 
we call this method \textbf{gradient boosting}. And it apparently works wonders
when used with decision trees as the weak predictors.\footnote{
  For the right sorts of problems, when you tune it correctly!
  }

The expression for our pseudo-residuals may seem to come out of nowhere, but let's
consider \emph{actual} residuals for a moment: \(y_r = y_p - y_a\). When we fit
our weak learner to try and predict the negative of these residuals \(\{-y_r\}\) 
and add this to our ensemble, 
we're taking a step in minimizing the mean-square-error
of our predictor: \(L = MSE = \frac{1}{2} \left(y_p - y_a\right)^2\). And you can see that
\(- \frac{d(MSE)}{dy_p} = -y_r\). So rather than stick to just this one type of residual,
why not fit to the gradient of whatever loss function we'd like? 
Hence the term and our expression for ``pseudo-residuals''.

(Note: For further references and useful links, see this footnote.)\footnote{
(The explanation provided by Abhishek Ghose in \href{https://www.quora.com/What-is-an-intuitive-explanation-of-Gradient-Boosting}
{this Quora post} is quite good and helped me properly grasp
the concept of gradient boosting. Other main reference was
\href{http://blog.kaggle.com/2017/01/23/a-kaggle-master-explains-gradient-boosting/}
{this Kaggle blog post}. \href{http://arogozhnikov.github.io/2016/07/05/gradient_boosting_playground.html}
{This page} allows for an interactive demo of gradient-boosted decision trees in action.)
}
\\
\\
- DK, 5/17/18 (shoot, it's late again...)

\chapter{Physics.}\label{chapter:physics}



\section{The Boltzmann distribution}



I have an ultimate goal of having a good understanding as to why classical mechanics
fails to predict blackbody radiation properly and leads to the so-called ultraviolet
catastrophe. The situation involves analyzing the classical formulation of the
Rayleigh-Jeans law, which relies on the classical equipartition theorem, which for specific
cases can be derived from classical statistical mechanics.
\\
Eventually, we'd like to also take a look at \emph{why} particles absorb the frequencies
they do, which involves understanding molecular orbital theory, and so the Schrodinger equation,
and so the Hamiltonian, which implies Hamiltonian mechanics, which is apparently an improvement
on Lagrangian mechanics, which is an application of the Lagrangian (which 
we've talked about before at
\ref{sec:the-lagrangian-a-packaged-function}
and so would be our
jumping-off point).
\\
But first, counting. (And by the way, a lot of the insight I gained came from
\href{https://courses.physics.ucsd.edu/2017/Spring/physics4e/boltzmann.pdf}
{this source} (which isn't explicitly mentioned in this writeup itself).)

\subsection{Stating our problem.}

Let's say we have a system with \(N\) \emph{indistinguishable}
particles 
(i.e., the particles are all of the same ``type'') 
and a total energy
\(E\) that can be swapped freely between the particles at increments of \(\Delta E\) (and
\(E\) must be an integer multiple of \(\Delta E\)), say
\(E = k\Delta E\). We presume that our system is in 
thermal equilibrium and no energy or particles enter or exit the system (i.e. that the system
is \textbf{closed}, and more specifically a
\emph{\href{https://en.wikipedia.org/wiki/Microcanonical_ensemble}{microcanonical ensemble}}.).
Given this situation, how can we figure out the probability
that this system is in a particular state?


Before we continue, we need to determine exactly what we mean by ``state''. We consider
two different levels of a state's description:
\begin{enumerate}
  \tightlist
  \item
    \emph{macrostate}: This description only has a summarized view of the system. In this case,
      it describes how many particles at each energy (in terms of a multiple of \(\Delta E\)).
      \footnote{
      The macrostate focuses on energy because of energy's importance in physical systems.
      Minimizing some function of energy is often used to solve problems in Lagrangian and
      Hamiltonian mechanics, and more generally in
      \href{https://en.wikipedia.org/wiki/Calculus_of_variations}{the calculus of variations}
      (which I hope to better understand and write about at some point).
      }
  \item 
    \emph{microstate}: This description holds information not only about the macrostate,
      but also about other parameters that would predict how the macrostate would evolve
      (over time). In our case, in addition to the
      energies of each particle, 
      we'd need to know each particle's position and momentum to know which particles collide
      with which other particles on the next ``timestep'' of the system's (\emph{dynamic})
      equilibrium. In this case, since we know each particle's momentum, 
      \emph{we also know each particle's energy.}
\end{enumerate}

We're interested in determining the probability that a system is in some particular
\emph{macrostate}.

\subsection{A lazy way out.}
One way we could potentially figure out our macrostate distribution
is by starting at some given
microstate, and observe the system evolve over a really long time, recording the macrostates
at every differential step in time. The assumptions we'd be making were that:
\begin{enumerate}
  \tightlist
  \item
    the entire 
    microstate space is connected, i.e., 
    that there is a path from any one microstate configuration
    to any other microstate configuration (otherwise, we'd miss the macrostates
    associated with the disconnected microstates); and that
  \item
    the proportion of times a macrostate appears in our record of the system's evolution
    corresponds to the probability that the macrostate would occur. \footnote{
      This is actually an important assumption \textemdash{} that observing one system for
      a very long time tells us information about what a distribution of a large ensemble
      of systems would look like, and vice-versa.
      Such systems are called \href{https://en.wikipedia.org/wiki/Ergodicity}{ergodic systems}.

      It's also important to remember that this is an \emph{assumption}. There are cases
      where a system can be shown not to be ergodic.
    }
\end{enumerate}
The specifics of how the simulation ran would also imply other assumptions, which will
be discussed below. But we can approach it through
more direct means.

\subsection{Counting (is hard).}\label{subsec:counting-is-hard}

Let's try to count how many microstates correspond to each macrostate. Before we can
do that, we need to determine all the valid macrostates.
%  remembering that
% we're considering indistinguishable particles. 
We denote the number of particles with energy \(i\Delta E\) as \(n_i\) 
(in terms of integer multiples of \(\Delta E\)) that particle \(i\)
has as \(n_i\), then the valid macrostate configurations are
vectors \(\vec{n} = (n_0, n_1, n_2, \cdots, n_k)\) such that %its first moment equals N:

\[\sum_{i=0}^{k} i \times n_i = k, \quad \sum_{i=0}^{k} n_i = N\]

Figuring out which vectors \(\vec{n}\) for which this holds is a bit difficult to describe
cleanly. The problem can be described as a form of the
\href{https://www.geeksforgeeks.org/dynamic-programming-set-7-coin-change/}
{coin change problem}.\footnote{
  Which is itself a specific form of the
  \href{https://en.wikipedia.org/wiki/Knapsack_problem}{knapsack problem}.
}
\index{coin change problem}
In our case, the coins have value \(1, 2, \cdots, k\), and we remove sets of cardinality
greater than \(N\).\footnote{
  In this case, we're allowing sets to contain duplicates of the same element.
 }
The remaining sets are distinct valid macrostates. We can get each
set into the form of \(\vec{n}\) by setting \(n_i\) as the number of times \(i\) appears in
the set, and setting \(n_0 := N - \sum_{i=1}^{k} n_i\).
We'll refer to the set of all valid macrostate configurations as \(V\).


Now that we have valid macrostates, we can count how many \emph{different} 
microstates yield a given
macrostate. If all \(N\) particles were distinguishable, then every rearrangment of them
would result in a distinguishable microstate that would correspond to the same macrostate,
and so the number of microstates that make up a macrostate would be \(N!\) for any given
macrostate.

However, we're dealing with \emph{indistinguishable} particles, so we wouldn't notice if
we swapped around particles at the same energy state \(i\). We have to divide out these
redundant occurrences, because otherwise we're overcounting the \emph{number} of different
microstates.
There are \(n_i\) such particles at any given \(i\), and so the total number of different
microstates for a given macrostate \(\vec{n}\) would be
\[w\left(\vec{n}\right) = \frac{N!}{\prod_{i=0}^{k}n_i!}\]

If we make the reasonable assumption that all distinguishable microstates are equally likely
(after all, if none of them are more energetically favorable than any of the others, 
why would there be favorites?),
\(w(\vec{n})\) provides a way of \emph{weighting} the different macrostates 
\(\vec{n}\).
\footnote{
  If, for some reason, this assumption doesn't sit well, 
  one need only tweak \(w(\vec{n})\) to their liking.
  But considering experiments match the previously made assumption, Occam's
  Razor seems relevant.
}
% \footnote{
%   If, for some reason, you wanted to weight microstates differently, you'd have to
%   use \emph{generating functions}\index{generating function}, as described in
%   \ref{sec:generating-functions}. In that case, our generating function would index
%   by microstate (with the probability of that microstate being the index's coefficient)
% }
We can calculate the probability that the system is in a \emph{macrostate} corresponding
to \(\vec{n}\) via a proportion:
\[P(\vec{n}) = \frac{w(\vec{n})}{ \sum_{\vec{n}' \in V} w({\vec{n}'}) } \]

\subsection{Counting Is Hard II: Probability Strikes Back.}

\emph{6/6/2018 Update:} It has come to my attention that
\href{https://courses.physics.ucsd.edu/2017/Spring/physics4e/boltzmann.pdf}{my original main source}
could have been a bit clearer by stating that the assumption of a system with
total fixed total energy
(a \emph{microcanonical ensemble}, which eventually reaches thermal equilibrium within itself)
is appropriate for the discussion of the previous section,
whereas a more ``relaxed'' assumption of a system with fixed \emph{temperature}
\footnote{
  Remember that temperature (``intuitively'', based on kinetic theory) 
  is related to the average \emph{kinetic} energy
  of particles in a system. Therefore, a fixed temperature only implies a fixed total
  \emph{kinetic} energy within a system, not total energy (which could also include
  potential energies of various forms).
  \par
  It's worth noting that in thermodynamics,
  temperature is actually used as an
  \href{https://en.wikipedia.org/wiki/Temperature\#Temperature_as_an_intensive_variable}{intensive variable of the system}, 
  but it would take quite a while to justify/explicate that.
  The above kinetic-theory way of thinking about temperature
  should be good enough to explain where the ``extra'' energy
  could come from in a system with fixed temperature.
}
(a \emph{canonical ensemble}, which reaches thermal equilibrium with the system's surroundings,
an entity which can supply energy into the system)
is more appropriate in the following discussion wherein we
end up assuming independence of energies between particles inside the system 
\textemdash{} they can independently ``get'' their energy from the system's surroundings
and store it as potential energy, thereby ``gaining'' energy without violating the
fixed temperature stipulation.
(No wonder I needed to go through so much mental gymnastics to try and rationalize
the independence step...)

\par
Currently the below explanation remains unchanged since this discovery.
Once the section is adjusted, this note will no longer precede it.
\\
\\
So we've figured out the probability of the \emph{system} being in some \emph{macrostate}.
But what if we're interested in individual particles?
What would be the probability that, given some total
amount of energy in the system \(E_{{T}}=D\Delta E\),
a particle would be at energy level \(i\Delta E\)?

Well, we know the probability of being in any particular macrostate, and within each
macrostate, we know the fraction of particles at energy level \(i\Delta E\) 
(that being \(n_i)\),
so we do a weighted sum over the macrostate probabilities:
\[Pr[\text{particle in system has energy }i\Delta E] = P(i) =
\sum_{\vec{n}\in V}n_i \times P(\vec{n})\]

and we can have the expected number of particles in a particular energy level (assuming
\(N\) particles in the system):
\[\left<n_i\right> = \left<n\right>(i) = N \times P(i)\]

So what's the shape of \(P(i)\) (or \(\left<n\right>(i)\)) for large \(N\) and as 
\(\left(D,\Delta E\right) \rightarrow \left(\infty,0\right)\)?\footnote{
  The two limiting cases serve different purposes. We're trying to fill in our graph
  of (y = number of particles) vs (x = energy level). Large \(N\) makes the ordinate
  take on more continuous values (making it be better approximated by a continuous function),
  while \(\Delta E \rightarrow 0\) lets the abscissa take
  on more continuous values (which gives more points onto which one can fit a function).
} 
Well, let's consider how we expect the probability function to behave under a few actions.


First, let's figure out the probability that some particle is at energy \(i\) and
some other particle is at energy \(j\). The usual expression would be
\(Pr[i,j] = P(i) \times P(j \mid i)\), involving a conditional probability.
But wait, we're assuming that we have an arbitarily large
number of \(\Delta E\) that we're able to partition. If that's the case, then no matter
what \(i\) is, there's still (approximately) the \emph{same} number of \(\Delta E\) from which
the second particle can take its \(j\) \(\Delta E\)'s. That is, the probability distribution
(basically) \emph{doesn't change} after taking away \(i\) of the \(\Delta E\)'s, so
\(P(j \mid i) \approx P(j)\) and \(Pr[i,j] \approx P(i) \times P(j)\), implying that
the energy levels of two individual particles are (approximately) \emph{independent}. 
The approximation becomes more and more exact the more slices of energy there are, i.e.,
the closer \(\Delta E\) is to zero, 
and when \(\Delta E = dE\), independence holds exactly.\footnote
{
  This is because 
  there would always be an infinite number of \(\Delta E\)'s to choose from, no matter
  how many had been taken up via assignment to previous particles
  (since \(D \rightarrow \infty\))\textemdash{} a weird thought,
  but would technically be true.
  \par
  Worth noting that assuming that particle energy levels are 
  completely independent is actually a bit crazy-making.
  If any particle can be at any energy level in \(\left[0, E_T\right]\), then we can have, say,
  two particles with energy level \(2E_T/3\) and exceed the total energy of the system!
  Indeed, according to the model, we could change the total energy of the system by a factor of
  \(N\). Or indeed, have it be zero.
  \par
  And in fact, having the step size be infinitesimal would mean
  that \(idE = 0 \  \forall i \in \mathbb{N}\), so that every particle would have to have
  an \emph{uncountably infinite} multiple of \(dE\) in order to have \emph{any} energy.
  (Differentials are weird.
  As is approximating a discrete situation with a continuous probability function.)
  \par
  All this to say, we should remember this an \emph{approximation} and we should
  keep in mind how if we take the approximation to be true,
  we're bending our problem statement in quite an exotic way.
  Alas, this is the price we pay for clean derivation results.
}\label{foot:independence-weirdness}
% \footnote
% {
%   The assumption of independence can be made less crazy-making if we assume
%   only fixed temperature
%   (i.e., that the system is in thermal equilibrium with its surroundings)
%   and not fixed total energy.
%   In such a situation, individual particles could ``get'' their 
%   energy from the surroundings and ``store'' it as potential energy of some sort,
%   thereby allowing a method for particles to have energy states independent of each
%   other (assuming that the effect of other particles in the system
%   on a particle's potential energy is zero or negligible \textemdash{} which
%   is true for situations like a system of traveling light particles
%   or chemical particles with no (or mostly screened) charge in solution).
% }
So, in the limiting case, \(Pr[i,j] = P(i) \times P(j)\).


What about the probability that sum of two particle's energies equal \(i+j\)?
Once again, as argued above, in our limiting case, the energy of a particle is independent
of the erngy of any other particles. So the probability we're looking for is independent
of all the other particles in the system \textemdash{} it only depends on the two we're looking
at. Then we could describe the probability as
\[Pr[i+j] = 
  \sum_{k=0}^{\frac{i+j}{\Delta E}}\left(P(k)
  \times P\left(\frac{i+j}{\Delta E} - k\right)\right)
  = q(i+j) 
  \]

which enumerates over energy pairs
\( (0,\frac{i+j}{\Delta E}), (\Delta E,\frac{i+j}{\Delta E} - \Delta E), \cdots,
  (i, j), \cdots, (\frac{i+j}{\Delta E}, 0) \).


We've previously assumed that all microstates have equal probability of occurring.
Then that means \emph{all} of the energy pairs above (which are essentially microstates of a
two-particle ensemble) occur with equal probability.
Then
\[Pr[i+j] = q(i+j) = \left(\frac{i+j}{\Delta E} + 1\right) \times Pr[i,j] = B \times Pr[i,j]\]

But we saw earlier that \(Pr[i,j] = P(i) \times P(j)\), so

\[q(i+j) = B \times P(i)P(j)\]

That means that 
\(P\) must be some function such that \(P(i)P(j)\) becomes a function of \(i+j\) \emph{only} (because \(q(i+j)\) is, as we described above, a function of only \(i+j\)).
And the only type of function with such a property is an exponential function:\footnote{
  Note that the base could be any positive number that isn't 1 and the statement would still
  hold true. \(e\) is just the most convenient base to use.
}

\[Pr[\text{particle in system has energy }i\Delta E] 
              = P_{+}(i) = C_{+}\times e^{a_{+}i}
\]

where \(C_{+}\) and \(a_{+}\) have yet to be determined. We'll do that soon.
But you know,
I think it's a \emph{tad} unlikely that a particle is more likely to be in high
energy levels than low ones (especially considering our calculations have shown a decrease in
probability as energy level increased),
so let's first guess that the exponent is negative \textemdash{} if we happen to be wrong,
the as-yet-undetermined parameter in the exponent will end up being negative:

\[Pr[\text{particle in system has energy }x\Delta E] = P(x; \Delta E) = C\times e^{-ax} \]

And finally, let's make an alternate expression in terms of energy directly, as opposed to
in terms of discrete energy levels, as deriving the Boltzmann distribution will rely on
an upcoming assumption on the energy specifically:

\[Pr[\text{particle in system has energy }E] = p(E) = Ae^{-bE}\]

It's worth noting that this is actually a pretty big shift in frame. \(P(x)\) implies the energy
is discretized by a step size \(\Delta E\) and has arguments \(x \in \mathbb{N}\) no matter
how small \(\Delta E\) becomes. The equivalent point for \(P(x)\) in the function \(p\)
is at \(E = x\Delta E\), that is,

\[P(x; \Delta E) = C\times e^{-ax} \mapsto p(x\Delta E) = Ae^{-bx\Delta E} \]


That means unlike \(x\), as \(\Delta E \rightarrow 0\), the values
\(E\) can take on becomes more and more granular, 
to the point where \(E\) can become arbitrarily
close to any desired value.\footnote{
  Though technically, it can never be \emph{any} arbitrary value since the domain of \(E\)
  is always a subset of the rational numbers \(\mathbb{Q}\).
} Also, if an explicit upper bound of \(E_T\) isn't placed on the argument of \(p\)
(which is not done in the original derivation of the Boltzmann distribution 
and so will not be done here), then we're now allowing particles to
(very improbably, but technically) be able to take on arbitrarily large energies!

\subsection{A fork in the road.}

Now we can proceed with the derivation in two ways:
\begin{enumerate}
  \tightlist
  \item
    \emph{Double-down on the fact that we're analyzing cases where
     \(\Delta E \rightarrow 0\); that is to say,
    make \(\Delta E = dE\) and \(E\) continuous}:
    This would arguably be the more logical of the choices given our preceding argument.
    In order for our exponential function to fit the probability distribution \emph{exactly},
    we needed the energies of two different particles be independent of each other,
    and in order for \emph{that} to have been the case we'd need to take \(\Delta E\) to 0.
    % However, it's worth keeping in mind that now we've contradicting our argument in
    % a different way now: 
    % \emph{our probability distribution \(p(E)\) impliies that our particles can have
    % arbitrarily large amounts of energy.} Of course, the probability is preposterously low,
    % but it is nonzero.
  \item
    \emph{Continue our analysis under the assumption that \(\Delta E\) is not a differential}:
    This technically aligns with how we understand physics today \textemdash{} i.e., that
    energy is quantized by the frequency of radiation,
    \(\Delta E(\nu) = nh\nu\), with \(n \in \mathbb{R}\) 
    and \(h\) as \textbf{Planck's constant}\index{Planck's constant},
    \(h \approx \SI{6.626}{J.s}\).
    In this case, we'll have to accept that our argument,
    having previously assumed continuous \(E\), 
    doesn't take ``full advantage'' of that assumption.
\end{enumerate}

We'll do the continuous case first, as it leads us to the form of the actual Boltzmann function.
We'll do the discrete case afterward for a bit of fun, and because one of the big
mathematical tricks needed in the derivation will be useful later on.\footnote{ 
  Specifically, it will be useful when trying to ``patch'' the classical equipartition theorem 
  and Boltzmann function in order to describe \emph{blackbody radiation}
  by having the average energy of an emitted electromagnetic wave 
  depend on both temperature
  \emph{and} frequency \textemdash{} but we're getting way ahead of ourselves!
}

\subsection{Planning our attack.}

But what's our plan of attack in finding out the correct parameters for our probability
distribution(s)? It's the same plan that can be used
when finding the correct parameters for \emph{any} probability distribution 
\textemdash{} derive relationships
via the distribution's moments. We can say some random variable \(\epsilon\) takes on
values \(E\) for the continuous case and some random variable \(X\) takes on values \(x\)
in the discrete case. Then we can say that:
\begin{itemize}
  \tightlist
  \item
    the zeroth moment equals 1 (because they're probability distributions).
  \item
    The first moment (centered around 0) equals the mean, 
    \(E[X]\) (also written \(\left<x\right>\).
  \item
    The second moment (centered around \(E[X]\)) equals the variance, \(E[X-E[X]]^2\).
\end{itemize}

...And so on. In this case, we'll only need to go as far as the first moment.\footnote
{
  As it turns out, a different result (the equipartition theorem) will tell us the value
  of \(\left<x\right>\) for a system at thermal equilbrium at a given temperature \(T\),
  so we'll eventually be able to pull the distribution out of the theoretical and into
  the practical \textemdash{} but again, we get ahead of ourselves.
}


\subsection{The continuous case: An easy battle.}

So right now we want to figure out \(A\) and \(b\) in

\[Pr[\text{particle in system has energy }E] = p(E) = Ae^{-bE}\]

using moments. Well, since we're treating energy as continuous, our moments
are integrals (which are \emph{much} nicer to deal with than series).

First, the zeroth moment of the probability distribution should equal 1:
 
\[\int_{E=0}^{\infty} Ae^{-bE}dE = 1 \]
\[  \cdots \]
\[  A = b \] 

OK, cool, we've stripped away a degree of freedom, but we need to remove one more
before we have something usable. Note also that the infinite integral would not converge
for exponentials with positive exponents. So common sense prevailed!
\\
Anyway, onto the next moment. The first moment of the probability distribution
should equal the mean of the random variable of our distribution:

\[\frac{\int_{E=0}^{\infty} AEe^{-bE}dE} {\int_{E=0}^{\infty} Ae^{-bE}dE} = \left<x\right> \]
\[  \cdots \]
\[  b = \frac{1}{\left<E\right>} \]

Awesome! We have the parameters for our probability distribution, and we have our final result:

\[Pr[\text{particle in system has energy }E] 
        = p(E) = \frac{e^{-E/\left<E\right>}}{\left<E\right>}
\]

% So given that these microstates form the weighting for the macrostates,
% what's the probability distribution of these macrostates?
% Boltzmann used the calculus of variations (using the constraint that solutions must
% maximize entropy (since we're at equilbrium)) to get kBT...

If you use the result from the classical equipartition theorem\footnote{
  Which we'll hopefully go over later.
}
, you'd have that
\(\left<E\right> = k_BT\) and so 

\[p(E) = \frac{e^{-E/k_BT}}{k_BT}\]

And \emph{that} is the \textbf{Boltzmann distribution}\index{Boltzmann distribution}
in all its glory!\par

% Although it may seem like all this work to derive the Boltzmann distribution 
% was done for a fairly meager cause
% \textemdash{} just for a group of particles? \textemdash{}
% its importance is tremendous.

\subsection{The Boltzmann factor.}

A convenient thing about the Boltzmann distribution is how simple it is to calculate
the relative probabilities of a particle being in one state over another.
If we have two states with energies \(E_1\) and \(E_2\), we can calculate their
relative probability by taking their ratio (we'll keep the denominator
of the function as \(\left<E\right>\) for now):

\[
\begin{split}
  \frac{P(E_2)}{P(E_1)} &= \frac{\frac{e^{-E_2/\left<E\right>}}{\left<E\right>}}{\frac{e^{-E_1/\left<E\right>}}{\left<E\right>}} \\
    &= \frac{e^{-E_2/\left<E\right>}}{e^{-E_1/\left<E\right>}} \\
    &= e^{-(E_2-E_1)/\left<E\right>} \\
    &= e^{-\Delta E/\left<E\right>}, \ \Delta E = E_2 - E_1
\end{split}
\]

% If we substitute in \(\left<E\right> = k_BT\), we get

% \[ \frac{P(E_2)}{P(E_1)} =  e^{-\Delta E/{k_BT}}, \ \Delta E = E_2 - E_1\]

% Rearranging to solve for \(\Delta E\), we get:

% \[ \Delta E = -k_BT\ln(\frac{P(E_2)}{P(E_1)}), \ \Delta E = E_2 - E_1\]

% The above two forms are seen all around chemistry and biology, often with
% \(\frac{P(E_2)}{P(E_1)}\) being written as some single dimensionless quantity
% (for example, \(K_{eq}\) or \(Q\) in chemical equilibrium situations)



The \emph{Boltzmann factor}\index{Boltzmann factor} is simply the relative probability
of a certain state compared to the state corresponding to \(E = 0\)
(however that may be described):

\[
\begin{split}
  \frac{P(E)}{P(0)} &:= F(E) \\
  &= \frac{\frac{e^{-E/\left<E\right>}}{\left<E\right>}}{\frac{e^{-0/\left<E\right>}}{\left<E\right>}} \\
    &= e^{-E/\left<E\right>}
\end{split}
\]

You'll see that the relative probability reflects the real-life observation that
objects in the world are more likely to be found in low-energy states than high-energy
ones (and so ``prefer'' low-energy states to high-energy ones). Note that a ratio
of Boltzmann factors gives a ratio of relative probabilities of the two states:

\[\frac{F(E_2)}{F(E_1)} = \frac{\frac{P(E_2)}{P(0)}}{\frac{P(E_1)}{P(0)}} \\
            = \frac{P(E_2)}{P(E_1)}\]

% For whatever reason, in statistical mechanics they like calculating probabilities
% using Boltzmann factors explicitly rather than integrating over the probability distribution,
% so rather than:

% \[p(E) = \frac{e^{-E/k_BT}}{k_BT}\]

% in (classical) statistical mechanics you'll often see:

% \[p(E) = \frac{e^{-E/k_BT}}{\sum_i e^{-E_i/k_BT}}\]

% where \(i\) enumerates through all possible microstates and
% \(\sum_i e^{-E_i/k_BT}\) is given the grandiose-sounding name
% \emph{partition function}\index{partition function!classical form}.
% \sout{(And that's the very least of the grandiose-sounding names.)
% \par

% Although we calculated the Boltzmann distribution for an ensemble of particles,
% the choice of the word ``particle'' was largely just for convenience. We can make the
% same argument and use the Boltzmann distribution 
% for an ensemble of \emph{anything} if they follow the assumptions we made above.



\subsection*{The discrete case: A battle of wits.}

Alright, now the discrete case. For ease of notation, we will work with \(P(x)\) instead of
\(p(E)\) \textemdash{} again, the mapping between the two is \(E \mapsto x\Delta E\), so it
isn't too bad to interconvert:

\[Pr[\text{particle in system has energy }x\Delta E] = P(x; \Delta E) = C\times e^{-ax} \]

Again, we have the normalization condition:

\[\sum_{x=0}^{\infty} Ce^{-ax} = 1 \]

This is a bit worrying at first. But note that

\[\begin{split}
  \sum_{x=0}^{\infty} e^{-ax} &= \sum_{x=0}^{\infty} \left(e^{-a}\right)^x \\
                      &= \frac{1}{1 - e^{-a}} = (1 - e^{-a})^{-1} \text{ if } \left|e^{-a}\right| < 1
\end{split}
\]

from the formula for the sum of an infinite converging geometric series.
\\
With that, we have that

\[ C = 1 - e^{-a} \]

OK, let's move on to the first moment to figure out what \(a\) is in terms of \(\left<x\right>\):

\[
\frac{\sum_{x=0}^{\infty} Cxe^{-ax}} {\sum_{x=0}^{\infty} Ce^{-ax}} 
= \frac{\sum_{x=0}^{\infty} xe^{-ax}} {\sum_{x=0}^{\infty} e^{-ax}}
= \left<x\right>
\]

This looks absolutely terrifying at first. And for quite a while. \sout{And possibly always.}
But there's a clever trick.\footnote
{
  Which I take no credit for. I saw this trick in Chapter 1 of
  \href{http://www.sicyon.com/resources/library/pdf/eisberg_resnick-quantum_physics.pdf}
  {Eisberg and Resnick's \textit{Quantum Physics of Atoms, 2nd ed.}}
}
Remember that in general,
\[\frac{d}{dx}\left(\ln{f(x)}\right) = \frac{\frac{df}{dx}}{f(x)}\]

In the same vein, we evaluate

\[\begin{split}
  \frac{d}{da}\ln{\sum_{x=0}^{\infty} e^{-ax}} &=
  \frac{\frac{d}{da}\sum_{x=0}^{\infty} e^{-ax}} {\sum_{x=0}^{\infty} e^{-ax}} \\
  &= \frac{\sum_{x=0}^{\infty}\frac{d}{da} e^{-ax}} {\sum_{x=0}^{\infty} e^{-ax}} \\
      &= \frac{-\sum_{x=0}^{\infty}xe^{-ax}} {\sum_{x=0}^{\infty} e^{-ax}} \\
      &= -\left<x\right> \\
\end{split}
\]

So then
\[
  \left<x\right> = -\frac{d}{da}\ln{\sum_{x=0}^{\infty} e^{-ax}}
\]

but we already saw that
\(\sum_{x=0}^{\infty} e^{-ax} = (1 - e^{-a})^{-1} \text{ if } \left|e^{-a}\right| < 1 \), so 

\[\begin{split}
  \left<x\right> &= -\frac{d}{da}\ln{\left(1 - e^{-a}\right)^{-1}} \\
      &= -\left(1 - e^{-a}\right) \times \frac{d}{da}\left(1 - e^{-a}\right)^{-1} \\
      &= -\left(1 - e^{-a}\right) \times \left(-\left(1 - e^{-a}\right)^{-2}\right) 
      \times \left(e^{-a}\right) \\ 
  \left<x\right> &= -\frac{e^{-a}}{1 - e^{-a}} = \frac{1}{e^a - 1} \\
\end{split}
\]
% We'll see this form again later.
% More specifically, again remembering our mapping
% \(x\Delta E \mapsto E, a \mapsto b\Delta E, C \mapsto A\),
% we'll see the following expression again:

% \[\left<E\right> = \left<x\Delta E\right> \\
%       = \left<x\right>\times\Delta E = \frac{\Delta E}{e^{b\Delta E} - 1} \]

% and at that point, we'll be trying to derive \(\left<E\right>\) 
% given \(b\) (as it's derived for the Boltzmann distribution)
% and \(\Delta E\).
% \\
% Anyway, let's solve for \(a\):
Now let's solve for \(a\):

\[a = \ln{\left(1 + \frac{1}{\left<x\right>}\right)} \]

Interestingly, this changes the base of our exponent:

\[e^{a} = \left(1 + \frac{1}{\left<x\right>}\right) \]

and so our probability distribution becomes:

\[P(x;\Delta E) = \frac{1}{1 + \left<x\right>} 
      \times {\left(1 + \frac{1}{\left<x\right>}\right)}^{-x}
\]

The form is reminscent of solutions to various simple steady-state problems \textemdash{} 
but alas, it's much less aesthetically pleasing and much less
convenient to manipulate/work with than the Boltzmann distribution!
\sout{Yet another reason why quantum phenomena can be unpleasant to some.}

\subsection{Bolting beyond Boltzmann.}

All of these calculations have been made with the assumptions that:
\begin{enumerate}
  \tightlist
  \item
    Particles within the system can be distinguished from one another by the 
    energy state they are in (and particles within the same energy level are
    indistnguishable about each other); and
  \item
    Any number of our identical particles can occupy any energy state.
\end{enumerate}

These assumptions do not hold for particles in the regime of quantum mechanics.
Specifically, in an ensemble of \emph{bosons}\index{boson} (particles with integer spin),
within a particular macrostate,
individual bosons can \emph{not} be distnguished from each other, even by energy level.
That means there are \emph{no} distinguishable microstates \textemdash{} the most granular
information you can get is from what we've been calling the macrostate
(the vectors \(\vec{n}\) of dimension \(\frac{E_T}{\Delta E}\) we would calculate via
solving the equivalent coin-change problem as mentioned in
\ref{subsec:counting-is-hard}).
Moving forward from there would eventually
lead to the \textbf{Bose-Einstein distribution}\index{Bose-Einstein distribution}.\par

If instead of bosons we're dealing with \emph{fermions}\index{fermion} (particles with
\(\frac{1}{2}\)-integral spin), then the \emph{Pauli exclusion principle} also applies,
meaning that our second assumption would \emph{also} be incorrect and only two such
fermions could be in the same energy state.\footnote{
  We could get the valid macrostates
  by solving the coin-change problem where the are only two coins per denomination
  (or solving it without the constraint then filtering out those sets with more than
  two coins of the same denomination).
}
Moving forward from \emph{there} would
eventually lead to the \textbf{Fermi-Dirac distribution}\index{Fermi-Dirac distribution}.
\\
\\
...Phew, that was exhausting!
\\
\\
- DK, 5/29/18


\chapter{Drafts and WIPs}\label{drafts-and-wips}

(Avert your eyes! Unless you're fine with drafts and WIPs \textemdash{} in which case, enjoy!)

% 6/8/18
% Relationship between carrier frequency and information/data rate?
% Some links:
% https://forums.anandtech.com/threads/is-there-a-relationship-between-information-data-rate-and-carrier-frequency.2051027/
% https://en.wikipedia.org/wiki/Shannon%E2%80%93Hartley_theorem





\section{Generating functions: dealing with uneven probabilities.}\label{sec:generating-functions} % 5/27/18

% https://math.stackexchange.com/questions/1646101/distribution-of-the-sum-of-n-loaded-dice-rolls













\section{Central Limit Theorem} % 5/27/18


% TODO: Try and get from
% Lagrangian -> Lagrangian mechanics -> Hamiltonian mechanics
% -> the Hamiltonian in quantum mechanics -> the Schrodinger equation
% -> molecular orbital theory -> Beer-Lambert Law and *absorption* of light in media
% next,
% what determines reflection vs transmittance (physics and light properties)
% finally,
% tie together to blackbody radiation and the ultraviolet catastrophe
% (interest in blackbody radiation came from reading it in the app Quantum)

% also before blackbody radiation, consider
% Boltzmann distribution / partition function
% approximation of multinomial distribution above with an exponential funciont
% via the *Central Limit Theorem* (explain)

% generator functions
% https://math.stackexchange.com/questions/1646101/distribution-of-the-sum-of-n-loaded-dice-rolls

% Boltzmann distribution (and statistics for for bosons and fermions) 
% -> one example of a derivation of the equipartition theorem
% -> classical Rayleigh-Jeans law -> UV catastrophe / blackbody radiation ()

% virial theorem

% generator functions to handle cases where states have unequal probabilities of
% occurring (a loaded die, for instance).

% ^ all of the above first wondered about ~5/26/18


% why is the reduced mass of a system calculated
% like the equivalent resistance of resistors in parallel? - 5/28/18

% What is up with the different ``forms'' of free energy? (Gibbs, Helmholtz, internal, enthalpy)

% what causes water bubbles at e.g. the base of a waterfall? (``Cavitation''?)

% harmonic mean vs geometric mean vs arithmetic mean: when is it appropriate to use which one?

% why do accelerating charges radiate energy?
% What sort of waves are emitted by moving matter? (Matter waves.)
% What is the *speed* of matter waves? Are they waves in a classical sense or a quantum sense?






% \section{Integral transforms.}

% While this certainly won't answer all questions about integral transforms, it can hopefully
% shed some light on ``where'' integral transforms come from.\par
% tl;dr: they can be thought of as changes of basis in a particular function space,
% with the basis vectors chosen based on what best fits the problem at hand.
% However, our transformation matrix (the kernel) is necessarily ``infinite-dimensional''
% in order to span the function space, so finding the inverse transformation matrix
% (and therefore the inverse (integral) transform) is nontrivial.
% % Like how a transformation on a vector \(T(x)\) can be modeled as matrix-vector multiplication
% % \(T_Mx\), the integral transform is like a continuous, infinite-dimensional matrix-vector
% % multiplication (with the function we're transforming as the vector and the
% % kernel of the transform being the matrix). Just as not all ``regular'' transformations are
% % invertible (the matrix representation needs to be non-singular), 
% % not all transforms have an inverse operation
% % (the kernel, our ``infinite-dimensional matrix'', may be singular),
% % and discovering the inverse transform is generally nontrivial.

% \subsection{Short refresher on vector spaces.}

% (In the following discussion,
% may be lazy and drop the \(\vec{\cdot}\) arrow where we believe the fact
% that it's a vector is clear.)

% We're already pretty comfortable with the idea of describing vectors as a linear combination
% of basis vectors. If we have some vector space \(V \subseteq \mathbb{R}^n\),
% % some \(\vec{v} := \left[v_1, v_2, \cdots, v_n\right]^T \in V\),
% some \(\vec{v} \in V\),
% and
% a basis \(\{\vec{b}_1, \vec{b}_2, \cdots, \vec{b}_n\}\) for \(V\), we can write
% \(v\) as a linear combination of basis vectors:

% \[\vec{v} = \sum_{i=1}^{n} a_i\vec{b}_i \]

% % or, if we build a matrix out of the basis vectors via 
% % \(D = 
% % \begin{bmatrix}
% %   e_1 & e_2 & \dots & e_n 
% % \end{bmatrix}
% % \)
% % and stack our coefficients into a vectors \(a = \left[a_1, a_2, \cdots, a_n\right]\),
% % then we can write

% % \[\vec{v} = aB\]

% Let's emphasize the fact that we chose \emph{a} basis for \(V\)
% and in most cases there is no such thing as \emph{the} basis for \(V\), since a
% \textbf{basis}\index{basis!in linear algebra} (of a vector space \(V\))
% is just a set of linearly independent vectors that span \(V\).
% (We'll refer back to linear independence in a bit.)\par

% Going back to \(\vec{v}\), if we agree of the basis of \(V\) we're working with,
% we can fully ``encode'' \(v\) via an n-tuple where the \(i\)'th location holds \(a_i\),
% the scalar coefficient by which we multiply the basis vector \(\vec{b}_i\).
% In that sense, we can say that \(\vec{v} = \left[a_1, a_2, \cdots, a_n\right]\).
% But we can always choose some \emph{other} basis
% \(\{\vec{\beta}_1, \vec{\beta}_2, \cdots, \vec{\beta}_n\}\)
% with which to express \(v\), in which case we'd (almost certainly) need different coefficients
% \(\alpha_1, \alpha_2, \cdots, \alpha_n\) in order construct the same vector.
% We haven't changed the \emph{vector spaces}, we've only changed how we represent points
% \emph{in} our vector space.\par

% \subsubsection*{Tangent: The ``meaning'' of vectors.}

% So, what are we doing when we change bases? What does a vector of
% \(\left[1, 0, 0, \cdots, 0\right]\) mean anyway?
% Arguably, the answer can be given in a ``hippie''-sort of way:
% \textit{``It means, like, whatever you want it to, man\texttildelow''}\par

% As an example, think of how one can solve a linear system of equations,
% say of three variables x, y, and z,
% using a matrix.
% Often, you'd represent an equation by a row vector comprised of
% the \emph{coefficients} of the three variables,
% augmented by the scalar value they equal. That implies that
% a left-hand-side row vector \([1, 0, 0]\) ``maps to'' the expression \(x\),
% \([1,0,0] \mapsto x\). That's exactly why you know you're done when you have a 
% diagonal of ones on the left-hand-side of the augmented matrix, for example the row
% \(
% \left[
% \begin{array}{ccc|c}
% 1 & 0 & 0 & a
% \end{array}
% \right]
% \)
% would mean that \(x=a\).
% But if there's some better way to encode the equations, we could always choose
% one of those instead, e.g.
% \([1,0,0] \mapsto \frac{x+y}{2}\), 
% \([0,1,0] \mapsto \frac{y+z}{2}\),
% \([0,0,1] \mapsto \frac{x+z}{2}\).
% It's all in \emph{your} hands \texttildelow\par

% \subsubsection*{Back to bases, and inner products.}

% All that said, when we change our basis from
% \(\{\vec{b}_1, \vec{b}_2, \cdots, \vec{b}_n\}\)
% to 
% \(\{\vec{\beta}_1, \vec{\beta}_2, \cdots, \vec{\beta}_n\}\),
% we can think of it as changing our mapping of basis vectors from 
% \(e_i \mapsto b_i\)
% to
% \(e_i \mapsto \beta_i\)
% where \(e_i\) is our standard basis vector for the \(i\)'th coordinate, 
% \(e_i \in V, \, e_i[j] = \delta(i,j)\).
% \par

% But that's only part of the story. How do we describe \(\vec{v}\)
% with our new basis? That is to say, how do we find the mapping for our coefficients
% \(\{a_1, a_2, \cdots, a_n\} \mapsto \{\alpha_1, \alpha_2, \cdots, \alpha_n\}\)
% so that they ``encode'' the same point \(\vec{v}\) in the vector space, i.e. that
% \[\sum_{i=1}^{n}\alpha_i\vec{\beta}_i \]
% and
% \[\sum_{i=1}^{n}a_i\vec{b}_i\]
% refer to the same point in the vector space?\par

% Well, so far we have no way of measuring ``how much'' of one vector is in another.
% To make this clearer, consider two vectors \(\vec{v}\) and \(\vec{w}\), both in \(V\).
% We want to create an operation that allows us to describe \(v\) as an addition of two vectors,
% one ``parallel'' to \(w\) and one ``orthogonal'' to \(w\)
% (both in the intuitive senses of the word):

% \[\vec{v} = \vec{v}_{v\parallel w} + \vec{v}_{v\perp w}\]
 

% If we have an operation that gets us one of the two component vectors, we can
% always define the other as a subtraction from the resultant vector.
% Turns out, we focus on similarity more (what a pleasant thought)!
% More specifically, we need to 
% define an \textbf{inner product} \innerprod{} over the vector space \(V\),
% which is just an operation that that takes in two elements from \(V\) and returns a scalar,
% and which also follows four key properties:
% \begin{itemize}
%   \tightlist
%   \item
%     Symmetry: \innerprod{u}{v} = \innerprod{v}{u}
%   \item
%     Linearity: \(\alpha \in \domain{R} 
%           \rightarrow \innerprod{\alpha u}{v} = \alpha \innerprod{u}{v}\)
%   \item
%     Distributivity:
%       \(\innerprod{u+v}{w} = \innerprod{u}{w} + \innerprod{v}{w}\)
%   \item
%     Positive Definiteness:
%       \(\innerprod{v}{v} \geq 0. \innerprod{v}{v} = 0 \rightarrow v = \vec{0}\)
% \end{itemize}

% A vector space with an associated inner product is called an \textbf{inner product space}.
% We can choose \emph{any} operation that fulfills these four requirements to be our
% definition of an inner product! There are some standard ones though.
% In the case of vectors in \(\domain{R^n}\), the standard is
% \(\innerprod{u}{v} = \sum_{i=1}^{n}u_i v_i \).
% With our definition of an inner product in hand, we can have our \textbf{projection}\index{projection}

% \begin{equation}
% \vec{v}_{v\parallel w} = \frac{\innerprod{v}{w}}{\norm[w]} \hat{w}
% \end{equation}\label{equation:projection}

% where \(\hat{w}\) is the unit vector in the direction of \(w\) and \(\norm{w}\)
% is the
% \textbf{norm} (which we'll discuss in a moment)
% of the vector \(w\).
% The remaining component \(\vec{v}_{v\perp w}\) is \textbf{orthogonal} to \(w\),
% meaning \(\innerprod{\vec{v}_{v\perp w}}{w} = 0\).
% \par 

% Conveniently, if \(\vec{w}\) is already of unit magnitude, 
% then \(\norm[w] = 1, \hat{w} = \vec{w}\), and
% \(\vec{v}_{v\parallel w} = \innerprod{v}{w}\vec{w}\).
% That is, the ``amount/number'' of a 
% unit vector \(w\) inside \(v\) is simply \innerprod{u}{v}.
% And if the unit vector \(w\) is a basis vector \(b_i\), then 
% \(\vec{v}_{v\parallel w}\) is 
% that basis vector, scaled by \innerprod{v}{b_i}!\footnote{
%   ``!'' used to denote excitement, not factorialization.
% }
% \par

% Now we can finally answer our question! 
% If we want to describe the same vector \(\vec{v}\)
% in terms of a new basis \(\{\vec{\beta}_i\}\) 
% we can calculate the coefficients \(\{\alpha_i\}\)
% simply by calculating \emph{the projection of \(\vec{v}\)
% onto each of the basis vectors \(\vec{\beta_i}\)}
% \(\frac{\innerprod{v}{\beta_i}}{\norm[\beta_i]}\)
% (with both \(v\) and \(\{\beta_i\}\) represented
% in terms of a common basis
% \textemdash{} presumably the natural basis \(\{e_i\}\) \textemdash{}
% as needed in order to calculate the inner product).

% \subsubsection*{Wait, what's a norm?}

% A \textbf{norm}\index{norm} is a measure of the ``size''
% of a vector in a vector space.
% One of the most common norms seen
% in linear algebra is
% the \(L^2\) norm 
% (the \(L\) may be named after \textbf{L}ebesgue), 
% given in the discrete form by
% \[\norm[w]_2 := \left(\sum_{i} \abs{w_i}^2\right)^{1/2}\]

% This corresponds to the notion of
% length/distance that is physically intuitive to us,
% i.e. the Euclidean distance (where absolute values 
% \(\abs{\cdot}\) around the inputs
% are usually not explicitly written because each term
% is raised to an even power and so are not strictly necessary).
% Most likely, if no explict definition of a norm is given,
% the \(L^2\) norm is probably implied.\par
% We can generalize quite readily to the \(L^p\) norm:
% \[\norm[w]_p := \left(\sum_{i} \abs{w_i}^p\right)^{1/p}\]

% The above works for discrete vectors. 
% For continuous vectors (which we'll be thinking about in a bit),
% we can adapt our definition of a norm
% by switching from summation to
% integration (let's say \(w(n)\) gives the coefficient at
% ``index'' \(n\)):
% \[\norm[w]_p = \left(\int_{n \in Dom(n)} \abs{w(n)}^p dn\right)^{1/p} \]

% We'll see this continuous form return when considering
% the norms of continuous functions. (It's almost as if
% functions are vectors or something...)

% \subsection{Functions, function spaces.}

% Now what if I told you that a function is just a vector?
% More accurately, a function \emph{can be viewed}
% as an element of a \textbf{function space} (which needs qualification \textemdash{}
% e.g. ``the space of all continuous functions'' \(\domain{C}^0\)), 
% with specific coefficients
% based on the choice of \emph{basis functions} we use to span the function space in question.

% \subsubsection{Power series and discrete function spaces.}


% As a stepping stone, let's consider an n'th-degree power series:

% \[p(x;n) = \sum_{i=0}^{n}a_n x^n \]

% We're looking at a weighted sum of monomials. 
% What does this remind us of?\par

% Well, it kinda looks like a linear combination of vectors, doesn't it?\par

% To show the mapping, we can map the natural basis vector:
% \[\vec{e}_i \mapsto x^i \]

% If we agree to this basis,
% our function \(p(x;n)\) is fully described by the coefficients \(a_i\):

% \[p(x;n) = \left[a_1, a_2, \cdots, a_n\right]\]

% So we have \(p(x;n)\) as a vector
% in the vector space of all polynomials
% of degree no greater than \(n\).
% Let's call this vector space of functions
% (or \emph{function space}) \(P^n\).
% Then \(p(x;n) \in P^n\).
% \par


% But what's stopping us from
% letting \(n \rightarrow \infty\)?
% Let's let loose.
% If we do, we get the normal, ``full'' power series\footnote{
% A quick aside on \textbf{power series}\index{power series}:
% when dealing with the ``full'' (infinite-order) power series,
% we should consider radii of convergence if we want our 
% power series to represent a desired fucntion.
% Many vectors \(\{a_i\}\) map to functions that diverge
% pretty much anywhere besides \(x=0\). 
% This may sound exotic, but it pops up in ``mundane'' functions.
% The seemingly innocuous vector 
% \(\left[1, 1, 1, \cdots\right]\) corresponds
% to \(f(x) = \frac{1}{1-x}\) only within a radius of
% convergence of \(\abs{x} < 1\) 
% \textemdash{} outside that radius, the function explodes. If we instead had a vector
% \(\left[1, 2, 3, \cdots \right]\), and our function will
% blow up quite quickly for \(x=0\).\par

% But we needn't always care whether or not the function it
% represents converges if all that interests us 
% are the monomials' coefficients. 
% When we're viewing a power series just as a sequence of
% coefficients 
% (with operations that reflect
% polynomial arithmetic), we refer to it as a
% \textbf{formal power series}
% \index{power series!formal power series}.
% The way polynomial multiplication works allows us to
% solve potentially really tricky problems by creating a
% (finite or infinite) power series with the appropriate
% coefficients, exponentiating the power series,
% and reading off the coefficient of a particular monomial.
% More on that in \ref{sec:generating-functions}.
% }:

% \[p(x) = \sum_{i=0}^{\infty}a_n x^n\]


% When we start playing with infinity, we have to start
% being a bit more clever.
% Our previously used ``vector as an \(n\)-tuple''
% representation of a function \(f\) becomes a \emph{bit}
% unreasonable. Even if we've agreed on our basis mapping
% \(\{e_i \mapsto b_i(x)\}\)
% and we know exactly what the coefficient \(a_i\)
% for the \(i\)'th basis vector is for all \(i \in \domain{N}\),
% we can't literally sit here and list them all out manually:
% \(\vec{f} = \left[a_0, a_1, a_2, \cdots \right]\). \par

% But how about this? 
% Instead of encoding \(f\) into our vector space as a \emph{tuple}
% \(f \mapsto \vec{f} = \left[a_0, a_1, \cdots \right]\), 
% how about we encode 
% \(f\) as \emph{another function \(A_f(k)\) which
% outputs the appropriate coefficient for the \(k\)'th index of
% our (infinite-dimensional) vector}?
% Then if we ever need the value at the \(i\)'th index of our
% encoding, instead of indexing a tuple (\(\vec{f}[i]\)),
% we just call our coefficient-encoding function \(A_f(i)\).
% As long as there \emph{is} an underlying function that
% can give us \(A_f(k)\), we're good. Let's come back to that 
% in a moment.\par

% \subsection{An inner product for our function space.}

% Currently we have a vector space.
% But we \emph{don't} yet have an inner product space.
% What would be a sensible inner product,
% a sensible measure of (norm-sensitive) ``similarity''
% between functions?\par

% Well, these are functions over a variable \(x\).
% Let's hearken back to ye olde days of first-year calculus.
% Back then, we would construct Taylor-series approximations \(T\) 
% of some original function \(f\) around
% a point \(x=c\) and proclaim that \(T\) is ``similar'' to \(f\)
% at and in some neighborhood around that point \(c\). 
% Why? (Let's say \(T\) was a \(k\)'th-order approximation.)
% Well, because we specifically
% constructed \(T\) so that
% \[\forall \ i \in \{0, 1, \cdots, k\}, T^{(i)}(c) = f^{(i)}(c)\]
% \textemdash{} i.e., so that \(T\) matched \(f\)'s slope, concavity, etc., around \(x=c\)
% \textemdash{}
% we expect the \emph{output} of T to be approximately the same
% as the \emph{output} of f around \(x=c\), even for wacky 
% original functions \(f\).\footnote{
%   This breaks down with ``pathological''
%   functions such as
%   \(f(x) = 
%   \begin{cases}
%     0 & x = 0 \\
%     e^{1/x^2} & x \neq 0
%   \end{cases}
%   \),
%   which has \(f^{(i)}(0) = 0 \ \forall i \in \domain{N}\)
%   and therefore has a Taylor series of exactly \(T(x) = 0\)
%   if centered around \(c = 0\) \textemdash{} hardly how \(f\)
%   acts outside the origin!
%   But in this situation we can only grumble about these sorts of functions,
%   called \emph{non-analytic functions},
%   and explicitly exclude them from our analysis.
% }
% All this to say, in choosing the inner product for functions
% in a particular space,
% we probably want to measure the ``responses'' of these
% functions to the input variable(s) they permit.
% Keeping this in mind and looking back at the requirements to
% be an inner product, we can see that a reasonable choice
% is an integral over the domain of \(x\) if \(x\) is continuous:

% \[\innerprod{f}{g} = \int_{Dom(x)} f(x)g(x) dx \]

% or, if \(x\) is a discrete variable, then a summation instead:

% \[\innerprod{f}{g} = \sum_{x_i \in Dom(x)} f(x)g(x) \]

% Note that an inner product space only has one inner product.
% Part of the description of a function space is the domain 
% of the input variable, which would determine which form
% of the inner product would be appropriate.\par

% Before we continue, we should consider what we're going to
% consider the ``domain'' of the input variable for a function.
% More specifically, over what interval should we integrate
% a periodic function: all of \(\domain{R}\) or just one period's
% worth \(T\)?
% The ``better'' option might become clearer if we consider
% the \textbf{metric}\index{metric}, or \textbf{distance function},
% induced by our norm (which was in term induced by our
% inner product).\footnote{
%   Note that the inner product ``induces'' a norm
%   because if the inner product and norm of a vector space
%   are defined independently of each other,
%   we suddenly have inconsistent notions of ``similarity''
%   and ``length''. Likewise for norms and metrics
%   with ``lengths'' and ``distances''.
%   \href{http://people.math.gatech.edu/~heil/books/metricbrief.pdf}{This document}
%   seems promising in explaining why these ``inducements''
%   make sense \textemdash{} we will report back and
%   update this when it's been read.
% }
% When an inner product is defined, the metric 
% induced on our inner product space is

% \[d(x,y) = \norm[x-y] = \sqrt{\innerprod{x-y}{x-y}} \]

% Now, say we have \(f = \cos(t)\) and \(g\) as a sawtooth
% wave of period, amplitude, and phase equal to \(f\)
% (i.e., \(g(t) = Saw(t; T = 2\pi, A = 1, \phi = 0)\).
% What would be the distance between our functions if
% we decide to integrate over all of \(t\) for which
% they're defined?

% \[d(f,g) = \sqrt{\int_{t = -\infty}^{\infty} \left(\cos(t) - Saw(t; 2\pi, 1, 0)\right)^2 dt} \rightarrow \infty \]

% Ouch. I mean sure, the sawtooth doesn't perfectly match
% the sinusoid, but do they really feel \emph{infinitely apart}
% from each other? I don't know about you, but these two
% functions feel a lot closer to each other than, say,
% \(\cos(t)\) and \(\sin(t)\). What can we do to fix it?\par

% Well, when you think about it, \emph{all} of the information
% about both of our functions 
% is contained in the interval \([0, 2\pi)\).
% Really, it would be more appropriate to say that our
% functions are defined over an interval of length \(2\pi\),
% e.g.,
% \(t \mod 2\pi \) (where \(t \in \domain{R}\)), and that
% we're just making copies of our functions outside of
% the defined interval as a ``courtesy'' more than
% anything. If you don't perform the modulo operation,
% you end up with weirdness \textemdash{} if there's only
% \(2\pi\) radians in a circle, how do you get the
% \((x,y)\) coordinate (read: sine or cosine) of the
% point \(3\pi\) radians into the circle? We immediately
% say ``Oh that's just \(-1,0\)'', but that's just because
% we've been sort of ``programmed'' to perform the modulo
% operation without realizing what we're doing.
% And the
% fact that we usually tile \(\domain{R}\) 
% with our periodic function
% reinforces the idea that \(\cos(3\pi) = -1\) makes sense, when
% really what makes sense is that we've defined the cosine function
% over an interval (say \([0, 2\pi)\)), \(3\pi \mod 2\pi = \pi\),
% and \(\cos(\pi) = -1\).\par

% All this to say, it feels like if we want to compare periodic
% functions with identical periods, we should compare
% only over one period (which captures all the ``content''
% of the functions) instead of amplifying the difference an
% infinite number of times over a repeatedly tiled domain:

% \[
% \begin{split} d(f,g) &= \sqrt{\int_{t = 0}^{2\pi} \left(\cos(t) - Saw(t; 2\pi, 1, 0)\right)^2 dt} \\
%   &= \sqrt{4 \times \left(\frac{5\pi}{12} + \frac{4}{\pi} - 2\right)} \\
%   &< \infty
% \end{split}
% \]

% which just feels a whole lot more reasonable.
% In general then, if two functions are periodic with the same period \(T\),
% we would have the inner product be

% \[\innerprod{f}{g} = \int_{t=-T/2}^{T/2} f(t)g(t) dt \]

% with the norm and metric updated accordingly.
% This would seem to break the inner product for aperiodic functions.
% But what is an aperiodic function but a periodic function with
% infinite period \sout{and one less space}?
% Simply let \(T \rightarrow \infty\) in these cases.
% % What about if you have different periods?
% % Well, we want 
% % Well, we want to stay
% % in our vector space, which means some sort of ``average
% % value over all possible phases'' scheme is out of the picture
% % since there is no way to create a phase-shifted periodic function
% % out of a linear combination of basis functions without explicitly
% % adding basis functions unless we just happen to ``luck out''\footnote{
% %   As an example, remember that \(\sin(\omega t + \phi) = \sin(\omega t)\cos(\phi) + \cos(\omega t)\sin(\phi)\),
% %   which is not a linear combination of sines and cosines.
% % }. So we're stuck with the same starting phase for both.
% % Then we'll just take \(T\) to be the maximum of the periods of
% % the functions in question.

% \subsubsection{Aside on cross-correlations.}

% What if \(f\) and \(g\) have different periods? 
% \emph{Then} how would you measure the similarity between them?\par
% % OK, but what if we \emph{did} allow phase shifts?
% % \emph{Then} how would you measure the similarity between two functions
% % if the periods of the two functions \(f\) and \(g\) are different?
% Well, in that case we'd probably want to know how ``similar''
% the two are for a phase shift for each function \(\phi_f\) and \(\phi_g\),
% where we'd have \(\phi_X = 1\) be a phase shift corresponding to a full
% period of the function \(X\).\footnote
% {
%   Note that at the very least for sinusoidal waves, we can
%   describe a phase-shifted periodic function as a linear combination
%   of two non-shifted functions. That is to say, the statement
%   \(\exists b_1, b_2 \left(a\sin(\omega t + \phi) = b_1\cos(\omega t) + b_2\sin(\omega t)\right)\) is true.
% }
% Interestingly, only the \emph{relative} phase shift 
% \(\left(\phi_f - \phi_g\right) \mod 1 := \phi \)
% matters \textemdash{} when you're integrating over a full period,
% it doesn't matter where you start.
% So then we can calculate the ``similarity'' between two periodic
% functions as a function of phase difference between them:

% \[ \begin{split}
%   CC(s; F(s), G(s)) &:= 
%     \int_{\phi = 0}^{1} F(s + \phi) G(s) d\phi \\
%     &= \int_{\phi = 0}^{1} F(s) G(s + \phi) d\phi
% \end{split}
% \]

% % where we define a variable \(s_{\cdot}\) so that for a function \(x(t)\)
% % with period \(T\), \(s_x = \frac{t}{T}\).
% % This normalizes the units so that \(F(s)\) and \(G(s)\) 
% where we've defined \(F\) and \(G\) such that 
% \(F(s) = f(sT_f)\)
% and
% \(G(s) = g(sT_g)\). That is, we \emph{normalize} our input variable
% across all functions so that \(s\) reports how many periods into
% the function we are. 
% This assumes a measure where we weight a differential phase shift
% \(d\phi\) equally between the two functions, despite the same
% magnitude of phase shift requiring ``more'' of the original
% input variable in one function (the function with the larger period)
% than the other.\par

% While I feel the above form makes more intuitive sense (though
% it may take a bit of work to intuitively understand \(s\) and \(\phi\) since
% we need to decouple these variables from the original input variables
% in our minds), 
% we often
% instead see things in terms of periods and time.
% Given two periodic functions with periods \(T_1\) and \(T_2\),
% any arithmetic combination (sum or product) of them yields
% another periodic function of period \(T = \frac{T_1 T_2}{\gcf{T_1, T_2}}\). With this, we can continue:

% \[ \begin{split}
%   CC(t; f(t), g(t)) &:= 
%     \int_{\tau = 0}^{T} f(t + \tau) g(t) d\tau \\
%     &= \int_{\tau = 0}^{T} f(t) g(t + \tau) d\tau
% \end{split}
% \]

% Note that the \(CC\) function is slightly different depending
% on whether we're using our normalized input variable \(s\) or
% our unnormalized input variable \(t\) (they \emph{are}
% from different domains, after all). The latter
% requires some ``tiling'' of our periodic functions in
% \(t\)'s domain
% in order to compare all possible relative phase shifts.\par

% It would be fine if we kept things like this (the form suggests
% an intuition), but often people just immediately jump to
% aperiodic functions and so suddenly take limits to infinity\footnote
% {
%   The most common bounds given are
%   \([0, \infty)\) or \(\domain{R}\) depending on where
%   the function is defined \textemdash{} we'll show the most
%   commonly seen one (over \(\domain{R}\)), though it pains us slightly.
%   The pain is because sending both \(T_f\) and \(T_g\) to \(\infty\)
%   makes it extremely difficult to consider what changes when we
%   shift \(g\) instead of \(f\) \textemdash{} the darn thing
%   looks exactly the same, after all!
% }:

% \[ \begin{split}
%   CC(t; f(t), g(t)) &:= 
%     \int_{\tau = -\infty}^{\infty} f(t + \tau) g(t) d\tau \\
%     &= \int_{\tau = -\infty}^{\infty} f(t) g(t + \tau) d\tau
% \end{split}
% \]

% This is called the \textbf{cross-correlation}\index{cross-correlation}
% operator
% and can be used to determine ``how much'' of an aperiodic 
% function defined over \domain{R} is in another aperiodic function 
% defined over \domain{R}. Hopefully, now that we discussed it
% from its more natural situation with periodic functions,
% it makes more sense! \par

% And I suppose to answer our initial question,
% we would have to integrate over \(s\) (or \(t\)) so that
% our final value is a scalar (i.e. that our inner product maps
% to a field). In \(s\) form:

% \[ \innerprod{f(t)}{g(t)} =
% \int_{s=0}^{1}CC\left(s; F\left(s\right), G\left(s\right)\right) ds = 
% \int_{s=0}^{1} \int_{\phi = 0}^{1} F(s + \phi) G(s) d\phi ds
% \]

% and in \(t\) form:

% \[ \begin{split}
%   \innerprod{f(t)}{g(t)} &= \int_{t=0}^{T} CC(t; f(t), g(t)) dt \\
%   &= \int_{T = 0}^{T} \int_{\tau = 0}^{T} f(t + \tau) g(t) d\tau dt \\
%   &= \int_{T = 0}^{T} \int_{\tau = 0}^{T} f(t) g(t + \tau) d\tau dt
% \end{split}
% \]

% though as far as I'm aware, this value isn't used all that
% often (people seem to usually stick to the autocorrelation). \par

% We can define another very useful operator
% (the \emph{convolution} operator)
% by taking the above equation and tweaking it seemingly innocuously
% \textemdash{} which ends up being the \emph{Hermitian adjoint}
% of the cross-correlation operator\footnote{
%   Something I hope to actually understand the meaning of soon.
% } \textemdash{}
% but that equation comes far more naturally from consider
% \href{https://stats.stackexchange.com/a/332127}{the addition of two random variables}
% and would send us on an (even more) obtuse tangent from our talk of
% vector spaces and inner products. 

% \subsubsection*{Inner product wrap-up.}

% Now this definition of an inner product would be great 
% \textemdash{}
% we'd have a way of describing any function in terms of 
% our desired basis by doing an integral!
% \textemdash{}
% \emph{if}
% we can get it to converge.
% Hardly a guarantee: take \(f(x) = x^0, g(x) = x^2\)
% (both natural unit vectors in 
% our current basis of polynomial space!)
% and \(x \in \domain{R}\) as just one example.
% But let's not give up on it just yet \textemdash{}
% we may just be able to make things work with another basis!

% \subsection{Beyond the countable, and picking a basis.}

% It would help if
% we expanded our horizons a bit.
% I mean, getting to have \(x^i,\ i \in \domain{N}\)
% is great and all, but we're definitely not spanning nearly
% as much of our function space (say the space of smooth
% (i.e., infinitely differentiable) functions, 
% \(\domain{C}^\infty\)) as we could.
% I mean, I can't even fully encode \(x^\pi\) or \(x^e\) 
% with such a puny basis! And I like both \(\pi\) and \(e\)!
% \sout{And especially pie! \textit{Mmmm... Pie.}}\par

% Enough belly-aching. We're now going to expand our
% set of basis functions to be \emph{continuous}:
% \[B_{P+} = \{x^i \mid i \in [1, \infty)\} \] 

% Now we can handle \(x^\pi\) and \(x^e\) quite easily.
% But we still can't describe \emph{every} function (namely,
% any functions which contain \(x^i, i < 1\)).
% And we're probably still in trouble with using our
% inner product \textemdash{} outside of \(\abs{x} < 1\),
% we're probably exploding.\par

% Alright, let's not be so negative! Or wait... 
% Actually, let's be negative!

% \[B_{P} = \{x^{-i} \mid i \in [1, \infty)\} \] 

% Alright, \emph{this} looks promising! Now if we restrict
% the domain of our input variable to 
% something like \(x \in [1, \infty)\),
% our inner product
% won't explode just by taking the norm of a basis vector.
% This is promising, but there may be an even more useful
% set of basis functions to use \textemdash{} it's kind of
% lame to have to start at \(x=1\). What if we're modeling
% something over time? We don't want to just chop
% off the first horizontal unit's worth of data if we 
% could help it! \par

% Before we get carried away, we should realize what these
% changes of basis mean. You'll notice that in our switch
% from \(B_{P+}\) to \(B_{P}\), we lost the ability to
% fully/easily encode \(x^\pi\) and \(x^e\), and we
% had to change our allowed input domain from \((-1,1)\)
% to \([1, \infty)\). So when we change our basis, it
% isn't without consequence. We end up
% \emph{changing the part of function space we can describe}. 
% Put in a way that may sound almost tautological,
% \emph{different sets of basis functions
% describe different (potentially disjoint) sets of functions}.
% This is worth keeping in mind as we finally talk about
% integral transforms.


% \subsection{The integral transform: a concatenation of inner products.}

% So what is an integral transform?
% You could make your own if you wanted to!
% (Whether or not it would be useful is another question.)
% Essentially all it is is \emph{a bunch of inner products
% over a set of previously selected basis functions}.
% As we talked about before, taking inner products of
% a vector \(\vec{v}\)
% with basis vectors \(\{\vec{b_i}\}\)
% and dividing by the norm of the basis vector
% provides us the coefficients needed to describe \(\vec{v}\)
% in the vector space spanned by \(\{\vec{b_i}\}\).
% Going back to functions again, 
% we can say that \emph{the integral transform
% produces from the input function \(f(x)\)
% the vector representation of \(f\),
% in the form of the coefficient-generating function \(A_f(n)\),
% in the function space spanned by a set of basis functions
% \(\{K(x;n)\}\)}. In mathematical form,

% \[A_f(n) = \innerprod{K(x;n)}{f(x)} 
%   = \int_{Dom(x)} K(x;n) f(x) dx \]

% You're free to choose whatever set of basis functions
% \(\{K(x;n)\}\) you'd like, and have the size of this set
% (measured by \(n\)) be whatever you'd like.
% Since the choice of \(K(x;n)\) is at the core of
% the integral transform (it determines what functions you
% can represent and in what way they're represented),
% \(K(x;n)\) is called the 
% \textbf{kernel}\index{kernel!in integral transforms}
% of the transform.\footnote{
%   Which is unrelated to the kernel, i.e. nullspace, of a matrix in linear algebra\par
%   ...Yeah, I know, it would have been nice if they were a
%   bit more creative with the names. 
%   \sout{I mean, ``nullspace''
%   was perfectly good for what it was describing \textemdash{}
%   why'd they have to go and call it the ``kernel'' too?
%   All it succeeded in doing was make me try to understand
%   the connection between these two kernels in vain until
%   I realized there was no connection and they're just
%   named the same thing.}
%   I found it confusing too.
% }
% With a properly restricted input domain and properly
% chosen kernel, the mapping between \(f\) and \(A_f\)
% is one-to-one, implying both the uniqueness of \(A_f\)
% \emph{and} a possible to ``invert'' the transform
% back into the form of the original function!\footnote{
%   Technically, not entirely true.
%   Two functions \(f\) and \(g\) would have identical encodings
%   (\(A_f = A_g\)) if \(f\) and \(g\)
%   differ at only a countable set of points \(S\),
%   i.e., that the Lebesgue measure of \(S\) is 0.
%   This is because we're using an integral as our inner product
%   \textemdash{}
%   we're performing a ``simple'' integral over the input domain,
%   implying the use of the Lebesgue measure,
%   and since \(S\) would has a Lebesgue measure of zero,
%   it doesn't affect the output of the inner product at all.
%   So it isn't foolproof, but it works if you restrict
%   the portion of function space so as to not have
%   these sorts of ``antagonistic'' examples in them.
% }\par

% You'll notice that in the above definition of the transform,
% I didn't divide by the norm of the basis vectors.
% Technically, if we want \(A_f\) to capture the \emph{projections}
% of \(f\) onto the basis functions, we would indeed need
% to divide through by the norm
% (expansion assumes an \(L^2\) norm):

% \[\begin{split}
%   \vec{f} &= \frac{\innerprod{K(x;n)}{f(x)}}{\norm[K(x;n)]} \\
%   &= \frac {\int_{Dom(x)} K(x;n) f(x) dx } {\sqrt{\int_{Dom(x)} K(x;n)^2 dx }}
% \end{split}
% \]

% Personally, I kind of like this form a little bit more, mainly
% because its seemingly out-of-nowhere form might prompt a student
% to ask ``What's up with that denominator?''. Which
% may lead to a discussion about 
% this pretty interesting way of thinking of functions,
% which may help demystify the Transforms\textsuperscript{TM} 
% that otherwise appear to be delivered from on high.
% However, in this case, simplicity prevails.
% See, because the denominator is invariant to the input function
% \(f(x)\), all coefficient-generating functions \(A_f\) for
% \emph{every} function is off by the same factor
% \(1/\sqrt{\int_{Dom(x)} K(x;n)^2 dx}\) at any particular \(n\).
% But if they're \emph{all} off by the exact same factor at the
% exact same locations,
% the coefficient-generating functions will be unique
% whether you correct by that factor or not.
% This is good, because many times the norm will contain
% factors which would complicate the transform's output.
% And so often they don't until/unless they perform an
% inverse transform, 
% at which point they make up for the difference.
% We'll argue through an example of this in a moment.

% \subsection{The Laplace Transform: a differential equations-friendly integral transform.}

% One of the most common integral transforms one comes across,
% often in the context of differential equations,
% is the Laplace Transform, often written with 
% \(t\)'s and \(s\)'s instead of \(x\)'s and \(n\)'s as
% I had done above:

% \[\mathcal{L}(f(t)) = F(s) = \int_{t=0}^{\infty}e^{-st}f(t)dt\]

% Considering this from a function space perspective,
% we're choosing to describe a function \(f(t)\)
% using the set of basis functions \(\{e^{-st}\}\).
% So the kernel of the Laplace transform is \(K(s,t) = e^{-st}\),
% chosen because exponential functions turn on Easy Mode
% for integration/differentiation, making them
% incredibly useful for certain classes of
% differential equations (which anyone who has 
% taken a differential equations course is probably 
% already familiar with). Not only
% that, they have an associated inverse transform!
% (More on that in a moment.)
% The transform gives us the (unscaled)
% coefficient-generating function \(F(s)\),
% which, given an \(s\), computes the (function space's)
% inner product of \(f(t)\) 
% with the appropriate basis vector \(K(s,t)\), and therefore
% the (not properly rescaled) coefficient of \(f\) at the index \(s\)
% in the chosen basis \(\{e^{-st}\}\).\par

% Technically, we haven't fully specified the transform yet.
% As with any transform, the domain of the input variable 
% (in this case, \(t\)) and the basis we've chosen
% (specified by both the kernel \(K(s,t)\) and the domain of
% the ``indexing'' variable, \(s\)) determine which functions
% we can describe, because we need the inner product 
% \innerprod{K(s,t)}{f(t)} to converge.\par

% As it turns out, there are two variations on a theme here:
% imaginary vs complex domains.
% Let's start with the one I personally find more interesting.

% \subsubsection{The Fourier transform: Orthogonal Bases.}

% If we want to transform a function \(f(t)\) that's defined
% for \(t \in (-\infty, \infty)\), we need \(f(t)\) to be a member
% of \(L^1\) space; i.e. that \(f(t)\) be \(L^1\)-integrable; 
% i.e. (less cryptically) we need the \(L^1\) norm of \(f(t)\)
% to converge:
% \[\norm[f(t)]_1 = 
%   \int_{t=-\infty}^{\infty} \abs{f(t)} dt < \infty \]

% If this condition is met, then we would want to constrict
% \(s\) to be an imaginary number, which we'll parametrize
% i.e. \(s \in S = (-i\infty, i\infty)\) 
% where \(i := \sqrt(-1)\).
% To make the path explicit, we can have \(s = i\omega\),
% in which case we'd have \(\omega \in (-\infty, \infty)\).\par

% What does that make our kernel look like?
% In parametrized form, we have \(K(\omega) = e^{i\omega t}\).
% But wait, we know from Euler's formula
% (provable using Taylor series) that
% \[e^{i\omega t} = \cos{\omega t} + i\sin{\omega t} \]

% that is, \(e^{i\omega t}\) involves a rotation around 
% the unit circle in the real-imaginary plane 
% (counterclockwise if \(\omega t\) is positive,
% clockwise if it's negative).\footnote
% {
%   What's especially neat is
%   that it can be shown that the two functions that ``comprise''
%   \(e^{i\omega t}\), 
%   \(\{\cos{\omega t} \mid \omega \in \Omega_{C} = (0, \infty) \}\) and 
%   \(\{\sin{\omega t} \mid \omega \in \Omega_{S} = [0, \infty) \}\) are 
%   \emph{mutually orthogonal} functions, i.e. that
%   for a period \(T\) over the implied resultant waves
%   for the integrands below,
%   \begin{itemize}
%     % \tightlist
%     \item
%       \(\int_{T}\cos{\omega_1t}\cos{\omega_2t} > 0 \implies \omega_1 = \omega_2\)
%     \item
%       \(\int_{T}\sin{\omega_1t}\sin{\omega_2t} > 0 \implies \omega_1 = \omega_2\)
%     \item
%     \(\forall 
%       (\omega_1, \omega_2) \in \Omega_{C} \times \Omega_{S}, 
%     \int_{T}\cos{\omega_1t}\sin{\omega_2t} = 0 \)
%   \end{itemize}
%   So we can view \(e^{i\omega t}\) as ``packaging together''
%   these mutually orthogonal basis functions into one nice,
%   easily integrable/differentiable function.\par

%   The cosine function is even (\(\cos(-x) = \cos(x)\)) and
%   the sine function is odd (\(\sin(-x) = -\sin(x)\)), so
%   changing \(\Omega_{C}\) and \(\Omega_{S}\) to 
%   \((-\infty, \infty)\) would gain no new information.
%   (And in fact, the \emph{cosine} integral transform and
%   \emph{sine} integral transform need not be over \domain{R}.)
%   However, \(e^{i\omega t}\), being the sum of an even function
%   and an odd function, is neither even nor odd and so
%   has no simple symmetry and
%   \(\{e^{i\omega t} \mid \omega \in (-\infty, \infty)\}\)
%   are all linearly independent.
% }
% So when we perform the integral transform:

% \[\fancy{F}(f(t)) = F(\omega) 
%   \propto \innerprod{e^{-i\omega t}}{f(t)}
%   = \int_{t=-\infty}^{\infty} e^{-i\omega t} f(t) dt\]

% (we'll explain the \(\propto\) in a bit)
% we can interpret the (probably complex-valued) coefficient
% ``indexed'' by \(\omega\) as answering the following
% (very long-winded) question:\par

% Let's say
% I coiled output of \(f(t)\), 
% starting from \(t=-t'\) and going until \(t=t'\),
% around the complex unit circle clockwise\footnote{
%   ``Clockwise'' because the exponent of \(e^{-i\omega t}\)
%   has a negative sign at the front.
% }
% with a winding angular frequency of \(\omega\). I'd end
% up somewhere in the complex plane, say at a point \(p(t')\).
% Where would I end up as I use more and more of
% \(f(t)\), i.e., what is \(\lim_{t' \rightarrow \infty} p(t')\)?
% \par

% This doesn't necessarily help us in performing the transform
% itself (called the \textbf{Fourier transform}),
% and it may be harder to wrap one's head around it
% compared to the (potentially) more straightforward
% ``\(F(\omega)\) is the inner product between the input function
% and the basis function indexed by \(\omega\)''.
% But this complex-plane rotation interpretation is
% incredibly important in just about any topic involving
% the \emph{phasor} interpretation of (sinusoidal) waves. 
% For example, the above question
% pegs \(\omega\) as a frequency of rotation. If we observed
% a circuit element's 
% voltage response to a signal over time as \(f(t)\),
% then the Fourier transform of that signal 
% \(\fancy{F}(f(t)) = F(\omega)\) shows how the circuit element
% respond to the \emph{frequencies} of the signal.
% This \emph{transfer function} becomes useful in determining
% what wave content is amplified/preserved/attenuated when
% passing through the element, among other things.\par

% Perhaps more importantly for the moment, this gives us
% an intuitive way of determining the form of the 
% \emph{inverse Fourier transform}.
% Remember, we said that
% by coiling \(f(t)\) around the complex unit circle clockwise
% with a winding frequency of \(\omega\), we got \(F(\omega)\).
% How would we undo such an operation?\par
% Well, how about taking \(F(\omega)\) and rotating it
% \emph{counterclockwise}?
% \[f(t) \propto 
%   \int_{\omega=-\infty}^{\infty} e^{i\omega t} F(\omega) d\omega \]

% We're \emph{almost} there, but notice the use of \(\propto\)
% and not \(=\).
% As it turns out, the proportionality constant is \(1/2\pi\).
% The constant comes out a bit more naturally if one arrives
% at the Fourier transform by taking the limit of
% the Fourier series,
% but we can still make sense of it.
% We'll try to make sense of it in a few ways:

% \paragraph{An appeal to physical intuition: ``natural'' units.}

% In our phasor interpretation of the Fourier trasnform,
% we pegged \(\omega\) as a frequency of rotation.
% More specifically, it is an \emph{angular} frequency of rotation,
% where one full revolution around the unit circle
% (in the real-imaginary plane)
% has a period of \(2\pi\).\par

% Let's continue to ``interpret'' the inputs of our functions
% by considering \(f(t)\). In particular, let's say \(f(t)\)
% is a periodic function in time \(t\) with period \(T\). 
% We already said we're converting to a function based
% on frequency (currently, angular frequency).
% In terms of units,
% the simplest way to convert from time to frequency?
% Well, just inversion, no?

% \[t \mapsfrom \frac{1}{t} := f \]

% where \(f\) is the \emph{temporal frequency} of the function.
% With such a transformation, \(f=1\) corresponds to
% a winding frequency such that a time-interval
% of \(\Delta t = 1\) corresponds to one revolution around
% the unit circle (in the real-imaginary plane).
% It's like there's the same ``density'' between the two units,
% i.e., \(df = dt\).
% That just \emph{feels} right, doesn't it?
% \par

% But as it stands, we \emph{don't} have such a setup.
% Instead we have \(\omega = 2\pi\) corresponding to
% a wrap of \(\Delta t = 1\) around the unit circle.
% So we kind of have to go through ``more'' \(\omega\)
% to wrap all of \(t\). 
% So it's like if we were comparing
% differential amounts of \(t\) with \(\omega\), we'd have

% \[d\omega = 2\pi dt\]

% rather than our cleaner \(df = dt\).\par
% But if we're going back to a domain in \(t\) 
% via an inverse transform, we don't
% want our unit's ``thickness'' to have dilated by \(2\pi\)
% \textemdash{} that'd be a different function (with a period
% larger than our original by a factor of \(2\pi\))!
% So we need to change our thickness of \(d\omega\) to match
% that of \(dt\):

% \[  dt = \frac{d\omega}{2\pi} \]

% and so

% \[\begin{split}
%   f(t) &=
%   \int_{\omega=-\infty}^{\infty} e^{i\omega t} F(\omega) \frac{d\omega}{2\pi} \\
%   &= \frac{1}{2\pi}\int_{\omega=-\infty}^{\infty} e^{i\omega t} F(\omega) d\omega
% \end{split}\]

% A much cleaner way of having this all work out
% is to have our transform map directly to the unit that just
% ``feels'' right, i.e., the temporal frequency.
% That would mean making a \(\Delta f = 1\) correspond to
% our phasor spinning around the unit circle exactly once.
% This is easily done by pulling out the period of \(2\pi\)
% out of our frequency unit and sticking it as a constant
% in the exponennt, i.e., changing our basis function
% for our original transform:

% \[e^{-i\omega t} \mapsto e^{-2\pi ift} \]

% making our Fourier transform be

% \[\fancy{F}(x(t)) = X(f) 
%   = \int_{t=-\infty}^{\infty} e^{-2\pi ift} x(t) df\]

% where we've denoted the function as \(x\)
% to avoid confusion with the temporal frequency \(f\).
% Now because our unit's ``thicknesses'' are equal,
% the ``size'' of the domain of \(t\) and the size of the
% domain of \(f\) are equal, so when we perform the
% inverse transform and integrate over \(f\), we don't
% have to divide our differential by any factor:

% \[x(t) \propto 
%   \int_{f=-\infty}^{\infty} e^{2\pi ift} X(f) df \]

% This definitely is the ``nicer'' option, but it required
% to interpret the math to reach it. Physical intuitions
% of mathematical operations are wonderful, but it would
% be shocking if there weren't a way to get to the answer
% while ``sticking to math''.
% % Well, such methods exist, but it involves considering
% % how to construct unitary operators over a Hilbert space
% % (and the revelation that we usually ``intuitively''
% % think of functions as elements of a Hilbert space).
% % I hope to get the chance to give such explanations
% % once I learn them properly myself.
% % When that's the case, we'll expand this section
% % more mathematical rigor.\par

% % All this to say, inverse transforms are tricky things
% % to find! After all, such a task amounts to inverting
% % an infinite-dimensional transformation matrix \textemdash{}
% % would be a shock if it \emph{were} simple!

% \paragraph{An appeal to norms.}

% Well, what about norms? I made such a big hullabaloo about
% it earlier. Why don't we use them here? \par

% Say our basis function \(e^{-i\omega t}\) is a periodic function
% with period \(T = 2\pi\).\footnote
% {
%   Remember Euler's formula: \(e^{ix} = \cos(x) + i\sin(x)\).
%   The periodicity of the complex exponential \(e^{ix}\) is then
%   the period of the sum of the two sinusoidal functions.
%   Since they both have the same period \(T = 2\pi\),
%   then \(e^{ix}\) does as well.
% } What is its norm? 
% (Remember that for complex numbers, we use an 
% \(L^2\) norm and consider \(z\)'s domain isomorphic to 
% \(\domain{R}^2\), so that \(\abs{z} = \abs{x + iy}
% = \sqrt{\abs{x}^2 + \abs{y}^2}\).)


% \[\begin{split}
%   \norm[e^{-i\omega t}] &= \sqrt{\int_{\omega t=0}^{2\pi} \abs{e^{-i\omega t}}^2 d(\omega t)} \\
%   &= \sqrt{\int_{\omega t=0}^{2\pi} \abs{\cos(\omega t) + i\sin(\omega t)}^2 d(\omega t)} \\
%   &= \sqrt{\int_{\omega t=0}^{2\pi}
%   \left(
%     \left(\cos(\omega t)^2 + \sin(\omega t)^2\right)^{1/2}
%     \right)^2 
%     d(\omega t)} \\
%   &= \sqrt{\int_{\omega t=0}^{2\pi} 1 d(\omega t)} \\
%   &= \sqrt{2\pi}
% \end{split}\]

% This is regardless of one's choice in \(\omega\) or \(t\) and
% so is the same even if the the exponent didn't have the
% negative sign.
% (The result of the integral is
% also not too suprising when you consider the phasor interpretation of the complex exponential function).
% So then our basis functions for \emph{both} the Fourier
% transform \emph{and} the inverse Fourier transform have
% norms of \(\sqrt{2\pi}\). Then it would probably be better
% to divide through by the relevant norm in both directions, no?

% \[\fancy{F}(f(t)) = F(\omega)
%   = \frac{1}{2\pi} \int_{t=-\infty}^{\infty} e^{-i\omega t} f(t) dt\]

% \[\fancy{F}^{-1}(F(\omega)) = f(t)
%   = \frac{1}{2\pi} \int_{\omega=-\infty}^{\infty} e^{i\omega t} F(\omega) d\omega\]

% Dividing through by the norm has had the added consequence
% of preserving inner products of two functions \(f\) and \(g\)
% before and after our transform, i.e., 
% \(\innerprod{f}{g} = \innerprod{\fancy{F}(f)}{\fancy{F}(g)}\).
% This makes the transform as we defined it just now a
% \textbf{unitary operator}\index{unitary operator} over
% our function space (which, through our choice of inner product,
% is a Hilbert space and so has a notion of ``unitary operator'').
% Intuitively, a unitary operator doesn't squash or stretch
% the norms of the objects it operates on. For a more easily
% graspable example, the linear transformation \(T\)
% on \(\domain{C}\)
% corresponding to a rotation in the counterclockwise direction
% by an angle \(\phi\) is an example of a unitary operator.
% If we find another unitary operator \(T^{-1}\) 
% that ``undoes'' \(T\), then everything ends up exactly where
% it used to be. And that's exactly what we just did,
% but with functions 
% over an uncountably infinite-dimensional vector space!
% Pretty neat, huh?

% \subsubsection{The Laplace transform: just a slightly augmented Fourier transform.}

% Now, not all functions \(f(t)\)
% are \(L^1\)-integrable over their domain
% at first, so the Fourier transform wouldn't work on those. 
% But what if we actively tried to squish our 
% function down with our basis functions? Specifically,
% let's squash it by an exponentially decaying function.
% That would mean allowing \(s\) of our basis functions 
% to have a real component to them and so be complex.
% Say \(s = r + i\omega\)
% All we'd need to do is choose an \(r\) large enough to
% make the following integral converge:

% \[ \fancy{L}(f(t)) = F(s) = \int_{t \in Dom(t)} e^{-(r + i\omega)t} f(t) dt \]

% This would be our \textbf{Laplace transform}. It amounts
% to the same sort of winding action described in the Fourier
% transform, except that we're also squashing all the vectors
% we eventually sum (and so the resultant vector)
% by a factor of \(e^{-rt}\). Since it's so similar to
% the Fourier transform, we can use the same idea as
% before to get the inverse transform: wind the output
% of the forward transform in the opposite direction
% (i.e., clockwise). We just will also need to de-squashify
% our values by now stretching each of our points by
% \(e^{rt}\), but besides that, we'd be set to migrate
% over our inverse transform from our analysis of the Fourier
% transform:

% \[ \fancy{L}^{-1}(F(s)) = f(t) = \frac{1}{2\pi} \int_{\omega = -\infty}^{\infty} e^{i\omega t} e^{rt} f(t) d\omega \]

% where we have that factor of \(1/2\pi\) because of the reasons
% we've discussed before.\par

% Some people like to be real fancy and describe the above
% as a path integral in the complex plane that covers the
% same path we've implicitly described above:

% \[ \fancy{L}^{-1}(F(s)) = f(t) = \frac{1}{2\pi i} \int_{s = r - i\infty}^{r + i\infty} e^{st} f(t) ds \]

% You'll notice a factor of \(1/i\) pop out of seemingly
% nowhere.
% Well, when we integrate over a domain \(D\),
% our resulting value is expressed with the domain we integrated
% over as our ``reference point''. This works great if
% we're integrating over one domain and going to
% that same domain (say, integrating over \(\omega \in \domain{R}\)
% and ending up with a function of \(t \in \domain{R}\)).
% But if we integrate along the imaginary axis, our result
% will be in reference to an axis \emph{perpendicular} to
% our target domain of \(t \in \domain{R}\), specifically
% to an axis that is oriented \(\pi/2\) radians counterclockwise
% compared to the axis we desire. So we stick
% a \(1/i = i^{-1}\), a root of unity corresponding
% to a clockwise rotation of a point in the
% complex plane by\(\pi/2\) radians 
% when thought of as an operator (and applied via multiplication),
% to orient our output to the axis we want, that is, the
% real numbers.
% \\
% \subsection{Exiting the rollercoaster.}
% Well, that was quite a ride.
% And this was to derive a relatively \emph{straightforward}
% inverse transform \textemdash{}
% inverse transforms are tricky things
% to find! After all, such a task amounts to inverting
% an infinite-dimensional transformation matrix
% (set by our choice of basis functions and therefore our choice
% of kernel).
% It would be a shock if such a 
% monumental task \emph{were} simple!\par

% In any case, I think that's enough for now.
% \\
% \\
% - DK, 6/19/18


% \paragraph{An appeal to (and amendment of) norms in function spaces over complex numbers.}

% Another way to think about it is to go back to the idea of
% dividing through by the norm of a function.
% We previously defined our norm in function space to
% be over all values for which the input is defined, which led
% to functions defined on \(\domain{R}\) to have the following norm if it converges:

% \[\norm[f] = \innerprod{f}{f} = \sqrt{\int_{t=-\infty}^{\infty} \abs{f}^2 dt}\]

% The choice of this sort of \(L^2\) norm over our function space
% means that the function space we're working with is a
% \textbf{Hilbert space}\index{Hilbert space}, an inner product
% space that reflects many phenomena of reality when
% interpreted in physics/engineering.\par
% Now, we run into a problem if we try to calculate
% the norm of a periodic function, say \(e^{i\omega t}\):\footnote{Remember that, staying consistent
% with the \(L^2\) norm, a complex
% number has magnitude

% \[\abs{z} = \abs{x + iy} = \left(\abs{x}^2 + \abs{y}^2\right)^{1/2} \]}


% \[\norm[e^{i\omega t}] = \sqrt{\int_{t=-\infty}^{\infty} \abs{e^{i\omega t}}^2 dt} \rightarrow \infty \]

% But that just feels wrong, doesn't it? This ``length'' doesn't
% really capture how this function responds to its input \(t\).
% It makes even less sense if we compare the distance between
% two basis vectors 
% \(f(t) = e^{i\omega_2 t}, g(t) = e^{i\omega_1 t}\):\footnote
% {
%   The \textbf{distance}\index{distance!in inner product spaces}
%   between two vectors is implicitly defined when an
%   inner product is defined, via

%   \[d(x,y) = \innerprod{x-y}{x-y} \]
% }

% \[d(f, g) = \sqrt{\int_{t=-\infty}^{\infty} \abs{a - b} dt} \rightarrow \infty\]


% or even the distance between two constant functions 
% \(f(t) = a, g(t) = b\):

% \[d(f, g) = \sqrt{\int_{t=-\infty}^{\infty} \abs{(a - b) dt}} \rightarrow \infty\]

% I mean, sure, they're not the same vector, but are they really
% \emph{infinitely} far apart?\par

% This suggests a change in how we define our inner product,
% because it's clearly busted at the moment.

% % ... my period argument doesn't really work...



% but we can still make sense of it
% by considering the effect the basis functions have
% on the ``amplitude'' of the output.
% We'll keep our discussion below about the
% inverse transform (so we're integrating over \(\omega\)).\par

% Our inner product of \(\innerprod{f}{e^{i\omega t}}\)
% is ``overshooting'' the length of the projection 
% \(\norm{\vec{f}_{f \parallel e^{i\omega t}}}\) by
% \(\norm{e^i\omega t}\). Remember that by Euler's formula,

% \[ e^{i\omega t} = \cos{\omega t} + i\sin(\omega t) \]

% Now, sadly, it won't be as simple as calculating a norm 
% for each individual basis function indexed at \(\omega_i\),
% \(\norm{K(t; \omega_i)}\), and dividing
% our inner product by this ala 
% \(\vec{f}_{f \parallel K(t;\omega_i)} = 
% \frac{\innerprod{f}{K(t;\omega_i)}}{\norm{K(t; \omega_i)}\), 
% as

% \[\int_{x=-\infty}^{\infty} \abs{e^{ix}} dx \rightarrow \infty \]

% The solution above would deal with each kernel's contribution
% to the overshooting individually, and by dealing with all
% of them \emph{individually}, we would have fully corrected for
% the effect of the kernel on our function.\par

% But! Since our kernel is a periodic function,
% we can calculate a root-mean-square value
% for it, and the result happens to be invariant
% to its input \(\omega\) or \(t\)
% (assuming they're nonzero):

% \[\begin{split}
%   RMS(e^{i\omega t}) &= \sqrt{\frac{1}{T}\in}
% \end{split}
% \]

% ???
% show that integral over one period is sqrt(2pi)
% (and also T = 2pi)
% so for any omega, the interval's output is
% stretched by sqrt(2pi)
% equal probability that an f(t) falls on one
% point on e^jwt vs some other point on e^jwt
% average over all functions 3^jwt and you get out
% the norm, sqrt(2pi)


% Consider the contribution of each basis function
% The likelihood that a particular point f(t)
% falls on 






% See, as mentioned before, when we performed our inner products,
% we didn't normalize by the norm of the basis vector we're
% using, meaning we're ``overshooting'' the actual coefficient
% that we'd obta So let's normalize our basis vectors \(e^{-i\omega t}\).
% Let's keep in mind that the norm of our basis function
% is supposed to describe by how much the inner product
% ``overshoots'' the coefficient
% First, let's remember Euler's formula:
% \[ e^{-i\omega t} = \cos{-\omega t} + i\sin(-omega t) \]

% So we're dealing with a periodic function. Then
% an integral over all of its domain 
% The reason is the lack of parity between \(t\) and \(\omega\),
% and the need to reach \(2\pi\) in order to complete a single
% winding.

% Earlier, we described \(\omega\) as an \emph{angular} frequency,
% while \(t\) is regular old time.
% One lengt
% From physics, we have that \(\omega = 2\pi/T\).
% A period of \(T=1\) 
% Let's remember that the number of windings around the unit
% circle is \(\omega t\). Now, if 

% This comes from having not dividing our inner products
% out by the norms of our kernels. Now, the \(L^2\) norm
% of the kernel for the ``forward'' transform
% \[\left(\int_{t=-\infty}^{\infty} \abs{e^{i\omega t}}^2 dt\right)^{1/2} \]

% and for the inverse transform
% \[\left(\int_{\omega=-\infty}^{\infty} \abs{e^{i\omega t}}^2 d\omega\right)^{1/2} \]

% might seem like nightmares to compute.
% But the above expression is
% for the root-mean-square of a periodic function,
% meaning that the integral's value over all of the domain
% is equal to its value over a single period.
% Invoking Euler's formula and solving yields
% \(\sqrt{\frac{2\pi}{\omega}}\)
% and
% \(\sqrt{\frac{2\pi}{t}}\)
% respectively.
% We would have to divide these through \(F(\omega)\)
% and \(f(t)\) respectively, so the factors in multiplicative
% form would be 
% \(\sqrt{{\omega}\frac{2\pi}}\)
% and
% \(\sqrt{{t}\frac{2\pi}}\)
% .
% But that would ruin
% our beautiful easy-to-differentiate/integrate thing... 
% Can we wiggle our way out of it?
% (Warning: This is going to be somewhat loosey-goosey.
% A more rigorous explanation for the coefficient comes
% from taking the limit of a complex Fourier series.)\par

% % Let's multiply the two factors together and look at it
% % for a bit:
% % \[\(\sqrt{{t}\frac{2\pi}}\) \(\sqrt{{\omega}\frac{2\pi}}\)
% %   = \frac{\sqrt{\omega t}{2\pi}\]

% % Well, the \(1/2\pi\) part is just a constant and not
% % at all a problem.

% Well, as we talked about before, we can just choose to let
% the encoding function \(F(\omega)\) remain ``unscaled'',
% which means we have to account for it when inverting.
% Alright, well both forwards and backwards we need to squash
% by a factor of \(1/\sqrt{2\pi}\), so we'll have a
% factor of \(\left(1/\sqrt{2\pi}\right)^2 = 1/2\pi\) in
% our inverse transform. \par
% What about the \(\sqrt{\omega t}\) term?
% This is where the 
% ``winding'' interpretation comes in handy.
% At 

% ``natural'' basis of function space would probably be {x^i}
% as we can describe pretty much any function 
% (to a pretty close approximation)
% as a Taylor series, which uses a countable subset
% of {x^i}
% And in fact, exp() cos() sin() are all exactly encoded
% in a Taylor series.





% Well, so far we have no way of measuring ``similarity'' between two vectors, where here
% two vectors are similar when they point in the same ``direction'' in vector space.
% Vector spaces, in and of themselves, do not have a requirement that there be such a notion.
% But if we \emph{define} an operation that can gives us a scalar ``score'' of the similarity
% between two vectors, we'd be in business.
% First we'll define an operation that \emph{is} sensitive
% The operation we define, called the \textbf{inner product}, \emph{will} be sensitive
% to vector magnitudes in order to fulfill the requirements needed for som

% inner product & orthogonality, inner product with basis vector -> coefficients

% invert the transformation? Depends on the mapping (may not be one-to-one)

% A set of coordinates in a vector space only have ``meaning'' 
% in as much as your axes have meaning.
% Consider a Cartesian system with two axes \(x\) and \(y\) and coordinates \(\left(x,y\right)\).
% Does the coordinate \(\vec{p} = \left(x_0, y_0\right)\) have any intrinsic meaning?
% Not unless the axes do, right?
% So far, all it ``means'' is that \(\vec{p}\) ``goes'' an amount \(x_0\) in the x-direction
% (whatever \(x\) is)
% and an amount \(y_0\) in the y-direction (whatever \(y\) is).
% But we \emph{give} meaning to the axes \(x\) and \(y\).
% Say that we designate an \emph{origin}, with coordinates (0,0).
% Then we can assign the meanings:
% \begin{itemize}
%   \tightlist
%   \item
%     \(x \leftarrow \) the horizontal displacement from the origin
%   \item
%     \(y \leftarrow \) the vertical displacement from the origin
% \end{itemize}

% Equivalently, we can assign meanings to \emph{a specific point} along an axis:

% \begin{itemize}
%   \tightlist
%   \item
%     \(\left(1,0\right) \leftarrow \) the point corresponding to being 
%     from the origin by 1 unit to the right
%   \item
%     \(\left(0,1\right) \leftarrow \) the point corresponding to being displacement vertically
%     from the origin by 1 unit
% \end{itemize}

% Then  horizontal and vertical displacements from 
% a particular point, designated the \emph{origin} with coordinates (0,0)
% \textemdash{}
% \emph{then} we can ascribe a particular ``meaning'' from \(\vec{p}\).

% They only have ``meaning'' once we decide that
% Then the coordinates of a vector
% We're changing which way we vectors are the simplest
% to represent. A vector 
% The ``simplest'' basis to work with is an
% orthogonal (and orthonormal) basis consisting of
% one-hot vectors, i.e., a set of vectors \(\{e_1, e_2, \cdots, e_n\}\)
% where
% \(e_i[j] = \delta(i,j)\),
% but it's by no means the only one.
% \par

% OK, but how do we change bases, i.e., how do we get from the 


% But what does it mean to be orthogonal? So far, the word is meaningless. In fact,
% the notion of ``similarity'' is meaningless so far as well, outside of saying
% whether or not two vectors are linearly dependent or linearly independent. as we only
% have a way of determining linear dependence/independence.
% For that matter, we don't 
% However, if we define some operation to measure some sense of ``similarity'' 
% between points in our vector space,
% % In fact, saying that \(\vec{v} := \left[v_1, v_2, \cdots, v_n\right]^T \) means
% % nothing without saying 

% % \[v `  \]

% \subsection{Functions, function spaces, and operators.}



\chapter{The graveyard.}

This is where some nontrivial amount of work
went into a writeup before realizing that I was
basically confusing \emph{myself} with how I was going
about things.


\section{The Fourier and Laplace Transforms.}



% good source (unfortunately discovered in the middle of the process of thinking up/writing up
% an explanation): https://www.youtube.com/watch?v=zvbdoSeGAgI

% Laplace trasnform *as phasors*!?
% https://www.youtube.com/watch?v=6MXMDrs6ZmA




General flow is:

\begin{enumerate}
  \item
    Explore the Taylor series in terms of linear algebra (countably infinite basis vectors).
    Consider where and how it lacks power (discrete vs continuous).
    From this discrete case, extend to a continuous case
    (\emph{un}countably infinite basis vectors!)
  \item
    Define \emph{linear independence} for vectors in function space, 
    and use this to determine what our basis vectors can be.
    Massage out pseudo-Laplace transforms.

\end{enumerate}

\subsection{Revisiting the Taylor series.}

Previously, we've discussed the Taylor series centered around a fixed point \(a\):
\begin{equation}
\begin{split}
  f(x) &\approx \sum_{n=0}^{k}\frac{f^{(n)}(a)}{n!}(x-a)^n  \\
       &= \sum_{n=0}^{k}F_D(n; a) (x-a)^n \text{ where } F_D(n;a) = \frac{f^{(n)}(a)}{n!} \\
       &= p(x;a,k)
\end{split}
\end{equation}\label{equation:taylor-series}

The rationale for the expression we choose for \(F_D(n;a)\) 
is that \(p(x;a,k)\) will have identical \(i\)'th order derivatives with \(f(x)\) at \(a\)
for \(k \in \{0, 1, \cdots, k\}\), i.e., that
\[p^{(i)}(x=a) = f^{(i)}(x=a) \ \forall \ i \in \{0, 1, \cdots, k \}    \]

So as \(k\) becomes larger and larger, \(p(x;a,k)\) acts more and more like \(f(x)\)
at \(x=a\),
which we hope makes it ``act'' more and more like \(f(x)\)
in a neighborhood around \(a\).\footnote
{
  This doesn't always work perfectly for all functions. For example,
  \(f(x) = 
  \begin{cases}
    0 & x = 0 \\
    e^{1/x^2} & x \neq 0
  \end{cases}
  \)
  has \(f^{(i)}(0) = 0 \ \forall i \in \mathbb{N}\), 
  so its Taylor series centered at \(x=0\) yields \(p(x) = 0\),
  which certainly doesn't capture how \(f\) acts outisde the origin!
}
Letting \(k \rightarrow \infty\) makes \(p(x;a,k)\) ``act'' \emph{exactly}
like \(f(x)\) at \(x=a\).
For some functions \(f\), 
we can show that \(p(x) = f(x) \ \forall x\in \mathbb{R}\) for \emph{all} 
inputs of the function
by taking the limit of \href{https://math.stackexchange.com/a/2136695}{Lagrange remainders}.
The most notable functions for this are \(f(x) \in \{e^x,\cos(x),\sin(x)\}\).\par

% But what if we looked at the Taylor series a different way, through the lens of linear algebra?
% To make notation a bit less cumbersome, let's have \(F_D(n) := F_D(n; a = 0)\) and
% \(
%   p_T(x) = \lim_{k\rightarrow \infty}P(x;a=0,k)
%         =\lim_{k\rightarrow \infty}\sum_{n=0}^{k}F_D(n) x^n
% \).\par

\subsection{Discrete function spaces.}

Now let's go on a seemingly random \sout{secant} tangent.\par
Consider some random polynomial \(p(x;k)\) of order no greater than \(k\).
We could describe it as weighted sum of monomials, where the coefficients
of the monomial \(x^n\) is given by the function \(A(n)\):

\[p(x;k) = \sum_{n=0}^{k}A(n) x^n\]
What does this remind us of?\par
Well, it kinda looks like a linear combination of vectors, doesn't it?
\\
\\
Consider a \(k\)-dimensional vector space with the basis
\(B = \{e_n \mid n \in \{0, 1, \cdots, k\}\}\).
If we wanted to describe some vector in this space, we could do so as a linear
combination of the basis vectors:
\[ \vec{v}_A = \sum_{n=0}^{k}A(n) e_n = [A_0, A_1, \cdots, A_k]\]

That alone is kind of boring. What becomes more interesting is if we can create an
\emph{isomorphism} between this
(to be honest, kinda bland) \(k\)-dimensional vector space and something
more interesting. If we think of a neat, valid isomorphism for our basis vectors,
some other neat consequences should follow. \par
Well, what if we make the mapping \(e_n \leftrightarrow x^n\) (for \(x \in \mathbb{R}\))?
This is a valid mapping, because we can describe the set of
\(\{x^n \mid n \in \{0, 1, \cdots, k\}\}\) as linearly independent, in that
there's no way to make a weighted sum of other monomials \(x^b, b\neq n\)
identical to \(x^n\) for all \(x\) in our domain of interest.\par
Let's keep this going. What would \(\vec{v}_A\) mean on the other side of our
isomorphism?
\[ \vec{v}_A = [A_0, A_1, \cdots, A_k] = \sum_{n=0}^{k}A(n) e_n
  \longleftrightarrow \sum_{n=0}^{k}A(n)x^n = p_A(x)  \]

where \(p_A(x)\) is the (no greater than \(k\)'th-order) whose coefficients
are enumerated by \(A(n)\). This means that if we agree on a valid ``encoding'' process
\textemdash{}
in this case, that we'll describe \(p_A(x)\) as a weighted sum of monomials \(x^n\)
\textemdash{}
then we can go from function to vector and vice-versa.
When we let \(k \rightarrow \infty\), we're allowing our vectors, described by \(A(n)\),
to describe the space of all possible polynomials \(P^\infty\).\par




Worth noting is that there's no reason why we \emph{have} to choose the function \(x^n\)
in our mapping
\(e_n \leftrightarrow x^n\)
. We can choose \emph{any} function where it makes sense to say
that the functions for different \(n\) are linearly independent of each other.
For example, we could choose to have the mapping
\[e_n \leftrightarrow e^{nx}\]
since the set 
\(\{e^nx \mid n \in \{0, 1, \cdots, k\}\}\)
is linearly independent.\footnote
{
  In fact, since \(e^nx = \left(\sum_{k=0}^{\infty}\frac{x^k}{k!}\right)^n\), we could view
  \(e^{nx}\) as a(n oddly) weighted combination of \(x^0, x^n, x^{2n}, \cdots\),
  implicitly describing an infinite-degree polynomial with potentially only a finite
  number of coefficients.
}
The more-or-less complete freedom in choosing our basis mapping will prove useful,
as some ``encodings'' (i.e. some sets of basis functions)
will be more useful in some contexts than others.
The main caveat is that subset of function space implied by your basis mapping
is always limited as to what functions it can
``encode'' fully and/or concisely 
\textemdash{} if you choose the mapping \(e_n \leftrightarrow e^{nx}\),
there is no \(A(n)\) such that \(\sum_{n=0}^{k}A(n)e^{nx} = x \), so we're definitely not
spanning \(P^\infty\) with this new mapping and if you're interested
in simple finite-degree polynomials, you're probably better off
with our earlier encoding.\footnote
{
  Worth noting is that trying to create a mapping that can recreate \emph{all}
  functions is not the best idea.
  Recall that if we designate an input \(x\) in some domain \(D_x\)
  and some output \(y\) in some domain \(D_y\),
  then a \textbf{function}\index{function} \(f \in \mathcal{P}(D_x \times D_y)\)
  is simply a set of ordered pairs \((x,y)\)
  where each \(x\) appears only once in the set.
  (Here \(\mathcal{P}\) denotes the \emph{powerset} operator
  and \(\times\) the \emph{Cartesian product} operator.)
  So there are \emph{many} wacky functions one could make and basically no
  hope in accounting for all of them in a simple-to-describe basis.
}

\subsection{Equipping our vector space.}

% Now here's something trickier: how do we map a \textbf{metric}, or a function describing
% the ``distance'' between two points, so that it captures how close a function is
% to a point outside our function subspace? After all, not all functions can be described as
% a sum of the natural-order monomials \(x^n, n \in \mathbb{R}\), so how do we choose
% the point that matches it the best? Well, that depends on what we're aiming for.
% If we only care about the polynomial \(p\) associated with our function
% having identical derivatives to the target function \(f\) at some point \(x=a\),
% we could have the metric associated with each point be 
% \(\sum_i|f^{(i)}(a) - p^{(i)}(a)|\)
Here's something trickier: how do we perform a projection on a function that's 
outside our subspace? That is to say, how would we actually \emph{calculate} the
function \(A(n)\)?\par
Well, so far we haven't touched on that. We've defined a vector space, but we
haven't defined a \emph{norm} \(|\cdot|\) or
an
\emph{inner product} \(\left<\cdot, \cdot\right>\) for our vector space meant to represent,
in this case, polynomials.
% http://mathworld.wolfram.com/InnerProduct.html
% http://mathworld.wolfram.com/SchwarzsInequality.html
which 
we'd want to have capture ``how much'' of one vector is along in another vector would
be used to calculate our coefficients \(A(n)\) by supplying \(e_n\) as one of the
inputs to our inner product.
Well, that depends on what our metric is for considering two
functions to be ``close''
% \begin{verbatim}
%   Need to figure out how metrics change.

%   ...?
% \end{verbatim}



\subsection{Re-revisiting the Taylor series.}

Let's consider the Taylor series again,
where we'll set \(a:=0\) and let \(k \rightarrow \infty\):

\begin{equation}
  \begin{split}
    f(x) &\approx \sum_{n=0}^{\infty}F_D(n) x^n \text{ where } F_D(n) = \frac{f^{(n)}(0)}{n!} \\
         &= p(x)
  \end{split}
\end{equation}

Relating the Taylor series to our vector-space isomorphism,
setting \(A(n) := F_D(n)\) picks a point \(\vec{p}\) in our vector space \(P^k\) that corresponds
to the polynomial \(p\) which is ``closest'' to \(f\)
in the sense that \(p\)'s value and first \(k\)'th derivatives agree with \(f\)'s at \(x=0\).
It's almost like \(p\) is \(f\)'s ``\emph{projection}'' onto \(P^k\) under our chosen
definition of ``closeness''.\par
Our point \(p\) \textemdash{} and as such, our function that captures the
coordinates of \(p\), \(A(n)\) \textemdash{}
definitely has to be related to our definition of ``closeness''.
In this case, the metric we're minimizing is
\(\sum_i|f^{(i)}(0) - p^{(i)}(0)|\).
If our definition of closeness was instead, say, that they have identical definite integrals
\( \int_{x=0}^{1}f(x)dx = \int_{x=0}^{1}p(x)dx \),
we'd have a different ``closest'' point in our vector space.
This also shows that the metric of ``closeness'' we use determines whether there is a
\emph{unique} \(A(n)\) for \(f(x)\) in a given function space. If we're considering the space
of polynomials, there is only one point that satisfies the ``equal derivatives at \(a\)''
constraint, but many that satisfy the ``equal definite integral over \([0,1]\)'' constraint.



% % Let's say if can find some isomo
% % where the basis vector
% % \(e_n\) corresponds to the coefficient of \(x^n\) for
% % \(n \in \{0, 1, \cdots, k\}\).
% % The set \(B = \{e_n \mid n \in \{0, 1, \cdots, k\}\}\)
% % is in fact a valid basis,
% % because there's no way to make a weighted sum of other monomials \(x^b, b\neq n\)
% % identical to \(x^n\) for all \(x\)
% % and therefore no way to create one basis vector from a linear combination of the others.
% % Then \(B\) forms a basis for the vector space in question.
% Let's call this vector space \(P^k\), for the space of all \textbf{p}olynomials with order
% no greater than \(k\) with domain \(\mathbb{R}\).\par
% If \(P^k\) contains all \(k\)'th order polynomials, how is
% \(p(x;k)\) represented?
% It's represented by the (zero-indexed) vector whose \(n\)'th index equals \(A(n)\) for all
% \(n \in \{0, 1, \cdots, k\}\), because in such a case, the vector is described by
% \[ \sum_{n=0}^{k}A(n) e_n = \vec{v} \]
% If we allow \(e_n\) to ``assume'' its identity as the monomial \(x^n\), then we get
% \[\sum_{n=0}^{k}A(n) e_n = \vec{v} \longrightarrow \sum_{n=0}^{k}A(n) x^n = p(x;k)\]

% \(A(n)\) encodes exactly the coefficients needed that get us to \(p(x;k)\).
% It just so happens that if we have some function \(f(x)\), the point in \(P^k\)
% that has identical derivatives to \(f(x)\) at \(x=0\) is the point reached by
% setting the coefficients via \(F_D(n) = \frac{f^{(n)}(0)}{n!}\).
% Letting \(k\) grow larger means that we're choosing a polynomial
% from a larger and larger space of polynomials
% (which is itself a subset of all possible functions\footnote
% {
%   Recall that if we designate an input \(x\) in some domain \(D_x\)
%   and some output \(y\) in some domain \(D_y\),
%   then a \textbf{function}\index{function} \(f \in \mathcal{P}(D_x \times D_y)\)
%   is simply a set of ordered pairs \((x,y)\)
%   where each \(x\) appears only once in the set.
%   (Here \(\mathcal{P}\) denotes the \emph{powerset} operator
%   and \(\times\) the \emph{Cartesian product} operator.)
%   So there are \emph{many} wacky functions one could make.
% })






\subsection{But first, let's talk eigenfunctions.}

In order to ``trust'' that the Fourier and Laplace transforms are capturing all of the
``content'' of our input function


\subsection{The Fourier transform and wiggles over time.}

The Fourier transform is the way one maps a function
\(f: t \mapsto y,\ t \in \mathbb{R}, \ y \in \mathbb{R}\)
to its uniquely equivalent function \(F: \omega \mapsto z, \ z \in \mathbb{R}^2\).
Looking at it more explicitly through the lens of linear algebra,
if we say \(f\) came from a space \(\mathcal{F}_t\), i.e., \(f \in \mathcal{F}_t\), and
likewise that \(F \in \mathcal{F}_\omega\), then the Fourier transform \(FT\) is a mapping
\(FT: \mathcal{F}_t \mapsto \mathcal{F}_\omega\).\par

The actual definition for the \textbf{Fourier transform}\index{Fourier transform} is
\begin{equation}
  F(\omega) = \frac{1}{2\pi} \int_{-\infty}^{\infty}e^{-i\omega t}f(t)dt
\end{equation}\label{equation:fourier-transform}


and can be derived from taking 
the limit of the \emph{discrete-time Fourier series} of \(f(t)\) as the function's
period \(T \rightarrow \infty\).\footnote{
  Which we hope to discuss further at some point.
}
The inverse map, aptly named the
\textbf{inverse Fourier transform}\index{inverse Fourier transform},
is 

\begin{equation}
  f(t) = {2\pi} \int_{-\infty}^{\infty}e^{-it\omega}F(\omega)d\omega
\end{equation}\label{equation:inverse-fourier-transform}

Here we use \(i = \sqrt{-1}\).
(We may flip-flop between \(i\) and \(j\) to denote \(\sqrt{-1}\).
It should be clear from context.)\par

OK, cool, but what does it mean? Well, first let's remember one of the neatest things
you can prove using Taylor series, i.e. \emph{Euler's formula}\index{Euler's formula}:

\[e^{ix} = \cos(x) + i\sin(x) \]

% Also, the set of sine and cosine functions (and in fact any two sinusoids that are not
% identical) are orthogonal, i.e.,
% Also, any two non-identical sinusoidal functions are orthogonal, i.e.,
Also, note that
% \(\sin(\omega t)\) and \(\cos(\omega t)\)
\(\sin(x)\) and \(\cos(x)\)
% TODO: define orthogonality more clearly 
% (integral is zero in product with anyone except itself)
are mutually orthogonal\footnote{
  That is to say,
  % \[\int_{-1/\omega}^{1/\omega} \cos(\omega t)\sin(\omega t)dt = 0 \]
  \(\int_T \cos(x)\sin(x)dx = 0 \)
  over their aggregate period \(T\).
  To be proven later. See in the meantime
  \href{http://tutorial.math.lamar.edu/Classes/DE/PeriodicOrthogonal.aspx}{this resource}.
},
and in fact all sinusoidal functions
\(\{\cos(ax) \mid a \in \mathbb{R}\}\) and \(\{\sin(bx) \mid b \in \mathbb{R}-\{0\}\}\)
are mutually orthogonal.\footnote{
  When \(b=0\), \(\sin(bx) = \sin(0) = 0 \forall x\) and so doesn't fit the definition 
  of orthogonality since \(\int_{-\infty}^{\infty}sin(0x)sin(0x)dx = \int_T 0 dx = 0 \ngtr 0\).
}
That means that if we were to describe functions in a vector space, then each 
sinusoidal function \(\cos(ax), a \geq 0\) and 
\(\sin(bx), b > 0\) 
would be linearly independent from one another.\footnote{
  Only one ``half'' of \(\mathbb{R}\) is needed to capture all non-redundant information
  as \(\cos(-x) = \cos(x)\) and \(\sin(-x) = -\sin(x)\).
}
\par

All this points to a \emph{different basis} on which we can describe functions.
The origina
% every point in F(\omega) has some information about f across *all* of t
% it's like having an infinite ``power series'' built
% using the orthogonal functions cos(wt) and sin(wt).
% We've already established that cos(wt) and sin(wt) are orthogonal,
% So the only parts of f(t) that contributes to F(w) is the part that would
% correspond to cos(wt) (or sin(wt)) in a ``frequency series'' of f.
% I.E., if we could say that f(t) = 1 + a_1cos(wt) + a_2cos(2wt) + ...
% then F(w) = \int cos(wt)f(t)dt = \int cos(wt) \times a_1cos(wt)dt



% It's important to remember that this is a transformation of \emph{function to a function
% with a different input space}, and not a transformation of our coordinate system.
% There is no clear way of s

% Considering that functions are themselves maps (from their input space to their output space),
% the Fourier transform is a map of maps.
% This may sound like voodoo magic, 
% That is to say, the Fourier transform is a mapping for



\subsection{The Laplace transform as a patch for the Fourier transform.}

What if the Fourier transform explodes?
For example, consider the function






% \appendix

% \chapter{Weird things about integrals.}\label{apx:weird-things-about-integrals}

% In our derivation of the Boltzmann distribution, we reached the point where to derive
% the exponential form, we needed to assume independence of energies among the particles 
% in the system, an assumption which, 
% as mentioned in the footnote at \ref{foot:independence-weirdness},
% is a bit crazy-making (by exceeding the total energy of the system).
% It can be even weirder than that when we think about what we consider the differential
% \(dE\) to be.
% \\
% In standard analysis, differentials are not numbers, they are operators.

% And if we don't subscribe to the tenets of
% \emph{nonstandard analysis}\index{nonstandard analysis} wherein
% there exist hyperreals which are larger than the countable infinity \(\aleph_0\)
% and whose reciprocals are differentials (a potential extension of
% number theory that is realized
% if one takes \(\neg G\) from \emph{Godel's incompleteness theorem}
% to be an axiom), then things become even \emph{more} ridiculous.
% % [adapt the stuff below]
% % Or, if we stick to talking about the discrete energy steps, if we say there are an
% % infinite number of them, then no matter how large an energy step \(i\)
% % we assign to each particle, \(\sum_{i=1}^{N} E_idE = 0 \neq E_T \), 
% % so energy in the system completely disappears and all particles have
% % \emph{zero} energy! Unless we have an \emph{infinite} number of particles.
% % In which case, we would have that
% % \(\lim_{N\rightarrow \infty}\sum_{i=1}^{N} E_idE = E_T \), but still 
% % These are the questionable outcomes that com

    % Add a bibliography block to the postdoc
    
    



    % show index
      % add index to table of contents
      \clearpage % flush all floating figures, "officially" start on next page
      \addcontentsline{toc}{chapter}{Index}  % add entry to ToC
    \printindex % actually print the index
    
\end{document}
